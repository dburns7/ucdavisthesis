\chapter{Conclusions}

The discovery of the Higgs boson was the highlight of the physics results of Run 1 at the LHC. In addition to measuring the properties and couplings of the Higgs, efforts were launched in Run 2 to use the Higgs as a probe for new physics, including to search for dark matter. In parallel with these studies, progress was made by complementary, non-collider searches. Although no direct detection has been confirmed using any approach, large areas of parameter space have been excluded. Collider searches, including the search for the mono-Higgs signature, have much better sensitivity to low-mass dark matter than direct searches, as well as the ability to study higher order couplings that are not accessible to direct searches. The main results of this dissertation are (1) the rediscovery of the Higgs boson with Run 2 data and (2) the cross section limits on two dark matter models, contributing to the world-leading low-mass limits and mediator mass exclusions for these models.

The impact of this study is very high, being among the first-ever searches for the mono-Higgs signature. Several key contributions were made to the standard model Higgs search in the four-lepton final state, including (1) a cross-check exercise to validate the event selection with other groups and (2) the definition of a new event category for Higgs candidate events with large missing transverse momentum. The key ingredient that this study added to the Standard Model Higgs search and collaboration documentation is the study of missing energy, in particular, whether the background and statistical modeling techniques remain valid in the high missing energy regime. 

Numerous extensions of this analysis would have a large impact on the field. The selection used to identify events with a Higgs boson can be reoptimized for mono-Higgs signals rather than for the Standard Model Higgs signals. This could significantly increase the limit-setting sensitivity. The limits found with the four-lepton final state can be combined with those from the other Higgs decay channel analyses to obtain even stronger results. This work is underway. There are additional models that predict a mono-Higgs signature that can be studied with the datasets currently available. The most commonly studied models are those that predict very hard missing energy spectra, which gives an advantage to other Higgs decay channels. However, the four-lepton channel would have better sensitivity for the unstudied models with softer spectra. Finally, a more detailed study can be done to understand the relationship of the limits found here with those set by other search strategies in order to guide the design of future analyses and experiments. 

There is strong motivation for the existence of a coupling between the dark sector and ordinary matter. Studies such as this one shed light on the nature of this interaction. Technology and analysis techniques are constantly advancing. If such an interaction exists, it is only a matter of time before dark matter is observed, expanding our understanding of the mass-energy content of the universe beyond the mere 5\% that we currently know.
