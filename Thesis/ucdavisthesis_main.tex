\documentclass[12pt]{ucdavisthesis}
%\documentclass[12pt,draftcls]{ucdavisthesis}

% PLEASE READ THE MANUAL - ucdavisthesis.pdf (in the package installation directory)

%%%%%%%%%%%%%%%%%%%%%%%%%%%%%%%%%%%%%%%%%%%%%%%%%%%%%%%%%%%%%%%%%%%%%%%%
%                                                                      %
%               LATEX COMMANDS FOR DOCUMENT SETUP                      %
%                                                                      %
%%%%%%%%%%%%%%%%%%%%%%%%%%%%%%%%%%%%%%%%%%%%%%%%%%%%%%%%%%%%%%%%%%%%%%%%

%\usepackage{bookmark}
\usepackage[us,nodayofweek,12hr]{datetime}
\usepackage{graphicx}
\usepackage{caption}
\usepackage{subcaption}
\usepackage{rotating}
\usepackage{hypernat}
\usepackage{adjustbox}
\usepackage{hyperref}
%\usepackage{abntcite}
\usepackage{amsmath}
\usepackage{multirow}
\usepackage{caption}
\usepackage{supertabular}
%\usepackage{fontspec}
\usepackage{graphicx,xspace}
\usepackage{bm}
\usepackage [english]{babel}
\usepackage [autostyle, english = american]{csquotes}
\MakeOuterQuote{"}

\usepackage[ansinew]{inputenc}
\usepackage[T1]{fontenc}
\usepackage{libertine}
\input{glyphtounicode}
\pdfglyphtounicode{f_f}{FB00}
\pdfglyphtounicode{f_f_i}{FB03}
\pdfglyphtounicode{f_f_l}{FB04}
\pdfglyphtounicode{f_i}{FB01}
\pdfgentounicode=1


%\usepackage[square,comma,numbers,sort&compress]{natbib}
%\usepackage{hypernat}
% Other useful packages to try
%\usepackage{amsmath}
%\usepackage{amssymb}
%
% Different fonts to try (uncomment only fontenc and one font at a time)
% (you may need to install these first)
%\usepackage[T1]{fontenc} %enable fontenc package if using one of the fonts below
%\usepackage[adobe-utopia]{mathdesign}
%\usepackage{tgschola}
%\usepackage{tgbonum}
%\usepackage{tgpagella}
%\usepackage{tgtermes}
%\usepackage{fourier}
%\usepackage{fouriernc}
%\usepackage{kmath,kerkis}
%\usepackage{kpfonts}
%\usepackage[urw-garamond]{mathdesign}
%\usepackage[bitstream-charter]{mathdesign}
%\usepackage[sc]{mathpazo}
%\usepackage{mathptmx}
%\usepackage[varg]{txfonts}


\usepackage[toc,page]{appendix}

\hyphenation{dis-ser-ta-tion blue-print man-u-script pre-par-ing} %add hyphenation rules for words TeX doesn't know


%\renewcommand{\rightmark}{\scriptsize A University of California Davis\ldots \hfill Rev.~\#1.0 \quad Compiled: \currenttime, \today}
% a fancier running header that can be used with draftcls options

%%%%%%%%%%%%%%%%%%%%%%%%%%%%%%%%%%%%%%%%%%%%%%%%%%%%%%%%%%%%%%%%%%%%%%%%
%                                                                      %
%        DOCUMENT SETUP AND INFORMATION FOR PRELIMINARY PAGES          %
%                                                                      %
%%%%%%%%%%%%%%%%%%%%%%%%%%%%%%%%%%%%%%%%%%%%%%%%%%%%%%%%%%%%%%%%%%%%%%%%

\title          {A Collider Search for Dark Matter Produced in Association\\
                 with a Higgs Boson with the CMS Detector at the 13 TeV LHC}
%Exact title of your thesis. Indicate italics where necessary by underlining or using italics. Please capitalize the first letter of each word that would normally be capitalized in a title.

\author         {Dustin Ray Burns}
%Your full name as it appears on University records. Do not use initials.

\authordegrees  {B.S. (Georgia Institute of Technology) 2011 \\
                 M.S. (University of California, Davis) 2012}
%Indicate your previous degrees conferred.

\officialmajor  {Physics}
%This is your official major as it appears on your University records.

\graduateprogram{Physics}
%This is your official graduate program name. Used for UMI abstract.

\degreeyear     {2017}
% Indicate the year in which your degree will be officially conferred.

\degreemonth    {June}
% Indicate the month in which your degree will be officially conferred. Used for UMI abstract.

\committee{Michael Mulhearn}{Robin Erbacher}{Albert De Roeck}{}{}
% These are your committee members. The command accepts up to five committee members so be sure to have five sets of braces, even if there are empties.

%%%%%%%%%%%%%%%%%%%%%%%%%%%%%%%%%%%%%%%%%%%%%%%%%%%%%%%%%%%%%%%%%%%%%%%%

\copyrightyear{2017}
%\nocopyright

%%%%%%%%%%%%%%%%%%%%%%%%%%%%%%%%%%%%%%%%%%%%%%%%%%%%%%%%%%%%%%%%%%%%%%%%

\dedication{\textsl{To my teachers.}}

%%%%%%%%%%%%%%%%%%%%%%%%%%%%%%%%%%%%%%%%%%%%%%%%%%%%%%%%%%%%%%%%%%%%%%%%

\abstract{
The study presented in this dissertation is a search for dark matter produced in 13 TeV proton-proton collisions at the Large Hadron Collider (LHC) using $\usedLumi$ of data collected in 2016 with the Compact Muon Solenoid (CMS) detector. Dark matter escapes the detector without interacting, resulting in a large imbalance of transverse momentum, which can be observed when a Higgs boson is tagged in the opposite direction. A variety of models which motivate a dark matter and Higgs interaction are discussed. The experimental signature of these models is called mono-Higgs. 

In this search, the Higgs is produced primarily from gluon fusion and decays to four leptons via two $Z$ bosons ($\HZZfl$). In addition to observing the Higgs in the four-lepton final state, an extensive study of missing transverse energy (MET) is required to search for the mono-Higgs signature. 
A background model is developed for the Standard Model processes that result in the same final state as the signal, then a counting experiment is performed in an optimized signal region. There is no evidence for an excess of events in the signal region above the backgrounds, so cross section limits are set for two kinematically distinct signal models.
}

%%%%%%%%%%%%%%%%%%%%%%%%%%%%%%%%%%%%%%%%%%%%%%%%%%%%%%%%%%%%%%%%%%%%%%%%

\acknowledgments{
I would like to thank the numerous groups and individuals who have facilitated my academic journey. 

First, the teachers and professors who made a particularly strong impact on my education or personal and professional development: my graduate adviser Michael Mulhearn, Professors Mani Tripathi and Robin Erbacher from UC Davis, Professors David Finkelstein, Dierdre Shoemaker, and Yuri Bakhtin from Georgia Tech, Professors Yosi Balas, Catherine Davis, and Imad El-Jeaid from Middle Georgia College, and Scott Hammond, Kate Hoppenrath, Cherie McAdams, Terasa Martin, Vicki Dobbs, and Bonnie Rary from Dacula High School. Lastly, Peg Alton from Dacula Middle School, who convinced me to join the gifted program, the first step on my journey of academic accomplishment. Even a small conversation can make a very large impact.

Second, I would like to thank my primary collaborators in CMS, Giorgia Miniello and Nicola De Filippis, who helped make the completion of this analysis possible. I am honored to be a member of the same team that discovered the Higgs boson in the four-lepton final state, and I am grateful for having access to their expertise, feedback, and documentation. It has been a very rewarding process to watch the different mono-Higgs analysis teams grow and succeed. I am grateful for having the chance to work with the analyzers on all of the mono-Higgs teams. Harrison Prosper, Shalhout Shalhout, and Eiko Yu have always made themselves available to help with technical questions for my analysis. I am also appreciative of the research and personal support I have received from Francesca Ricci-Tam, Wells Wulsin, Rachel Yohay, and Justin Pilot. 

I am lucky to have had a great group of friends in my cohort, including Karen Ng, Henry Stoltenberg, Dusty Stolp, Chad Flores, and James Morad, who have encouraged and supported me throughout graduate school. Kyle Tos, Christine McLean, and Mengyao Shi have been exceptional friends and office-mates. I have relied heavily on the help and friendship from students in my lab, including Zhangqier Wang, Chris Brainerd, and Brandon Buonacorsi.

Finally, I would like to thank my family: my parents David Burns and Tina Liston, for giving me the opportunity to succeed, and my wife, Kayla Burns, for providing me with stability and support throughout college, and for thoroughly editing this document. Appa and Dax Burns have helped me work through many research issues on our evening walks.
}

%%%%%%%%%%%%%%%%%%%%%%%%%%%%%%%%%%%%%%%%%%%%%%%%%%%%%%%%%%%%%%%%%%%%%%%%

% Each chapter can be in its own file for easier editing and brought in with the \include command.
% Then use the \includeonly command to speed compilation when working on a particular chapter.
%%% \includeonly{ucdavisthesis_example_Chap1}

\begin{document}

\newcommand{\Pg}{\ensuremath{g}}
\newcommand{\Pe}{\ensuremath{e}}
\newcommand{\PH}{\ensuremath{H}}
\newcommand{\Pgm}{\ensuremath{\mu}}
\newcommand{\cPq}{\ensuremath{q}}
\newcommand{\PW}{\ensuremath{W}}
\newcommand{\cPZ}{\ensuremath{Z}}
\newcommand{\Z}{\ensuremath{Z}}
\newcommand{\cPqt}{\ensuremath{t}}
\newcommand{\cPqb}{\ensuremath{b}}
\newcommand{\ttbar}{\ensuremath{t\bar{t}}}
\newcommand{\fbinv}{\ensuremath{\rm{\ fb}^{-1}}}
\newcommand{\pt}{\ensuremath{p_{\rm{T}}}}
\newcommand{\ET}{\ensuremath{E_{\rm{T}}}}
\newcommand{\MET}{\ensuremath{E_{\rm{T}}^{\rm{MISS}}}}
\newcommand{\JPsi}{\ensuremath{J/\Psi}}
\newcommand{\keV}    {\ensuremath{\mathrm{ke\kern -0.1em V}}}
\newcommand{\MeV}    {\ensuremath{\mathrm{Me\kern -0.1em V}}}
\newcommand{\GeV}    {\ensuremath{\mathrm{Ge\kern -0.1em V}}}
\newcommand{\TeV}    {\ensuremath{\mathrm{Te\kern -0.1em V}}}

\newcommand{\bibfont}{\singlespacing}
% need this command to keep single spacing in the bibliography when using natbib
\newcommand{\ggH}{\ensuremath{\Pg\Pg\to\PH}}
\newcommand{\qqH}{\ensuremath{\cPq\cPq\to\cPq\cPq\PH}}
\newcommand{\WH}{\ensuremath{\PW\PH}\xspace}
\newcommand{\ZH}{\ensuremath{\cPZ\PH}\xspace}
\newcommand{\ttH}{\ensuremath{\cPqt\bar{\cPqt}\PH}}
%\newcommand{\ttH}{\ensuremath{\cPqt\bar{\cPqt}\PH}\xspace}
\newcommand{\bbH}{\ensuremath{\cPqb\bar{\cPqb}\PH}\xspace}
\newcommand{\qqZZ}{\ensuremath{\cPq\bar{\cPq}\to\cPZ\cPZ}}
\newcommand{\ggZZ}{\ensuremath{\Pg\Pg\to\cPZ\cPZ}}
\newcommand{\HZZfl}{\ensuremath{\PH\to\cPZ\cPZ\to4\ell}}
\newcommand{\Hllll}{\ensuremath{\PH\to4\ell}}
\newcommand{\mH}{\ensuremath{m_{\PH}}}
\newcommand{\mllll}{\ensuremath{m_{4\ell}}}
\newcommand{\mpllll}{\ensuremath{m'_{4\ell}}}
\newcommand{\ptllll}{\ensuremath{p_{\mathrm{T},4\ell}}}
\newcommand{\KD}{\ensuremath{{\cal D}^{\rm kin}_{\rm bkg}} }
\newcommand{\VDj}{\mathrm{{\cal D}_{\rm jet}} }
\newcommand{\DMeVbfjj}{\ensuremath{{\mathcal D}_{\rm 2jet}}\xspace}
\newcommand{\DMeVbfj}{\ensuremath{{\mathcal D}_{\rm 1jet}}\xspace}
\newcommand{\mll}{\ensuremath{m_{\ell\ell}}}
\newcommand{\mlplm}{\ensuremath{m_{\ell^{+}\ell^{-}}}}
\newcommand{\DMeWh}{\ensuremath{{\mathcal D}_{\rm \WH}}\xspace}
\newcommand{\DMeZh}{\ensuremath{{\mathcal D}_{\rm \ZH}}\xspace}
\newcommand{\DQgjj}{\ensuremath{{\mathcal D}^{\rm q/g}_{\rm 2jet}}\xspace}
\newcommand{\DQgj}{\ensuremath{{\mathcal D}^{\rm q/g}_{\rm 1jet}}\xspace}
\newcommand{\DCombVbfjj}{\ensuremath{{\mathcal D}^{\rm comb.}_{\rm 2jet}}\xspace}
\newcommand{\DCombVbfj}{\ensuremath{{\mathcal D}^{\rm comb.}_{\rm 1jet}}\xspace}
\newcommand{\DCombWh}{\ensuremath{{\mathcal D}^{\rm comb.}_{\rm \WH}}\xspace}
\newcommand{\DCombZh}{\ensuremath{{\mathcal D}^{\rm comb.}_{\rm \ZH}}\xspace}
\newcommand{\sip}{\ensuremath{\text{SIP}_\text{3D}} }
\newcommand{\Zee}{\mathrm{\cPZ\rightarrow\Pe^{+}\Pe^{-}}}
\newcommand{\Zone}{\mathrm{Z}}
\newcommand{\MassD}{\mathrm{{\cal D}_{\rm mass}} }
\newcommand{\MassDprime}{\mathrm{{\cal D}'_{\rm mass}} }
\newcommand{\LikMass}{\mathcal{L}_{3D}^{m,\Gamma} }
\newcommand{\LikMassTwoD}{\mathcal{L}_{2D}^{m,\Gamma} }
\newcommand{\LikMassOneD}{\mathcal{L}_{1D}^{m,\Gamma} }
\newcommand{\LikMuOneD}{\mathrm{{\cal L}_{1D}(m_{4l})} }
\newcommand{\LikMuTwoD}{\mathrm{{\cal L}_{2D}(m_{4l},\KD)} }
\newcommand{\LikMuOneDPrime}{\mathrm{{\cal L}_{1D}(m_{4l}')} }
\newcommand{\LikMuTwoDPrime}{\mathrm{{\cal L}_{2D}(m_{4l}',\KD)} }
\newcommand{\muV}{\ensuremath{\mu_{\mathrm{VBF},\mathrm{V\PH}}} }
\newcommand{\muF}{\ensuremath{\mu_{\Pg\Pg\PH,\,\ttbar\PH}} }

\newcommand{\usedLumi}{35.9\fbinv}
%\newcommand{\usedLumi}{35.9\fbinv}
\newcommand{\posOfPValueMinimum}{12X.X}
\newcommand{\expSignAtMinimum}{xx.x}
\newcommand{\obsSignAtMinimum}{yy.y}
\newcommand{\expSignAtRunIMass}{10.5}
\newcommand{\obsSignAtRunIMass}{10.8}
\newcommand{\valMuAtRunIMass}{\ensuremath{1.04^{+0.19}_{-0.17}}}
\newcommand{\valMuVAtRunIMass}{\ensuremath{0.00}^{+1.37}_{-0.00}}
\newcommand{\valMuFAtRunIMass}{\ensuremath{1.20}^{+0.35}_{-0.31}}
\newcommand{\valMass}{\ensuremath{12X.XX^{+0.xx}_{-0.yy}}}
\newcommand{\valMassThreeDRefit}{\ensuremath{125.25 \pm 0.20 (\mathrm{stat.}) \pm 0.08 (\mathrm{sys.})}}

\bibliographystyle{unsrturl}
%\bibliographystyle{unsrtnat}
%many other bibliography styles are available (IEEEtran, mla, etc.). Use one appropriate for your field.

\makeintropages %Processes/produces the preliminary pages

%\chapter{test}
%\newcommand{\usedLumi}{35.9}
$\usedLumi$

\chapter{Motivation and theoretical background}

The first section of this chapter lays the theoretical framework of the Standard Model (SM) of particle physics, including the historical discovery timeline of symmetries of nature, each corresponding to a leap forward in our physical understanding, and the model's particle content and their properties. The second section extends the symmetry group of the SM to include candidates for the observed cosmic dark matter (DM) and the particles that could mediate the DM-SM interactions.

\section{The Standard Model}

\subsection{Symmetries of nature}

Each major advance in the history of physics has corresponded to the discovery and mathematical implementation of a new global space-time, global discrete, or local gauge symmetry. In this section, I will review these discoveries, informally developing the mathematical framework needed to understand the symmetry groups of the SM. Along the way, two important subplots will play out: the development of our understanding of physics at smaller distance scales and higher energies, and the unification of previously separate physical sectors.

\indent Our ancient ancestors were aware that certain geometrical shapes, e.g. the Platonic solids, possessed the quality of symmetry, and were driven to understand the composition of physical substances by breaking them down into fundamental, indivisible units. These units, known as atoms by the ancient Greeks, interact and rearrange themselves according to physical laws to account for the variety of substances and physical phenomena we observe. The attraction to applying symmetry to nature is evidenced by the centuries-long belief that the Earth lie at the center of the universe, with the celestial bodies orbiting in perfect, divine circles.

\indent The end of the scientifically repressive middle ages brought along an improvement in astronomical observations and the growth of the pseudo-scientific field of alchemy, which attempted to reduce, understand, and manipulate the fundamental elements of physical substances. When Kepler discovered three laws of planetary motion, he unified the description of the motion of the planets. For the first time, the conservation of a physical quantity, what we now know as angular momentum, was associated with a general set of physical objects. At the burgeoning of the scientific revolution, Newton championed the idea that the same physical laws can be applied to all physical events, and that properties of these laws can be abstracted to apply to nature at a fundamental level.

\indent Newton defined an inertial reference frame implicitly as one where his first law held, that is, that an object remains at constant motion unless acted on by an outside force. Since his laws were the same in all inertial reference frames, a new symmetry of nature was discovered, now called symmetry under Euclidean transformations. Since Euclidean transformations form a mathematical group, it is said that classical mechanics is invariant under the Euclidean group. The invariance under the Euclidean transformations can be used to derive conservation laws: Newton's laws don't depend on spatial translations or rotations, implying the conservation of linear and angular momentum, respectively. These relationships foreshadow Noether's abstraction of the connection between symmetries and conserved quantities. She proved that there is a conserved quantity, or current, associated with every symmetry of a physical system. This famous theorem facilitates the derivation of conserved quantities and will be used extensively in the theories that follow. The extension of the Euclidean transformations to include time translations and motion at constant velocity (boosts) forms the Galilean group.

\indent The next symmetry of nature to be discovered came when Lorentz found that Maxwell's equations, which unified the classical eletricity and magnetism sectors, were invariant under a set of transformations that generalized the classical Galiliean translations, now called Lorentz transformations, which form the Lorentz group. Einstein

\indent Maxwell's equations unify EM, invariant under Lorentz group, conservation of charge/currents/etc? 

\indent Einstein developes special relativity, deriving Lorentz group as fundamental symmetry of nature

\indent Euclidean group plus Lorentz group is Poincare group, global space-time symmetry of SM. Conservation of ? 

\indent Quantum theory developed to describe both bosons and fermions, with respective statistics and permutational invariance

\indent Weyl applies gauge invariance in quantum theory, electric charge conserved

\indent Dirac incorporates relativity in quantum theory of spin 1/2 particles. Invariant under Poincare group, predicts antiparticles discovered in CR by Carl Anderson

\indent T, C, P, PCT, P breaking observed by Wu, Yang, Lee, CP breaking obseved by Cronin, Fitch

\indent Spontaneous symmetry breaking, Higgs mechanism

\indent Electroweak unification

\indent Gell-mann Neeman SU(3) symmetry from strong force
 
\indent Standard model

\subsection{Particle content}



\section{Dark matter}

\subsection{Observational evidence}

\subsection{Beyond the Standard Model}

\cite{Carpenter:2013xra}
\cite{Gibaldi:80}

\chapter{Experimental apparatus}

This chapter gives an overview of the experimental apparatus used to collect the data analyzed in this dissertation: the Large Hadron Collider (LHC) and Compact Muon Solenoid (CMS) detector. The first section reviews the design and performance of the LHC. The second section reviews the design of CMS, its component subdetectors, and data acquisition system.  

\section{Large Hadron Collider}

This section reviews the construction and original design specifications of the LHC \cite{1748-0221-3-08-S08001}, leading up to the 7-8 TeV center of mass (COM) energy collisions recorded from March 2010 to February 2013 (Run 1), the upgrades and repairs made to the LHC and its pre-accelerators during Long Shutdown 1 (LS1) from February 2013 to April 2015, and finally, the 13 TeV COM energy collisions recorded from April 2015 through 2016. 

\indent Due to budgetary and logistical concerns, the LHC is located in the repurposed Large Electron-Positron (LEP) collider tunnel, constructed in the 1980s by the European Organization for Nuclear Research (CERN), which continues to operate the LHC accelerator facilities and whose laboratory hosts the staff, scientists, and engineers running the machines and detectors associated with it. Located beneath the border of Switzerland and France near Geneva, the LEP tunnel consists of eight straight sections and eight arced sections, totalling 26.7 km, at depths varying from 45 m to 170 m. Two 2.5 km transfer tunnels connect the main LEP tunnel to the rest of the CERN complex, which includes a series of pre-accelerators that increase the energy of ionized hydrogen gas protons to 450 GeV before they are injected into the LHC. In addition to repurposing the tunnel, the underground caverns at Points 2 and 8, which were built for LEP, are used for the ALICE and LHCb experiments, the two specific purpose experiments, designed to study quark-gluon plasma in heavy ion collisions and the matter-antimatter imbalance, respectively. The facilities at Points 1 and 5, where the general-purpose ATLAS and CMS experiments are located, were built new for the LHC. 

\indent Although the length of the LEP tunnel is sufficient for the LHC, the diameter of the tunnel and the geometry of the straight and arced sections are suboptimal for a proton-proton accelerator. Since sychrotron radiation emission is not as much of a problem for a protons, the LHC would ideally have longer arced sections. The two counter circulating particle-antiparticle beams of LEP could occupy the same pipe, being curved by the same magnets, but with an inside diameter of only 3.7 m, the tunnel is too narrow to accommodate the two pipes needed for counter circulating proton-proton beams, necessitating the use of the "two-in-one" super-conducting twin bore magnet design. The LHC beam is steared by 1232 8 T, superconducting dipole twin bore magnets, which are cooled by a system of NbTi Rutherford cables to below 2 K. This technology is essential to the LHC operation, but comes at the cost of a higher sensitivity to instabilities in the operation temperature, which may cause the magnet to quench, or lose its superconductivity and current.

\indent The LHC was designed to explore physics at the EW symmetry breaking scale, with a nominal COM energy for collisions of 14 TeV, and search for rare events produced by physics beyond the SM, with a target luminosity of $10^{34} \rm{cm}^{-2} \rm{s}^{-1}$, both the highest ever produced. For a general physics process, the rate of event production is given by
\begin{equation}
N = \sigma \times L \propto \sigma \times n_b N_b^2 f_{rev} \gamma
\end{equation}
where $\sigma$ is the process cross section, $L$ is the LHC luminosity, which is proportional to $n_B$, the number of bunches per beam, $N_b^2$, the number of particles per bunch, $f_{rev}$, the beam revolution frequency, and $\gamma$, the relativistic gamma factor. Consequently, to achieve higher event rates for rare processes, both high beam intensities and high beam energies are required. To search for rare events, such as H production, the basic strategy for designing the LHC was to maximize these luminosity parameters within the budgetary, engineering, and physical limitations, of which there are many. Combining these constraints yields nominal values of 2808 bunches per beam, $1.2\times10^{11}$ protons per bunch, and a revolution frequency of 11245 turns per second. The luminosity decays over a given run with a lifetime of $\tau \approx 15$ hours, due primarily to losses in particle intensity from collisions, and must periodically be dumped and refilled with an average turnaround time of around 7 hours. The integrated luminosity is the integral of the luminosity as a function of time $L(t) = L_0 / (1+t/\tau)^2$ over a run of length $T_{run}$ given by
\begin{equation}
L_{int} = L_0 \tau (1-e^{-T_{run}/\tau})
\end{equation}
where $L_0$ is the initial luminosity. If the LHC runs for 200 days per year with a peak luminosity of $10^{34} \rm{cm}^{-2} \rm{s}^{-1}$, the maximum total integrated luminosity, or sum of the integrated luminosity of all runs is about $80 \rm{fb}^{-1}$ per year. Due to unforeseen setbacks and inefficiencies in collecting data at the detectors, the total integrated luminosity collected by the experiments is far less than the maximum, totalling around $20 \rm{fb}^{-1}$ each from ATLAS and CMS in the whole Run 1, and about $2 \rm{fb}^{-1}$ each in 2015. 

\indent The LHC machine was designed to attain a per beam energy of 7 TeV, resulting in COM collisions of 14 TeV, but an accident during beam energy ramp-up in September 2008, caused by a faulty electrical connection between two magnets and resulting in the damage of numerous magnets, resulted in delays [http://press.cern/press-releases/2008/10/cern-releases-analysis-lhc-incident]. As a result, the Run 1 beam energy was set to 3.5 TeV and later increased to 4 TeV, for 7 and 8 TeV collisions. LS1 began at the conclusion of Run 1, and consisted of a two year period of maintenance and upgrades, including consolidating and repairing interconnections between about 500 magnet cryostats, adding shielding and relocating various electronic equipment, and upgrades to the LHC's ramp up accelerators [http://home.cern/about/updates/2013/02/long-shutdown-1-exciting-times-ahead]. It was decided that Run 2 would proceed with beam energies of 6.5 TeV instead of the originally planned 7 TeV in the interest of time, since it would have taken longer to retrain the magnets to not quench below currents required for 14 TeV than it would to retrain them for 13 TeV [http://home.cern/about/engineering/restarting-lhc-why-13-tev]. Overall, the LHC has performed and continues to perform at a very high level, supplying the experiments with beam collisions within the desired luminosity ranges. 

\section{Compact Muon Solenoid}

This section reviews the design and performance of the CMS detector \cite{1748-0221-3-08-S08004}, including its general layout, subdetector systems, and trigger and data acquisition (DAQ) systems. CMS was designed to explore physics at the TeV scale, recording collisions from the LHC proton beams at their crossing place at Point 5, near Cessy, France. The detector is multi-purpose, in that it is sensitive to detecting a wide array of new physics signatures, but the primary purpose is to validate or refute the Higgs mechanism as being responsible for EW symmetry breaking. This goal was accomplished in Run 1, so Run 2 looks forward to searching for physics beyond the SM, including signatures from new symmetries such as SUSY, extra dimensions, and dark matter. Additionally, CMS is disigned to record collisions of heavy ion beams to study QCD at this energy scale. CMS is distinguished from other general purpose detectors by its high magnetic field solenoidal structure, silicon-based inner tracker, and crystal scintillator electromagnetic calorimeter. 

\indent The primary challenges in designing CMS include: accounting for the pileup of inelastic collisions on each event with sufficiently high granularity detectors and small timing resolutioin, ensuring all electronics and detector components can withstand the high radiation exposure, and triggering on the roughly $10^9$ events per second to filter out interesting events to a rate managable by the read out and computing systems. The design requirements can be summarized as follows: good muon identification and charge determination, good charged-particle momentum resolution in the inner tracker, good EM energy resolution, good diphoton, dimuon, dijet, and dielectron mass resolutions, efficient photon and lepton isolation, and good MET measurement. All of these requirements will be addressed in the remainder of this chapter.

\indent With an overall length of 21.6 m and an outer diameter of 14.6 m, the cylindrical shape of CMS is divided into two regions, the barrel and endcaps, with the coordinate system centered at the collision point near the center of the cylinder. The standard coordinate definitions have the x-axis pointing inward toward the center of the LHC, the y-axis pointing upward, and the z-axis in the beam direction in a right-handed manner. The polar coordinates $r$ and $\phi$ are measured in the x-y plane, transverse to the beam, where the transverse momentum quantity $p_T$ is defined. The missing energy $E_T^{miss}$ is defined as the imbalance in measured $p_T$. The polar angle $\theta$ is measured from the z-axis. A convenient coordinate for relativistic measurements is the pseudorapidity, defined as $\eta = -\ln{\tan(\theta/2)}$. 

\begin{figure}[tbh]
\centering
\includegraphics[width=6in]{figures/cms.jpg}
\caption{Deconstructed view of the CMS subdetectors, with human figure for scale. From inside to out, the colored segments correspond to the following systems: light brown is the pixel tracker, cream is the strip tracker, green is the ECAL, orange is the HCAL, grey is the solenoid, red is the yoke with white muon chambers. }
\label{fig:cms}
\end{figure}

\indent The dominant feature of CMS is the superconducting solenoid, 13 m long and 6 m in diameter, supplying a field of 4 T required to bend charged particles at the energies produced in up to 14 TeV collisions for the momentum and charge measurements. Within and surrounding the solenoid is a series of layered detectors and support structure, a cutout of which is shown in Figure~\ref{fig:cms}. At the center of CMS, surrounding the beam interaction point, is the inner tracker, a combination of 10 layers of silicon microstrip detectors and 3 layers of silicon pixel detectors, which provide the required granularity for high occupancy collisions. The next layer, still within the solenoid bore, contains the calorimeters, first the electromagnetic calorimeter (ECAL), surrounded by the hadronic calorimeter (HCAL). The ECAL uses avalanche photodiodes in the barrel and vacuum photodiodes in the endcaps to read out scintillation light from lead tungstate crystals, produced by charged particle interactions. The HCAL in the barrel uses hybrid photodetectors to read scintillation light from hadronic interactions with the brass/scintillator detector material. The scintillation light is carried to the photodetectors with clear fibres, from wavelength shifting fibres embedded in the scintillator material. The various endcap HCAL systems ensure full coverage for measuring the missing energy. Finally, muon detecting stations are incorporated into and surround the solenoid support structure where the return field is present, including aluminum drift tubes (DTs) in the barrel and cathode strip chambers (CSCs) in the endcaps. The subdetector systems of CMS are covered in greater detail in the remainder of this chapter. 

\subsection{Tracking detectors}

The inner tracking detectors of CMS, supported by a 5.30 m long tube with an inner diameter of 2.38 m suspended from the HCAL barrel, contain 1440 pixel and 15148 strip detector modules, composing the pixel detector and silicon strip tracker, respectively. The detectors are responsible for measuring the trajectories of charged particles, essential to measuring the momenta of particles with energy $>$ 1 GeV in the range $|\eta|<2.5$, and reconstructing secondary vertices and impact parameters, needed to ID heavy flavor particles. Being closest to the beam interaction point (IP), the tracking detectors are subjected to the highest radiation doses, and their material may interfere with the trajectories of primary particles through multiple scattering, bremsstrahlung, photon conversion, or nuclear interactions, necessitating the use of silicon technology. Additionally, due to the high particle flux of around 1000 particles per 25 ns bunch crossing, the detectors must have high granularity to resolve the trajectories of particles reliably, and fast readout times to reduce occupancy from high flux and pileup conditions. 

\indent The detector modules of the tracking detector are shown schematically in Figure~\ref{fig:tracker}. The innermost section, labeled PIXEL, is the pixel detector, composed of 66 million 100 $\mu$m $\times$ 150 $\mu$m pixels on modules layered in three barrels at radii 4.4, 7.3, and 10.2 cm and two disks on each end at $z = \pm34.5, \pm46.5$ cm. A deconstructed barrel pixel module is shown in Figure~\ref{fig:pixmodule}, with the sensor bump-bonded onto read out chips (ROCs) controlled and powered by high-density-interconnect (HDI) boards. When a charged particle passes through a pixel sensor, consisting of n-type pixels implanted on a high-resistance n-type substrate, charge carriers are induced in the conduction band of the substrate. These charge carriers then drift in the 4 T magnetic field to the nearby pixels (called charge sharing), where an analog signal is read out, amplified, and digitized by the ROC. The endcap pixel modules have a similar construction, but with different pixel sensor geometries, called plaquettes. The pixel detector has a resolution of 10-40 $\mu$m, sufficient for the imposed design requirements. 

\begin{figure}[tbh]
\centering
\includegraphics[width=6in]{figures/tracker.pdf}
\caption{Schematic diagram of tracking detectors with radial distance of modules (shown as black lines) from center on the left axis, z-dimension on the bottom axis, and $\eta$ accross the top.}
\label{fig:tracker}
\end{figure}

\begin{figure}[tbh]
\centering
\includegraphics[width=5in]{figures/pixelmodule.pdf}
\caption{Deconstructed barrel pixel module showing module components.}
\label{fig:pixmodule}
\end{figure}

\indent The remaining modules of the pixel detector form the strip tracker. These are divided into the following sections: 

\subsection{Electromagnetic calorimeter}

\subsection{Hadronic calorimeter}

\subsection{Muon detectors}

\subsection{Trigger system}


\chapter{Event reconstruction and simulation}

This chapter overviews the algorithms used to reconstruct the trajectories and identify (ID) types of particles produced in proton-proton collisions in CMS, collectively known as Particle Flow (PF), and how these collisions are simulated. Unless otherwise noted, the material in the first section comes from reference \cite{CMS:2009nxa}.

\section{Particle reconstruction}

The PF algorithms combine information from all of the CMS sub-detectors discussed in the previous chapter to reconstruct the particles produced in the collision event. Since many of the particles produced initially in the collision are unstable, decaying before they have time to interact with the sub-detectors, PF reconstructs the stable particles: electrons, muons, photons, and hadrons. The remaining physics objects of interest, jets, missing energy, taus, etc, can be determined from the information provided by the stable PF-IDed particles. 

\indent The different particles are reconstructed and IDed using information from individual sub-detectors, or combinations of sub-detectors. The direction and momentum of charged particles is measured by the tracker. Electrons are reconstructed using tracks and energy deposits in the ECAL. Muons are reconstructed from a combination of tracker and muon chamber data. Photons are reconstructed from energy deposits in the ECAL. Finally, charged and neutral hadrons are reconstructed from energy deposited primarily in the HCAL, with a contribution from energy deposits in the ECAL. The missing transverse energy, an observable of particular importance to this analysis, used to identify DM that does not interact with the detector material, is the modulus of the sum of transverse momenta of all the PF reconstructed particles. 

\indent The basic pieces of information from the subdetectors used by PF are called elements, and consist of charged-particle tracks, muon tracks, and calorimeter clusters. The tracker provides charged-particle track elements. Since the tracker has the best momentum resolution of the subdetectors, it is of critical importance that the tracking efficiency be nearly 100\%, with as low a fake rate as possible, to reduce an excess in reconstructed energy. These two goals are accomplished using an iterative algorithm: first, tracks are seeded using very tight criteria, yielding a low efficiency, but negligible fake rate, then track seed criteria are loosened and hits that clearly belong to a track are removed, resulting in increasing efficiency. The ECAL and HCAL subsystems (ECAL barrel, HCAL barrel, HCAL endcap, PS first layer, and PS second layer) provide cluster elements. The calorimeter clustering algorithm measures the energy and direction of neutral particles (e.g. photons, neutral hadrons), differentiates energy deposits from neutral and charged hadrons, reconstructs electrons, and contributes to the reconstruction of charged hadrons. The algorithm is summarized as follows: cluster seeds are identified as energy deposit peaks over a given energy, from which topological clusters are grown by appending adjacent cells, and last, topological clusters seed PF clusters. An example is shown in Figure~\ref{fig:pf1} and Figure~\ref{fig:pf2}, where a simple jet is reconstructed into four clusters, shown as dots. 


\begin{figure}[tbh]
\centering
\includegraphics[width=4in]{figures/PFa.pdf}
\caption{Event display of hadronic jet in the x-y plane, with solid arcs at the ECAL and HCAL surfaces. The locations of clusters are given by the solid dots.}
\label{fig:pf1}
\end{figure}

\begin{figure}[tbh]
\begin{subfigure}{0.45\textwidth}
\centering
\includegraphics[width=2.8in]{figures/PFb.pdf}
\caption{}
\end{subfigure}
\begin{subfigure}{0.45\textwidth}
\centering
\includegraphics[width=2.8in]{figures/PFc.pdf}
\caption{}
\end{subfigure}
\caption{Event display of hadronic jet in the $\eta-\phi$ plane for the ECAL (a) and HCAL (b). The locations of clusters are given by the solid dots.}
\label{fig:pf2}
\end{figure}

\indent Once the PF elements are determined, they are linked together into blocks, which correspond to the signatures left in the sub-detectors of a single particle. Single particles typically leave one to three elements. The linking algorithm determines the quality of the link between all pairwise elements in an event, then forms blocks from the highest quality links, starting from the tracker, and proceeding outward through the calorimeters and muon chambers. Once the blocks are formed, PF associates a global event particle with each block. PF muons are formed from global muon candidates if its momentum is consistent across all track elements. PF electrons are IDed from electron candidates using tracker and ECAL cluster variables, accounting for the Bremsstrahlung photons produced when the electron passes through the tracker material. Once the elements associated to PF muons and PF electrons are removed, the remaining elements are analyzed to ID charged hadrons, photons, or neutral hadrons. PF charged hadrons are associated to remaining tracks if the linked clusters are consistent with the measured momenta. If the energy of the linked clusters is much larger than the track momentum, accounting for uncertainties, a PF photon or PF neutral hadron is formed. Any remaning clusters without linked tracks form PF photons or PF neutral hadrons. 

\indent As previously discussed, once the PF particles are identified, additional information about the event can be inferred. A quantity of particular importance to this analysis is the missing transverse energy (MET), defined above. The performance of the PF algorithms determination of the MET is shown in Figure~\ref{fig:pfmetres} by the resolution of PF measured MET as a function of the true MET to be within $\pm5\%$ above 20 GeV.


\begin{figure}[tbh]
\centering
\includegraphics[width=4in]{figures/pfmetres.pdf}
\caption{MET reconstruction resolution.}
\label{fig:pfmetres}
\end{figure}

\section{Monte Carlo event simulation}

The simulation of proton collision events and their detection using Monte Carlo (MC) techniques is useful for several purposes. In addition to using the simulated events to test the detector hardware and software performance without collecting true data, simulations are used to build background models when searching the data for new physics processes. New physics signatures usually appear as excesses in data above a SM background. The background model consists of SM processes which produce the same or similar signature as the new signal being searched for. These processes are modelled either using purely simulated events, or a combination of simulated events and data-driven techniques. In either case, it is often necessary to weight the background events by correction scale factors measured using data, which account for shortcomings of the simulations, such as the inability to perform perturbative QCD calculations for low momentum transfer processes. MC event generation can be factored into two parts: modeling the initial particles produced in a collision event and modeling how these initial particles evolve in and interact with the detector.

\indent The first part of MC event generation is modeling the proton-proton collision and the initial particles produced at the primary vertex. Several software packages are used by CMS analysts to generate collision events and calculate the cross sections of the simulated processes, including PYTHIA \cite{1126-6708-2006-05-026}, MADGRAPH \cite{Alwall2011}, BlackHat and Sherpa \cite{Berger:2009ba}, and POWHEG \cite{Alioli:2010ab}. The packages have different implementations, but the underlying principles are the same. The momenta of the proton partons (quarks and gluons) that interact in the initial scatter are determined probabilistically by random sampling from the parton distribution functions (PDFs), which give the probability that a parton will carry a fraction $x$ of the proton momentum. This is straightforward for processes with two incoming and one outgoing particle ($2 \rightarrow 1$) and two incoming and two outgoing particles ($2 \rightarrow 2$), in which the outcomes are weighted by their relative cross sections and determined probabalistically. For radiative processes such as ISR and FSR of a photon or gluon, generally $1 \rightarrow 2$ processes, higher order matrix elements must be calculated or approximated. Once the initial particles are determined, their fragmentation and decays are simulated in a process called hadronization, until the final stable particles are produced. 

\indent The second part of MC event generation is simulating the detector response to the stable particles produced in the first step, including their interaction with the detector material itself, both active elements and structural material. The primary software package used in this step by CMS is GEANT \cite{documents:998155}, in which a complete digital representation of the CMS detector is built. GEANT simulates the passage of each each stable particle, step-by-step, outward through the detector, probabilistically determining the interaction that occurs at each step depending on the particle's energy,  material it is in, and the EM field present. Since the detector is not perfectly efficienct, both because the acceptance is less than one and the reconstruction efficiency of the individual detector elements is suboptimal, calibration values must be measured at CMS and fed back in to the simulations, so that the performance of the detector can be accurately simulated. Once the final response of the detector is simulated, the resulting MC may be weighted by scale factors measured using real data, in order to correct for mismodelling of the detector.



%\include{ucdavisthesis_Chap4}
\chapter{Physics objects}\label{sec:objects}

\section{Electrons}

\subsection{Electron reconstruction}
\label{sec:eleReco}


Electron candidates are preselected using loose cuts on track-cluster matching observables, so as to preserve the highest possible efficiency while rejecting part of the QCD background. To be considered for the analysis, electrons are required to have a
transverse momentum $p^e_T >$ 7 $\GeV$, a reconstructed $|\eta^e| <$ 2.5, and to satisfy a loose primary vertex 
constraint defined as $d_{xy} < 0.5$ and $d_z < 1$. 
Such electrons are called "loose" electrons.

The early runs in the 2016 data-taking exhibit a tracking inefficiency originating from a reduced hit reconstruction efficiency in the strip detector (``HIP" effect). 
The resulting data-MC discrepancy is corrected using scale factors as is done for the electron selection with data efficiencies measured using the same tag-and-probe technique outlined later (see Section~\ref{sec:eleEffMeas}). 
These studies are carried out by the CMS electron and photon (EGM) physics object group (POG) and the results are summarized in this section.
The electron reconstruction scale factors are shown Figure~\ref{fig:ele_rec_scale_factors} as a one-dimensional function of the super cluster $\eta$ only, as it was shown that the $\pt$ dependence of the scale factor is negligible. For more details on electron reconstruction, see Khachatryon et al. \cite{ElectronLegacy}. 

\begin{figure}[!htb]
\vspace*{0.3cm}
\begin{center}
\includegraphics[width=0.6\textwidth]{Figures/Electrons/ele_rec_scale_factors.pdf}
\end{center}
\caption{Electron reconstruction efficiencies in data versus $\eta$ and data/MC scale factors as provided by the EGM POG.}
\label{fig:ele_rec_scale_factors}
\end{figure}

\subsection{Electron identification}
\label{sec:eleID}

Reconstructed electrons are identified by means of a Gradient Boosted Decision Tree (GBDT) multivariate classifier algorithm, which exploits observables from the EM cluster, the matching between the cluster and the electron track, as well as observables based exclusively on tracking measurements. 
The BDT has been retrained using CMSSW\_8\_0\_X samples. The classifier is trained on Drell-Yan plus jets MC sample for both signal and background.

%{
%\resizebox{\textwidth}{!}{%
%\tiny  
%/DYJetsToLL\_M-50\_TuneCUETP8M1\_13TeV-madgraphMLM-pythia8/RunIISpring16DR80-PUSpring16\_80X\_mcRun2\_asymptotic\_2016\_v3\_ext1-v1/
%}}
%{
%\tiny  
%\begin{verbatim}
%/DYJetsToLL_M-50_TuneCUETP8M1_13TeV-madgraphMLM-pythia8/RunIISpring16DR80-PUSpring16_80X_mcRun2_asymptotic_2016_v3_ext1-v1/
%\end{verbatim}
%}

The impact of the retraining of the ID for the 2016 conditions is illustrated in the receiver operating characteristic (ROC) curves shown in Figure~\ref{fig:ele_ID_ROC}. Several studies to improve the performance of the multivariate analysis (MVA) for the harsher 2016 running conditions were performed. 
One study considered a new splitting of the BDT training bins, where electrons falling into the gap regions of the ECAL, e.g. the EB-EE transition region, were trained separately from the non-gap electrons. 
However, no improvement for either population was observed, indicating that the current setup is already able to properly take the significantly differing input distributions in those regions into account. 
Additional variables were also studied, including more cluster-shape observables. 
Still, none of these variables helped to improve the performance in the relevant $>95\%$ signal efficiency regime, though up to a $20\%$ improved background rejection was seen for $80\%$ of working points. 
Finally, the hyperparameters of the MVA were systematically scanned for their optimal values, but the resulting configuration was found to improve the overall performance only marginally by $<10\%$ and introduced a significant overtraining effect. 
Due to the small gains and large overtraining, it was decided to not modify the hyperparameters beyond the interface changes from the latest 4.2.0 version of the TMVA package.

Figure~\ref{fig:ele_ID_BDT_output} shows the output of the BDT on the training and testing samples for true and fake electrons 
for the high-$p_T$ training bin in the end cap. 
The good agreement between the training and testing distributions is similar across the six training bins and indicates that the classifier has not been overtrained.

\begin{figure}[!htb]
\vspace*{0.3cm}
\begin{center}
\includegraphics[width=0.9\textwidth]{Figures/Electrons/ele_ROC.png}
\caption{Performance comparison of the MVA trained for the 2015 analysis and the retraining for 2016 conditions. 
The respective working points are indicated by the markers.
\label{fig:ele_ID_ROC}}
\end{center}
\end{figure}

\begin{figure}[!htb]
\vspace*{0.3cm}
\begin{center}
\includegraphics[width=0.5\textwidth]{Figures/Electrons/ele_overtraining.png}
\caption{Boosted decision tree output for the training and testing sample for true and fake electrons in the high-$p_T$ end cap training bins.
\label{fig:ele_ID_BDT_output}}
\end{center}
\end{figure}

Table~\ref{tab:ele_ID_input_variables} summarizes the full list of observables used as inputs to the classifier
and Table~\ref{tab:ele_ID_WP} lists the cut values applied to the BDT score for the chosen working point. 
For the analysis, we define "tight" electrons as the loose electrons that pass this MVA identification working point. 

 \begin{table}[h!]
\scriptsize
    \centering
\resizebox{\textwidth}{!}{%
    \begin{tabular}{c|l}
\hline %----------------------------------------------------------------------------------------
\hline %----------------------------------------------------------------------------------------
%\multicolumn{4}{|c|}{Datasets}                                                                \\
observable type    &  observable name      	\\
\hline %----------------------------------------------------------------------------------------

\multirow{6}{*}{cluster shape}
	&  RMS of the energy-crystal number spectrum along $\eta$ and $\varphi$; $\sigma_{i\eta i\eta}$, $\sigma_{i\varphi i\varphi}$		\\
	&  supercluster width along $\eta$ and $\phi$		\\
	&  'ratio of the hadronic energy behind the electron 
supercluster to the supercluster energy, $H/E$			\\
	&  circularity $(E_{5\times5} - E_{5\times1})/E_{5\times5}$			\\
	&  sum of the seed and adjacent crystal over the super cluster energy $R_{9}$			\\
	&  for end cap traing bins: energy fraction in preshower $E_{PS}/E_{raw}$			\\
\hline
\multirow{2}{*}{track-cluster matching}
	& energy-momentum agreement $E_{tot}/p_{in}$, $E_{ele}/p_{out}$, $1/E_{tot} - 1/p_{in}$ 			\\
	& position matching $\Delta\eta_{in}$, $\Delta\varphi_{in}$, $\Delta\eta_{seed}$			\\
\hline
\multirow{5}{*}{tracking}
        & fractional momentum loss $f_{brem} = 1 - p_{out}/p_{in}$	\\
        & number of hits of the KF and GSF track $N_{KF}$, $N_{GSF}$ $(\mathord{\cdot})$ \\
        & reduced $\chi^2$ of the KF and GSF track $\chi^{2}_{KF}$, $\chi^{2}_{\textrm{GSF}}$ \\
        & number of expected but missing inner hits $(\mathord{\cdot})$ 	\\
        & probability transform of conversion vertex fit $\chi^2$ $(\mathord{\cdot})$ \\

\hline %----------------------------------------------------------------------------------------
\hline %----------------------------------------------------------------------------------------
     \end{tabular}}
%\small
    \caption{Overview of input variables to the identification classifier. Variables not used in the Run 1 MVA are marked with  $(\mathord{\cdot})$.}
    \label{tab:ele_ID_input_variables}
\end{table}


\begin{table}[h!]
\scriptsize
    \centering
    \begin{tabular}{c|c c c}
%\multicolumn{4}{|c|}{Datasets}                                                                \\
\hline %----------------------------------------------------------------------------------------
minimum BDT score    &  $|\eta| < 0.8 $ & $0.8 < |\eta| < 1.479$ 	& $|\eta| > 1.479$      \\
\hline %----------------------------------------------------------------------------------------
$ 5 < p_T < 10 $ $\GeV$ &  -0.211      & -0.396  		& -0.215		\\
$p_T > 10$ $\GeV$       &  -0.870		& -0.838		& -0.763		\\
\hline %----------------------------------------------------------------------------------------
\hline %----------------------------------------------------------------------------------------
     \end{tabular}
\small
    \caption{Minimum boosted decision tree score required for passing the electron identification.}
    \label{tab:ele_ID_WP}
\end{table}


\subsection{Electron isolation}
\label{sec:eleiso}

The relative isolation for electrons is defined as: 

\begin{equation}
\text{RelPFiso} = (\sum_{\text{charged}} p_T + \sum^{\text{corr}}_{\text{neutral}} p_T)/p_T^{\text{lepton}}  
\label{eqn:elepfrelisoeqn}
\end{equation} 
where the corrected neutral component of isolation is computed using the formula:

\begin{equation}
\label{eqn:neutralea}
  \sum^{\text{corr}}_{\text{neutral}} p_T = \text{max}(\sum^{\text{uncorr}}_{\text{neutral}} p_T - \rho \times A_\text{eff},0 \GeV)  
\end{equation}
and the mean pileup contribution to the isolation cone is obtained as:  

\begin{equation}
 PU =  \rho \times A_\text{eff}
\label{eqn:purho}
\end{equation}
where $\rho$ is the mean energy density in the event, and the effective area $A_{eff}$ is defined as the ratio
between the slope of the average isolation and that of $\rho$ as a function of the number of vertices. 

The electron isolation working point was optimized and chosen to be $\text{RelPFiso}(\Delta R = 0.3) < 0.35$ \cite{AN-15-277}. 


\subsection{Electron energy calibrations}

Electrons in data are corrected for features in ECAL energy scale
in bins of $\pt$ and $\left| \eta \right|$. Corrections are calculated
on a $\cPZ \to \Pe\Pe$ sample to align the dielectron 
mass spectrum in the data to that in the MC and to
minimize its width.

The $\cPZ \to \Pe\Pe$ mass resolution in MC is made to match
data by applying a pseudorandom Gaussian smearing to electron energies,
with Gaussian parameters varying in bins of $\pt$ and $\left| \eta \right|$.
This has the effect of convoluting the electron energy spectrum with a
Gaussian.

The electron energy scale is measured in data by fitting a Crystal-Ball function to the dielectron mass spectrum around the $Z$ peak in the $Z+\ell$ control region. 
The energy scale for the full 2016 dataset is shown in Figure~\ref{fig:ele_energy_scale}(a) and agrees with the MC with 100 $\MeV$. 
The stability of the energy scale across different run periods is shown in Fig.~\ref{fig:ele_energy_scale}(b), where the data is binned into approximately 500~pb luminosity blocks.

%\begin{figure}[!htb]
%\vspace*{0.3cm}
%\begin{center}
%\subfigure [] {\resizebox{7.5cm}{!}{\includegraphics{Figures/Electrons/ele_energy_scale.pdf}}}
%\subfigure [] {\resizebox{9.5cm}{!}{\includegraphics{Figures/Electrons/ele_energy_scale_per_lumi.pdf}}}
%\end{center}
%\caption{
%(a): electron energy scale measured in the $Z+\ell$ control region for EB and EE electrons. The results of the Crystall-ball fit are reported in the figure. 
%(b): lepton energy scales per 500~pb luminosity block. 
%}
%\label{fig:ele_energy_scale}
%\end{figure}

\begin{figure}[tbh]
\centering
\begin{subfigure}{0.9\textwidth}
\centering
\includegraphics[width=4in]{Figures/Electrons/ele_energy_scale.pdf}
\caption{}
\end{subfigure}
\begin{subfigure}{0.9\textwidth}
\centering
\includegraphics[width=4.5in]{Figures/Electrons/ele_energy_scale_per_lumi.pdf}
\caption{}
\end{subfigure}
\caption{(a): Electron energy scale measured in the $Z+\ell$ control region for EB and EE electrons. The results of the Crystall-ball fit are reported in the figure. (b): lepton energy scales per 500~pb luminosity block.}
\label{fig:ele_energy_scale}
\end{figure}

\subsection{Electron efficiency measurements}
\label{sec:eleEffMeas}
%\input{Objects/eleEffMeas.tex}

The tag-and-probe (T\&P) study was performed on the single electron primary datasets listed in Table~\ref{tab:datasets_data} using the same golden JSON of 36.8 
fb$^{-1}$ as for the main analysis \cite{AN-15-277}. 

Tag electrons need to satisfy the following quality requirements:
(1) trigger matched to HLT\_Ele27\_eta2p1\_WPTight\_Gsf\_v*
(2) $p_{T} > 30$ $\GeV$, super cluster (SC) $\eta < 2.1$ but on in EB-EE gap ($1.4442<|\eta|<1.566$)
and (3) tight working point of the Spring16 cut-based electron ID.

Probe electrons only need to be reconstructed as GsfElectron, electrons associated with a GsfTrack object. The FSR recovery algorithm used in the main analysis is used consistently throughout the efficiency measurement; the isolations are calculated without any FSR photons matched to electrons and the probe electron $\pt$ as well as the dielectron invariant mass include the FSR photons, if any. 


The nominal MC efficiencies are evaluated from the LO MadGraph Drell-Yan sample, while the NLO systematics use the 0,1 jet MadGraph\_AMCatNLO sample listed in Table \ref{tab:MCsamples}.

In contrast to previous efficiency measurements, a template fit is used here. The $m_{ee}$ signal shape of the passing and failing probes is taken from MC and convoluted with a Gaussian. The data is then fitted with the convoluted MC template and a CMSShape, an error function with a one-sided exponential tail. This change follows from the usage of the new T\&P tool developed by the EGM POG.


%\paragraph{Electron selection efficiency measurements}\mbox{}\\
%\label{par:Efficiency_measurements}

The electron selection efficiency is measured as a function of the probe electron $\pt$ and its SC $\eta$, and separately for electrons falling in the ECAL gaps. Figure \ref{fig:ele_sel_pt_turn_on} shows the $\pt$ turn-on curves measured in data, and the final two-dimensional scale factor is shown in Figure~\ref{fig:ele_sel_scale_factors} together with the systematic uncertainties. These scale factors are very similar to the ICHEP figures, except more homogenous across $\eta$ and $\pt$ because of the higher statistics and the usage of more stable fitting routines in the new T\&P tool.


%\begin{figure}[!htb]
%\begin{center}
%    \subfigure [] {\resizebox{7.5cm}{!}{\includegraphics{Figures/Electrons/ele_eff_pt.pdf}}}
%    \subfigure [] {\resizebox{7.5cm}{!}{\includegraphics{Figures/Electrons/gap_ele_eff_pt.pdf}}}\\
%\caption{Electron selection efficiencies measured using the Tag-and-Probe technique described in the text, non-gap electrons (left) and gap electrons (right).}
%\label{fig:ele_sel_pt_turn_on}
%\end{center}
%\end{figure}

\begin{figure}[tbh]
\centering
\begin{subfigure}{0.95\textwidth}
\centering
\includegraphics[width=3.5in]{Figures/Electrons/ele_eff_pt.pdf}
\caption{}
\end{subfigure}
\begin{subfigure}{0.95\textwidth}
\centering
\includegraphics[width=3.5in]{Figures/Electrons/gap_ele_eff_pt.pdf}
\caption{}
\end{subfigure}
\caption{Electron selection efficiencies measured using the tag-and-probe technique described in the text, non-gap electrons (a) and gap electrons (b).}
\label{fig:ele_sel_pt_turn_on}
\end{figure}

%\begin{figure}[!htb]
%\begin{center}
%    \subfigure [] {\resizebox{15cm}{!}{\includegraphics{Figures/Electrons/ele_eff_sf_unc.pdf}}}\\
%    \subfigure [] {\resizebox{15cm}{!}{\includegraphics{Figures/Electrons/gap_ele_eff_sf_unc.pdf}}}
%\caption{Electron selection efficiencies measured using the Tag-and-Probe technique described in the text, non-gap electrons (top) and gap electrons (bottom).}
%\label{fig:ele_sel_scale_factors}
%\end{center}
%\end{figure}

\begin{figure}[tbh]
\centering
\begin{subfigure}{0.95\textwidth}
\centering
\includegraphics[width=5in]{Figures/Electrons/ele_eff_sf_unc.pdf}
\caption{}
\end{subfigure}
\begin{subfigure}{0.95\textwidth}
\centering
\includegraphics[width=5in]{Figures/Electrons/gap_ele_eff_sf_unc.pdf}
\caption{}
\end{subfigure}
\caption{Electron selection efficiencies measured using the tag-and-probe technique described in the text, non-gap electrons (a) and gap electrons (b)}
\label{fig:ele_sel_scale_factors}
\end{figure}


%\paragraph{Systematic uncertainties}\mbox{}\\
%\label{par:Systematic_uncertainties}
%%%%%%%%%%%%%%%%%%%%%%%%%%%%

The EGM recommendations on the evaluation of T\&P uncertainties for efficiency measurements are followed. Specifically: 
(1) Variation of the signal shape from a MC shape to an analytic shape (Crystal-Ball) fitted to the MC
(2) Variation of the background shape from a CMS-shape to a simple exponential in fits to data
(3) Variation of the tag selection: tag $p_{T}>$35 $\GeV$ and passes MVA-based ID, and
(4) Using an NLO MC sample for the signal templates.
The total uncertainty for the measurement of the scale factors is the quadratic sum of the statistical uncertainties returned from the fit and the aforementioned systematic uncertainties.


\section{Muons}

\subsection{Muon reconstruction and identification}
\label{sec:muonReco}

More details on muon reconstruction can be found in Ref.~\cite{AN-15-277}.
"Loose" muons are the muons that satisfy  
$p_T > 5$, $|\eta| < 2.4$, $d_{xy}< 0.5$, and $d_z < 1$, where $d_{xy}$ and $d_z$ are 
defined with respect to the primary vertex and using the muonBestTrack. Muons have to be 
reconstructed by either the Global Muon or Tracker Muon algorithm. Standalone 
muon tracks that are only reconstructed in the muon system are rejected.
Sstandalone muons are discarded even if they are marked as global or tracker muons. 

Loose muons with $\pt$ below 200 $\GeV$ are considered "tight" muons if they 
also pass the PF muon ID. Note that the naming 
convention used for these IDs differs from the muon POG naming scheme, in which
the ``tight ID'' used here is called the ``loose ID''. Loose muons with $\pt$ 
above 200 $\GeV$ are considered tight muons if they pass the PF ID or the Tracker
High-$\pt$ ID, the definition of which is shown in Table~\ref{tab:highPtID}.
This relaxed definition is used to increase signal efficiency for the high-mass
search. When a very heavy resonance decays to two $\cPZ$ bosons, both bosons
will be very boosted. In the laboratory frame, the leptons coming from the decay of
a highly boosted $\cPZ$ will be nearly collinear, and the PF ID loses 
efficiency for muons separated by approximately $\Delta R < 0.4$, which roughly 
corresponds to muons originating from $\cPZ$ bosons with $\pt > 500\ \GeV$.

\begin{table}[h]
    \begin{small}
    \begin{center}
    \begin{tabular}{|l|l|}
      \hline
      Plain-text description         & Technical description                 \\
      \hline
      Muon station matching          & Muon is matched to segments           \\
                                     & in at least two muon stations         \\
      \hline                                                          
      Good $\pt$ measurement         & $\frac{\pt}{\sigma_{\pt}} < 0.3$      \\
      \hline
      Vertex compatibility ($x-y$)   & $d_{xy} < 2$~mm                       \\
      \hline
      Vertex compatibility ($z$)     & $d_{z} < 5$~mm                        \\
      \hline
      Pixel hits                     & At least one pixel hit                \\
      \hline
      Tracker hits                   & Hits in at least six tracker layers   \\
      \hline
    \end{tabular}
    \caption{
      The requirements for a muon to pass the Tracker High-$\pt$ ID. Note that
      these are equivalent to the Muon POG High-$\pt$ ID with the global track 
      requirements removed.
      }
    \label{tab:highPtID}
    \end{center}
    \end{small}
\end{table}

An additional ``ghost-cleaning'' step is performed to deal with situations when a single muon
can be incorrectly reconstructed as two or more muons. In this step, Tracker Muons that are not Global Muons are required to be arbitrated, and if two muons are sharing 50\% or more of their segments, then the muon with lower quality is removed.

\subsection{Muon isolation}
\label{sec:muoniso}

PF-based isolation, described for electrons in Section~\ref{sec:eleiso}, is also used for the muons. 
The only difference is the way the pileup contribution is subtracted; for the muons, $\Delta\beta$ correction is applied, whereby $\Delta\beta = \frac{1}{2} \sum^\text{charged had.}_\text{PU} \pt$  gives an estimate of the energy deposit of neutral particles (i.e. hadrons and photons) from pileup vertices. 
The relative isolation for muons is then defined as:
\begin{equation}
\text{RelPFiso} = \frac{\sum^\text{charged had.} \pt + \max(\sum^\text{neutral had.} \ET 
+ \sum^\text{photon} \ET - \Delta \beta, 0)}{\pt^\text{lepton}}
\label{eqn:mupfiso}
\end{equation}

The isolation working point for muons was optimized and chosen to be the same as for electrons, $\text{RelPFiso}(\Delta R = 0.3) < 0.35$ \cite{AN-15-277}. 

%\subsection{Muon Energy Calibrations}
% \input{Objects/muCalib}

\subsection{Muon efficiency measurements}
\label{sec:muonEffMeas}

Muon efficiencies are measured with the T\&P method performed on
$\cPZ \to \Pgm\Pgm$ and $\JPsi\to\mu\mu$ events in bins of $\pt$ and $\eta$ \cite{AN-15-277}.
The $\Z$ sample is used to measure the muon reconstruction and identification efficiency at high $\pt$
and the efficiency of the isolation and impact parameter requirements at all $\pt$.
The $\JPsi$ sample is used to measure the reconstruction efficiency at low $\pt$,
as it benefits from a better purity in that kinematic regime. In this case,
events are collected using HLT\_Mu7p5\_Track2\_Jpsi\_v* when probing the
reconstruction and identification efficiency in the muon system and using the
 HLT\_Mu7p5\_L2Mu2\_Jpsi\_v* when probing the tracking efficiency.

\subsubsection{Reconstruction and identification}

Results for the muon reconstruction and identification efficiency for $\pt > 20\ \GeV$
have been derived by the Muon POG.
The probe in this measurement are tracks reconstructed in the inner tracker, and
the passing probes are those that are also reconstructed as a global or tracker muon 
and passing the Muon POG Loose muon identification.
%
Results for low $\pt$ muons were derived using $\JPsi$ events, with the same definitions
of probe and passing probes. The systematic uncertainties are estimated by varying the analytical signal and background shape models used to fit 
the dimuon invariant mass \cite{AN-15-277}. The efficiency and scale 
factors used for low $\pt$ muons are the ones derived using single muon prompt-reco dataset.
The efficiency in data and simulation is shown in Figure~\ref{fig:MuonIDEff_1}. 

\begin{figure}[tbh]
\centering
\begin{subfigure}{0.3\textwidth}
\centering
\includegraphics[width=2in]{Figures/Muons/mu_Loose_barrel.pdf}
\caption{}
\end{subfigure}
\begin{subfigure}{0.3\textwidth}
\centering
\includegraphics[width=2in]{Figures/Muons/mu_Loose_endcap.pdf}
\caption{}
\end{subfigure}
\begin{subfigure}{0.3\textwidth}
\centering
\includegraphics[width=2in]{Figures/Muons/mu_Loose_pt7.pdf}
\caption{}
\end{subfigure}
    \caption{Muon reconstruction and identification efficiency at low $\pt$, measured with the tag-and-probe method on $\JPsi$ events, as function of $\pt$ in the barrel (a) and end caps (b), and as function of $\eta$ for $\pt > 7\ \GeV$ (c). In the upper panel of each graph, the larger error bars include also the systematical uncertainties, while the smaller ones are purely statistical. Each graph's lower panel shows the ratio of the two efficiencies, the black error bars are for the statistical uncertainty, the orange rectangles for the systematic uncertainty, and the violet rectangles include both uncertainties.}
\label{fig:MuonIDEff_1}
\end{figure}

%\begin{figure}[htbp]
%  \begin{center}
%    \subfigure[]{\includegraphics[width=0.32\textwidth]{Figures/Muons/mu_Loose_barrel.pdf}}
%    \subfigure[]{\includegraphics[width=0.32\textwidth]{Figures/Muons/mu_Loose_endcap.pdf}}
%    \subfigure[]{\includegraphics[width=0.32\textwidth]{Figures/Muons/mu_Loose_pt7.pdf}}
%    \caption{Muon reconstruction and identification efficiency at low \pt, measured with the tag\&probe method on \JPsi events, as function of \pt in the barrel (left) and endcaps (center), and as function of $\eta$ for $\pt > 7\GeV$ (right). In the upper panel, the larger error bars include also the systematical uncertainties, while the smaller ones are purely statistical. In the lower panel showing the ratio of the two efficiencies, the black error bars are for the statistical uncertainty, the orange rectangles for the systematical uncertainty and the violet rectangles include both uncertainties.}
%    \label{fig:MuonIDEff_1}
%\end{center}
%\end{figure}

\subsubsection{Impact parameter requirements}
The measurement is performed using $\Z$ events. Events are selected with HLT\_IsoMu20\_v* or HLT\_IsoMu22\_v* triggers.
For this measurement, the probe is a muon passing the POG Loose identification criteria,
and it is considered a passing probe if it satisfies the SIP3D, $d_{xy}$, $d_z$ cuts of this analysis.
The results are shown in Figure~\ref{fig:MuonIDEff_2}.
%Very good agreement between data and simulation is observed in the barrel (Fig.~\ref{fig:MuonIDEff_2}, left)
%while some inefficiency is visible in the endcaps, especially at large values of $|\eta|$.
%The data to simulation scale factor is found to be flat as function of \pt, so, similarly to what done
%for the identification part, we apply a correction only as function of $\eta$.


\begin{figure}[tbh]
\centering
\begin{subfigure}{0.3\textwidth}
\centering
\includegraphics[width=2in]{Figures/Muons/mu_SIP4_barrel.pdf}
\caption{}
\end{subfigure}
\begin{subfigure}{0.3\textwidth}
\centering
\includegraphics[width=2in]{Figures/Muons/mu_SIP4_endcap.pdf}
\caption{}
\end{subfigure}
\begin{subfigure}{0.3\textwidth}
\centering
\includegraphics[width=2in]{Figures/Muons/mu_SIP4_pt20.pdf}
\caption{}
\end{subfigure}
\caption{Efficiency of the muon impact parameter requirements, measured with the tag-and-probe method on $Z$ events, as function of $\pt$ in the barrel (a) and end caps (b), and as function of $\eta$ for $\pt > 20\ \GeV$ (c). In the upper panel of each graph, the larger error bars include also the systematical uncertainties, while the smaller ones are purely statistical. Each graph's lower panel shows the ratio of the two efficiencies, the black error bars are for the statistical uncertainty, the orange rectangles for the systematical uncertainty, and the violet rectangles include both uncertainties.}
\label{fig:MuonIDEff_2}
\end{figure}

%\begin{figure}[htbp]
%  \begin{center}
%    \subfigure[]{\includegraphics[width=0.32\textwidth]{Figures/Muons/mu_SIP4_barrel.pdf}}
%    \subfigure[]{\includegraphics[width=0.32\textwidth]{Figures/Muons/mu_SIP4_endcap.pdf}}
%    \subfigure[]{\includegraphics[width=0.32\textwidth]{Figures/Muons/mu_SIP4_pt20.pdf}}
%    \caption{Efficiency of the muon impact parameter requirements, measured with the tag\&probe method on \Z events, as function of \pt in the barrel (left) and endcaps (center), and as function of $\eta$ for $\pt > 20\GeV$ (right). In the upper panel, the larger error bars include also the systematical uncertainties, while the smaller ones are purely statistical. In the lower panel showing the ratio of the two efficiencies, the black error bars are for the statistical uncertainty, the orange rectangles for the systematical uncertainty and the violet rectangles include both uncertainties.}
%    \label{fig:MuonIDEff_2}
%\end{center}
%\end{figure}

\subsubsection{Isolation requirements}
The isolation efficiency is measured using events from the $\Z$ decay for any $\pt$, selected with either of HLT\_IsoMu20\_v* or HLT\_IsoMu22\_v* triggers. The isolation of the muons is calculated after recovery of the FSR photons and subtracting their contribution to the isolation cone of the muons \cite{AN-16-217}.
The results are shown in Figure~\ref{fig:MuonIDEff_3}.

\begin{figure}[tbh]
\centering
\begin{subfigure}{0.45\textwidth}
\centering
\includegraphics[width=3in]{Figures/Muons/mu_iso_barrel.pdf}
\caption{}
\end{subfigure}
\begin{subfigure}{0.45\textwidth}
\centering
\includegraphics[width=3in]{Figures/Muons/mu_iso_endcap.pdf}
\caption{}
\end{subfigure}
    \caption{Efficiency of the muon isolation requirement, measured with the tag-and-probe method on $\Z$ events, as function of $\pt$ in the barrel (left) and end caps (right). In the upper panel of each graph, the larger error bars include also the systematical uncertainties, while the smaller ones are purely statistical. The lower panel of each graph shows the ratio of the two efficiencies, the black error bars are for the statistical uncertainty, the orange rectangles for the systematical uncertainty, and the violet rectangles include both uncertainties.}
\label{fig:MuonIDEff_3}
\end{figure}

%\begin{figure}[htbp]
%  \begin{center}
%    \subfigure[]{\includegraphics[width=0.32\textwidth]{Figures/Muons/mu_iso_barrel.pdf}}
%    \subfigure[]{\includegraphics[width=0.32\textwidth]{Figures/Muons/mu_iso_endcap.pdf}}
%    \caption{Efficiency of the muon isolation requirement, measured with the tag\&probe method on \Z events, as function of \pt in the barrel (left) and endcaps (right). In the upper panel, the larger error bars include also the systematical uncertainties, while the smaller ones are purely statistical. In the lower panel showing the ratio of the two efficiencies, the black error bars are for the statistical uncertainty, the orange rectangles for the systematical uncertainty and the violet rectangles include both uncertainties.}
%    \label{fig:MuonIDEff_3}
%\end{center}
%\end{figure}

\subsubsection{Tracking}
The efficiency to reconstruct a muon track in the inner detector is measured using probes tracks
reconstructed in the muon system \cite{CMS_AN_2015-215}. The efficiency and 
data-to-MC scale factors are measured from $Z$ events as a function of $\eta$ for $\pt > 10\ \GeV$ and $\pt < 10\ \GeV$. The values of data-to-MC scale factors 
used are from the ReReco version of the full dataset collected in 2016. 
The tracking efficiency in data and simulation as a function of $\eta$ is shown in Figure~\ref{fig:MuonIDEff_4}.

\begin{figure}[tbh]
\centering
\begin{subfigure}{0.45\textwidth}
\centering
\includegraphics[width=3in]{Figures/Muons/trackingEffptl10.pdf}
\caption{}
\end{subfigure}
\begin{subfigure}{0.45\textwidth}
\centering
\includegraphics[width=3in]{Figures/Muons/trackingEffptg10.pdf}
\caption{}
\end{subfigure}
    \caption{Tracking efficiency in data and simulation as a function of $\eta$ for muon $\pt < 10\ \GeV$(a) and $\pt > 10\ \GeV$(b) with ReReco data.}
    \label{fig:MuonIDEff_4}
\end{figure}

%\begin{figure}[htbp]
%  \begin{center}
%    \subfigure[]{\includegraphics[width=0.42\textwidth]{Figures/Muons/trackingEffptl10.pdf}}
%    \subfigure[]{\includegraphics[width=0.42\textwidth]{Figures/Muons/trackingEffptg10.pdf}}
%    \caption{Tracking efficiency in data and simulation as a function of $\eta$ for muon $\pt < 10\GeV$(left) and $\pt > 10\GeV$(right) with ReReco data.}
%    \label{fig:MuonIDEff_4}
%\end{center}
%\end{figure}

\subsubsection{Overall results}
The product of all the data-to-MC scale factors for muon tracking, reconstruction, identification, impact parameter, and isolation requirements is shown in Figure~\ref{fig:MuonIDEff_5}. 
%The overall correction is about $-1\%$ or less for most \pt and $\eta$ values, increasing to about $-2\%$ in for muons below $10\GeV$ or with $|\eta|>2$.

\begin{figure}[tbh]
\centering
\begin{subfigure}{0.45\textwidth}
\centering
\includegraphics[width=3in]{Figures/Muons/mu_sf.pdf}
\caption{}
\end{subfigure}
\begin{subfigure}{0.45\textwidth}
\centering
\includegraphics[width=3in]{Figures/Muons/mu_sf_unc.pdf}
\caption{}
\end{subfigure}
    \caption{(a) Overall data to simulation scale factors for muons, as function of $\pt$ and $\eta$. (b) Uncertainties on  data to simulation scale factors for muons, as function of $\pt$ and $\eta$.}
    \label{fig:MuonIDEff_5}
\end{figure}

%\begin{figure}[htbp]
%  \begin{center}
%    \includegraphics[width=0.45\textwidth]{Figures/Muons/mu_sf.pdf}
%    \includegraphics[width=0.45\textwidth]{Figures/Muons/mu_sf_unc.pdf}
%    \caption{Left: Overall data to simulation scale factors for muons, as function of \pt and $\eta$. Right: Uncertainties on  data to simulation scale factors for muons, as function of \pt and $\eta$.}
%    \label{fig:MuonIDEff_5}
%\end{center}
%\end{figure}


\section{Photons for FSR recovery}
\label{sec:FSRphotons}

The FSR recovery algorithm was considerably simplified with respect to what was done in Run 1, while maintaining similar performance. 
The selection of FSR photons is now only done per-lepton and no longer depends on any $Z$ mass criterion, simplifying the subsequent $ZZ$ candidate building and selection. for the association of photons with leptons, the rectangular cuts on $\Delta R(\gamma,l)$ and $E_{T,\gamma}$  have been replaced by a cut on $\Delta R(\gamma,l)/E_{T,\gamma}^{2}$.

Starting from the collection of PF photons provided by the PF algorithm, the selection of photons and their association to a lepton proceeds as follows \cite{AN-15-277, AN-16-217}:

\begin{enumerate}
\item The preselection of PF photons is done by requiring $p_{T,\gamma} > 2~\GeV$, $|\eta^{\gamma}| < 2.4$, and a relative PF isolation $<1.8$. The isolation is computed using a cone of radius $R=0.3$, a threshold of $0.2~\GeV$ on charged hadrons with a veto cone of $0.0001$, and $0.5~\GeV$ on neutral hadrons and photons with a veto cone of $0.01$, also including the contribution from pileup vertices (with the same radius and threshold as per charged isolation) .
\item Supercluster veto: all PF photons that match with any electron passing both the loose ID and SIP cuts are removed. The matching is peformed by directly associating the two PF candidates.
\item Photons are associated to the closest lepton in the event among all those that pass both the loose ID and SIP cuts.
\item Photons that fail the cuts $\Delta R(\gamma,l)/E_{T,\gamma}^2 < 0.012$, and $\Delta R(\gamma,l)<0.5$ are discarded.
\item If more than one photon is associated to the same lepton, the lowest-$\Delta R(\gamma,l)/E_{T,\gamma}^2$ is selected.
\item For each FSR photon that was selected, photons from the isolation sum of all the leptons in the event that pass both the loose ID and SIP cuts are excluded. This concerns the photons that are in the isolation cone and outside the isolation veto of said leptons ($\Delta R < 0.4$ AND $\Delta R > 0.01$ for muons and $\Delta R < 0.4$ AND ($\eta^{\text{SC}} < 1.479$ OR $\Delta R > 0.08$) for electrons).
\end{enumerate}

\section{Jets}
\label{sec:jets}

VBF and other $H$ production mechanisms normally have different jet kinematics. 
Therefore, jets can be used to categorize events based on the production mechanisms.

\subsubsection{Jet identification}

Jets are reconstructed by using the anti-$k_T$ clustering algorithm out of PF candidates, with a distance parameter $R = 0.4$, 
after rejecting the charged hadrons that are associated to a pileup primary vertex.

To reduce instrumental background, the loose working point jet ID suggested by the JetMET Physics Object Group is applied. 
In this analysis, the jets are required to be within $|\eta| < 4.7$ area and have a transverse momentum above 30 $\GeV$. 
In addition, the jets are cleaned from any of the tight leptons (passing the SIP and isolation cut computed after FSR correction) 
and FSR photons by a separation criterion: $\Delta R(\text{jet,lepton/photon}) > 0.4$.


\subsubsection{Jet energy corrections}

The calorimeter response to particles is not linear,
and it is not straightforward to translate the measured jet energy
to the true particle or parton energy; therefore, jet energy corrections (JECs) are needed.
In this analysis, standard jet energy corrections are applied to the reconstructed jets,
which consist of L1 Pileup, L2 Relative Jet Correction,
L3 Absolute Jet Correction for both MC samples and data,
and also residual calibration for data.

% Figure~\ref{fig:jets} shows the comparisoin between data and MC for the leading jet in Z events with exactly one jet,
% where a selection $\Delta\phi(Z,{\rm jet})>2.5$ has been applied.

% \begin{figure}[!h]
% \centering
% \includegraphics[width=0.49\linewidth]{Figures/Jets/Histo_etaj1_2e_dataeff.pdf}
% \includegraphics[width=0.49\linewidth]{Figures/Jets/Histo_etaj1_2mu_dataeff.pdf}
% \caption{Comparison between data and MC for jet $\eta$ in Z + 1 jet events. \label{fig:jets}}
% \end{figure}


\subsubsection{B-tagging}

For categorization purposes, whether or not a jet is a b-jet needs to be distinguished.
The Combined Secondary Vertex (CSV) algorithm is used as the b-tagging algorithm.
It combines information about impact parameter significance,
the secondary vertex and jet kinematics.
The variables are combined using a likelihood ratio technique to compute the b-tag discriminator.
In this analysis, a jet is considered to be b-tagged if it passes the \emph{CSVv2M} working point,
i.e. if its pfCombinedInclusiveSecondaryVertexV2BJetTags discriminator is greater than 0.8484~\cite{btagReferenceEffsRun2}.

Data-to-MC scale factors for b-tagging efficiency are provided for this working point for the full dataset as a function of jet $\pt$, $\eta$, and flavor.
They are applied to simulated jets by downgrading (upgrading) the b-tagging status of a fraction of the b-tagged (untagged) jets that have a scale factor smaller (larger) than one.

\section{MET}

The missing transverse energy, $E_{\rm{T}}^{\rm{MISS}}$ or MET, of an event is defined as the magnitude of the imbalance of momentum in the plane transverse to the beam line. Since momentum is conserved in this plane, any imbalance in momentum is attributed to particles escaping the detector without interacting with the detector material, such as neutrinos or hypothetical dark matter candidates. Raw MET or particle flow MET (PFMET) is defined as the magnitude of the negative vectorial sum of the transverse momentum of all reconstructed particle flow candidates, or

\begin{equation}
\overrightarrow{E}_{\rm{T}}^{\rm{MISS}} = - \sum_{i \in \rm{all}} \overrightarrow{p}_{\rm{T},\ i}
\end{equation}

The vector quantity that is the negative sum of reconstructed particle momenta is sometimes called the missing transverse momentum, although this term is used interchangably with its magnitude, the MET. 

An alternative definition of the MET, called the type-1 corrected MET, takes into account the jet energy corrections (JEC), correcting for mismeasurement of MET due to detector inefficiencies and non-linear responses in the calorimeters. The type-1 corrected MET definition is given by:

\begin{equation}
\label{eq:t1met}
\overrightarrow{E}_{\rm{T\ Type-1}}^{\rm{MISS}} = - \sum_{jet} \overrightarrow{p}_{\rm{T},\ jet}^{\rm{JEC}} - \sum_{i \in \rm{uncl.}} \overrightarrow{p}_{\rm{T},\ i}
\end{equation}
where the total contribution has been split into contributions from jets (first term) and contributions from unclustered objects (second term). The transverse momenta of jets in the first term is then replaced with the JEC transverse momenta.

Systematic uncertainties related to modeling real MET are obtained by varying the JEC and jet energy resolution (JER) and measuring the propogation of these variations to the MET uncertainty. These measurements are described in greater detail in Section~\ref{sec:metsyst}.

\subsubsection{MET filters}

Due to detector and instrumental noise, several filters are applied to veto noisy events \cite{mettwiki}:
HBHENoiseFilter,
HBHENoiseIsoFilter,
EcalDeadCellTriggerPrimitiveFilter, 
goodVertices, 
eeBadScFilter, 
globalTightHalo2016Filter, 
BadPFMuonFilter, and
BadChargedCandidateFilter.

The first two filters remove noisy events from the HCAL, where the HBHE scintillator produce anamolous signals with pulse shapes and pixel multiplicities discrepant from those from a clean signal. The EcalDeadCellTriggerPrimitiveFilter removes events with down ECAL data links, comparing the sum of energy deposited in each cell of a supercluster to the trigger primitive saturation energy. The goodVertices filter removes events with noisy vertex reconstruction from pileup effects. The eeBadScFilter removes events with noisy ECAL end cap superclusters. Tje globalTightHalo2016Filter filter removes events with enhanced MET from beam-halo particles which are in time with the beam. The last two filters remove events with mis-reconstructed muon and charged hadron particle flow candidates.

\subsubsection{Fake MET modeling}\label{sec:fakemet}

Figure~\ref{fig:pfmet_m4lblinded} shows a discrepancy between data and MC in the high-$E_{\rm{T}}^{\rm{MISS}}$ tail. These events typically contain a high $p_T$ object either back-to-back or collinear with the $E_{\rm{T}}^{\rm{MISS}}$, pointing to artificially high $E_{\rm{T}}^{\rm{MISS}}$ from mismeasurement of the object. These fake events are identified and removed by studying distributions of the transverse angular difference between the MET and various objects in the event \cite{CMS-AN-15-203}.

\begin{figure}[tbh]
\centering
\includegraphics[width=5in]{figures/hist_hPFMET_8.png}
\caption{Missing transverse energy (MET) after the standard model selection before removal of fake MET.}
\label{fig:pfmet_m4lblinded}
\end{figure}

The following variables are studied in order to understand how to remove fake events from data: (1) $max|\Delta\phi(jet, E_{\rm{T}}^{\rm{MISS}})|$ with the maximum taken over selected jets, Figure~\ref{fig:maxdeltaphijmet} and (2) $min|\Delta\phi(jet, E_{\rm{T}}^{\rm{MISS}})|$ with the minimum taken over selected jets, Figure~\ref{fig:mindeltaphijmet}
The maximum variable is to check for the occurance of objects with mismeasured energy back-to-back with the MET, while the minimum variable is to check for jets with mismeasured energy collinear with the MET.

\begin{figure}[tbh]
\centering
\includegraphics[width=5in]{figures/hist_hDPHI_MAX_JET_MET_8.png}
\caption{Maximum azimuthal angle difference between MET and jets.}
\label{fig:maxdeltaphijmet}
\end{figure}

\begin{figure}[tbh]
\centering
\includegraphics[width=5in]{figures/hist_hDPHI_MIN_JET_MET_8.png}
\caption{Minimum azimuthal angle difference between MET and jets.}
\label{fig:mindeltaphijmet}
\end{figure}

For jets with a high transverse momentum, $>50\ \GeV$, it is required that $max|\Delta\phi(jet, E_{\rm{T}}^{\rm{MISS}})|<2.7$ and $min|\Delta\phi(jet, E_{\rm{T}}^{\rm{MISS}})|>0.5$ to exclude events with large MET from mismeasurment of jet energies. These cuts are based on the Run 1 SM analysis selection, chosen to balance the small loss in signal efficiency with the increased systematic uncertainty from mismodeling of the MET in background MC simulations (described in Section~\ref{sec:metsyst}).


\chapter{Signal and control regions}

\section{Event selection}

\subsection{Trigger selection}
\label{sec:HLTsel}

The events are required to have fired the High-Level Trigger (HLT) paths described in section~\ref{sec:trigpaths}. Unlike in the Run 1 analysis, the trigger requirement in the Run 2 analysis does not depend on the selected final state. Instead, it is always the OR of all ten HLT paths, since associated production modes that can have additional leptons will be targeted. This improves the trigger efficiency further.

\subsection{Vertex selection}
\label{sec:vertexsel}

The events are required to have at least one good primary
vertex (PV) fullfilling all three of the following criteria: a high number of degrees
of freedom ($N_{PV}>4$), collisions restricted along the $z$-axis
($z_{PV}<24$ cm), and a small radius of the PV ($r_{PV}<2$ cm).


\subsection{$ZZ$ candidate selection}
\label{sec:zzcandsel}

The four-lepton candidates are built from ``selected'' leptons, which  
are the tight leptons (defined in Sections~\ref{sec:eleID} and~\ref{sec:muonReco}) that pass the ${\rm SIP_{3D}} < 4$ vertex constraint
and the isolation cuts (defined in Sections~\ref{sec:eleiso} and~\ref{sec:muoniso}), 
where FSR photons are subtracted as described in Section~\ref{sec:FSRphotons}. The muon isolation and SIP distributions are shown in Figure~\ref{fig:lepkin1}.
A lepton cross-cleaning is applied 
by discarding electrons that are within $\Delta R < 0.05$ of selected muons. Additional kinematics for these leptons at the stage they are selected are shown in Figure~\ref{fig:lepkin2}.

 
\begin{figure}[tbh]
\begin{subfigure}{0.50\textwidth}
\centering
\includegraphics[width=3.3in]{figures/hist_hPtLep_3.png}
\caption{}
\end{subfigure}
\begin{subfigure}{0.50\textwidth}
\centering
\includegraphics[width=3.3in]{figures/hist_hEtaLep_3.png}
\caption{}
\end{subfigure}
\caption{Selected lepton kinematics: (a) lepton isolation and (b) SIP vertex constraint.}
\label{fig:lepkin1}
\end{figure}
 
\begin{figure}[tbh]
\begin{subfigure}{0.50\textwidth}
\centering
\includegraphics[width=3.3in]{figures/hist_hIsoLep_3.png}
\caption{}
\end{subfigure}
\begin{subfigure}{0.50\textwidth}
\centering
\includegraphics[width=3.3in]{figures/hist_hSipLep_3.png}
\caption{}
\end{subfigure}
\caption{Selected lepton kinematics: (a) transverse momentum and (b) $\eta$}
\label{fig:lepkin2}
\end{figure}

 

The construction and selection of four-lepton candidates proceeds 
according to the following sequence:
\begin{enumerate}
\item {\bf $Z$ candidates} are defined as pairs of selected leptons
 of opposite charge and matching flavor (e.g. $e^+ e^-$ or $\mu^+\mu^-$)
 that satisfy $12 < m_{\ell\ell(\gamma)} < 120~\GeV$, where the $Z$ candidate mass
 includes the selected FSR photons if any. The dimuon invariant mass distribution for $Z$ candidates at this step is shown in Figure~\ref{fig:dimuonz}. The distribution of $\MET$ at this step, showing a large contribution from $Z+X$, is shown in Figure~\ref{fig:pfmet3}.

\begin{figure}[tbh]
\centering
\includegraphics[width=5.5in]{figures/hist_hMZ_3.png}
    \caption{Dimuon invariant mass distribution for $Z$ candidates.}
    \label{fig:dimuonz}
\end{figure}

\begin{figure}[tbh]
\centering
\includegraphics[width=5.5in]{figures/hist_hPFMET_3.png}
    \caption{Missing transverse energy distribution at the $Z$ candidate selection step.}
    \label{fig:dimuonz}
\end{figure}

\item {\bf $ZZ$ candidates} are defined as pairs of non-overlapping $Z$ candidates.
 The $Z$ candidate with reconstructed mass $m_{\ell\ell}$ closest to the nominal $Z$ boson
 mass is denoted as $\Z_1$, and the second one is denoted as $Z_2$. The dimuon distributions for $Z_1$ and $Z_2$ at this step are shown in Figure~\ref{fig:dimuonz5}. Additional kinematics for the $Z_1$ and $Z_2$ candidates are shown in Figures~\ref{fig:z15kin} and \ref{fig:z25kin}. Kinematics for the constituent muons are shown in Figures~\ref{fig:lep5kin1} and \ref{fig:lep5kin2}.

\begin{figure}[tbh]
\begin{subfigure}{0.50\textwidth}
\centering
\includegraphics[width=3.3in]{figures/hist_hMZ1_5.png}
\caption{}
\end{subfigure}
\begin{subfigure}{0.50\textwidth}
\centering
\includegraphics[width=3.3in]{figures/hist_hMZ2_5.png}
\caption{}
\end{subfigure}
\caption{Dimuon invariant mass at the $ZZ$ candidate selection step for (a) $Z_1$ and (b) $Z_2$.}
\label{fig:dimuonz5}
\end{figure}

\begin{figure}[tbh]
\begin{subfigure}{0.50\textwidth}
\centering
\includegraphics[width=3.3in]{figures/hist_hPtZ1_5.png}
\caption{}
\end{subfigure}
\begin{subfigure}{0.50\textwidth}
\centering
\includegraphics[width=3.3in]{figures/hist_hYZ1_5.png}
\caption{}
\end{subfigure}
\caption{Kinematic distributions for the $Z_1$ at the $ZZ$ candidate selection step: (a) $\pt$ and (b) $\eta$.}
\label{fig:z15kin}
\end{figure}

\begin{figure}[tbh]
\begin{subfigure}{0.50\textwidth}
\centering
\includegraphics[width=3.3in]{figures/hist_hPtZ2_5.png}
\caption{}
\end{subfigure}
\begin{subfigure}{0.50\textwidth}
\centering
\includegraphics[width=3.3in]{figures/hist_hYZ2_5.png}
\caption{}
\end{subfigure}
\caption{Kinematic distributions for the $Z_2$ at the $ZZ$ candidate selection step: (a) $\pt$ and (b) $\eta$.}
\label{fig:z25kin}
\end{figure}

\begin{figure}[tbh]
\begin{subfigure}{0.50\textwidth}
\centering
\includegraphics[width=3.3in]{figures/hist_hPtLep1_5.png}
\caption{}
\end{subfigure}
\begin{subfigure}{0.50\textwidth}
\centering
\includegraphics[width=3.3in]{figures/hist_hPtLep2_5.png}
\caption{}
\end{subfigure}
\begin{subfigure}{0.50\textwidth}
\centering
\includegraphics[width=3.3in]{figures/hist_hPtLep3_5.png}
\caption{}
\end{subfigure}
\begin{subfigure}{0.50\textwidth}
\centering
\includegraphics[width=3.3in]{figures/hist_hPtLep4_5.png}
\caption{}
\end{subfigure}
\caption{Transverse momentum distributions for the constituent leptons of the $Z_1$, (a) and (b), and of the $Z_2$, (c) and (d).}
\label{fig:lep5kin1}
\end{figure}

\begin{figure}[tbh]
\begin{subfigure}{0.50\textwidth}
\centering
\includegraphics[width=3.3in]{figures/hist_hEtaLep1_5.png}
\caption{}
\end{subfigure}
\begin{subfigure}{0.50\textwidth}
\centering
\includegraphics[width=3.3in]{figures/hist_hEtaLep2_5.png}
\caption{}
\end{subfigure}
\begin{subfigure}{0.50\textwidth}
\centering
\includegraphics[width=3.3in]{figures/hist_hEtaLep3_5.png}
\caption{}
\end{subfigure}
\begin{subfigure}{0.50\textwidth}
\centering
\includegraphics[width=3.3in]{figures/hist_hEtaLep4_5.png}
\caption{}
\end{subfigure}
\caption{$\eta$ distributions for the constituent leptons of the $Z_1$, (a) and (b), and of the $Z_2$, (c) and (d).}
\label{fig:lep5kin2}
\end{figure}


$ZZ$ candidates are required to satisfy the following list of requirements:
  \begin{itemize} 
  \item {\bf Ghost removal }: $\Delta R(\eta,\phi) > 0.02$ between each of the four leptons
  \item {\bf Lepton $p_T$}: two of the four selected leptons should pass 
     $p_{T,i} > 20~\GeV$ and $p_{T,j} > 10~\GeV$
  \item {\bf QCD suppression}: all four oppositely-signed pairs that can
     be built with the four leptons (regardless of lepton flavor)
     must satisfy $m_{\ell\ell} > 4~\GeV$.
     Here, selected FSR photons are not used in computing $m_{\ell\ell}$, 
     since a QCD-induced low-mass dilepton (e.g. $J/\Psi$) 
     may have photons nearby (e.g. from $\pi_0$)
  \item {\bf $Z_1$ mass}: $m_{Z1} > 40~\GeV$
  \item {\bf `smart cut'}: defining $Z_a$ and $Z_b$ as 
     the mass-sorted alternative pairing $Z$ candidates 
     ($Z_a$ being the one closest to the nominal $Z$ boson mass),
     require NOT($|m_{ Za}-m_{ Z}| < |m_{ Z1}-m_{ Z}|$ AND $m_{ Zb}<12$).
     Selected FSR photons are included in $m_{ Z}$'s computations.
     This cut discards $4\mu$ and $4e$ candidates where the alternative pairing
     looks like an on-shell $Z$ + low-mass $\ell^+ \ell^-$. 
     In Run 1, such a situation was avoided by choosing the best $ZZ$ candidate
     before applying kinematic cuts to it, most precisely before the $m_{Z2} > 12~\GeV$ cut.
     The present smart cut allows to choose the best $ZZ$ candidate after all kinematic cuts.
The four-lepton invariant mass for the $ZZ$ candidates at this stage is shown in Figure~\ref{fig:m4mu5}.   
\begin{figure}[tbh]
\centering
\includegraphics[width=5.5in]{figures/hist_hM4l_5_log.png}
    \caption{Four-muon invariant mass distribution at the $ZZ$ selection step in log scale.}
    \label{fig:m4mu5}
\end{figure}

  \item {\bf Four-lepton invariant mass}: $\mllll > 70~\GeV$. The four-muon invariant mass distribution after all of the previous selection steps are applied is shown in Figure~\ref{fig:m4mulog}, with a close up view of the $Z$ and $H$ peaks shown in Figure~\ref{fig:m4mulin}. The four-lepton invariant mass distributions for the other decay channels are qualitatively the same
  \end{itemize}	
\item Events containing at least one selected $ZZ$ candidate form the SM signal region.
\end{enumerate}	

\begin{figure}[tbh]
\centering
\includegraphics[width=5.5in]{figures/hist_hMZ1_8.png}
    \caption{Dimuon invariant mass distributions for ${\rm Z_1}$.}
    \label{fig:dimuonz1}
\end{figure}

\begin{figure}[tbh]
\centering
\includegraphics[width=5.5in]{figures/hist_hMZ2_8.png}
    \caption{Dimuon invariant mass distributions for ${\rm Z_2}$.}
    \label{fig:dimuonz2}
\end{figure}

\begin{figure}[tbh]
\centering
\includegraphics[width=5.5in]{figures/hist_hM4l_8_log.png}
    \caption{Four-muon invariant mass distribution in log scale.}
    \label{fig:m4mulog}
\end{figure}

\begin{figure}[tbh]
\centering
\includegraphics[width=5.5in]{figures/hist_hM4l_8_linear.png}
    \caption{Four-muon invariant mass distribution in linear scale.}
    \label{fig:m4mulin}
\end{figure}

\subsection{Choice of the best $ZZ$ candidate}
\label{sec:zzbestcand}

Unlike in the Run 1 analysis, the best $ZZ$ candidate in the Run 2 analysis is chosen after all kinematic cuts, a change that allows to test other selection strategies for this candidate choice. 
This is especially relevant for events with more than four selected leptons, such as $VH$ and $ttH$, with associated particles that can decay to additional leptons.

For the current analysis, if more than one $ZZ$ candidate survives the above selection,
we choose the one with the highest value of $\KD$, the MELA kinematic discriminant, except if
two candidates are composed of four leptons, in which case, the candidate with $Z_1$ closest in mass to nominal 
$Z$ boson mass is chosen.


\section{Signal region and blinding}

A one-step SR and optimization process is used in the current analysis, where a near-optimal SR is defined using all of the desired variables, instead of the two-step procedure used in previous anlayses. This simplification allows for the use of the same SR for all signal models, including the same variable validation and background model. The tradeoff is a small loss in sensitivity for models with tighter optimal cuts, but the benefit of simplicity outweighs the cost of this loss. In principle, each benchmark signal point can be optimized individually to gain a few percent in sensitivity, but this is left to a future study.

Two strategies are employed to define the signal region: cut-and-count based and multivariate analysis (MVA) based. Several key discriminating variables are studied as inputs to both SR definitions: the missing transverse energy, $\MET$ or MET; the four-lepton invariant mass, $m_{4l}$; the transverse mass of the Higgs and MET, $m_T(4l+\MET) \equiv m_T$; the difference in the transverse angle $\phi$ between the Higgs and MET, $|\Delta\phi(4l, \MET)| \equiv |\Delta\phi|$; as well as the lepton and jet multiplicities.

\subsection{Cut-and-count based signal region} \label{cutandcountopt}

The cut-and-count strategy is the simpler choice and is used as a baseline to measure the performance of the MVA. First, the following selection is applied to isolate events with a Higgs from events with additional prompt particles (e.g. VBF Higgs production):

\begin{itemize}
\item Tight-lepton multiplicity = 4
\item b-jet multiplicity $\leq$ 1
\item VBF jet multiplicity $\leq$ 1.
\end{itemize}

Next, the event selection is optimized by scanning over a range of cuts for the remaining variables and selecting the set of cuts that maximizes the sensitivity, measured directly by the cross section upper limit. The two variables with the greatest discriminating power between signal and background are $m_{4l}$ and $\MET$, so we maximize the sensitivity over these two variables. The best sensitivity occurs where the upper limit is minimumized, at approximately:

\begin{itemize}
\item $|m_{4l} - m_H| < 10$ $\GeV$
\item $\MET > 60$ $\GeV$
\end{itemize}

These cuts define the SR, which is applied to all of the signals, losing less than 10\% sensitivity for the models with a tighter optimal cut on $\MET$. Since the signal used to defined the SR has the softest $\MET$ spectrum of all the signals, this SR corresponds to the most modest or, meaning no signal will be cut too hard, while most of the background is still removed. 

%\begin{figure}[tbh]
%\centering
%\includegraphics[width=3in]{figures/}
%\caption{PLACEHOLDER The optimized cut-and-count based selection for ZpBaryonic, defining the signal region for all models.}
%\label{fig:2dscan}
%\end{figure}

Note that there is no additional blinding on the MET distribution above a certain threshold, as was the case previously in similar searches. This is due primarily to the need to understand events that contribute large amounts of fake MET. This is covered thoroughly, including the procedure for removing these events from data, in Section~\ref{sec:fakemet}. In order to validate the modeling in these SRs, they are split into control regions (CR) based on the MET being above or below 60 $\GeV$, referred to as the high and low MET regions. 


\subsection{MVA-based signal region} \label{mvaopt}

This search channel has the advantage of having backgrounds that are easily reduced by applying cuts on the discriminating variables. It was observed in the study of the 2015 data and MC samples that applying additional cuts did not improve the sensitivity since the background levels were already sufficiently low. Applying cuts on additional variables reduced the signal efficiency, which in turn reduced the sensitivity. This is also observed with 2016 MC samples, where applying the $|\Delta\phi(llll, \MET|$ does not improve the sensitivity. These observations motivate the use of MVA techniques, which can take all of the desired variables as inputs, but do not reduce the signal efficiency. Although it is simpler to cut on these discriminating variables using the cut-and-count method, there is potential for significant improvement in the sensitivity with an MVA approach.

The SR event selection is optimized for the MVA-based case by training a boosted decision tree (BDT) with the ROOT TMVA package with $m_{4l}$ and $\MET$ the input variables. Including additional input features does not significantly improve the performance of the MVA. Training is done over the weighted set of simulated backgrounds and an admixture of signal models to reduce bias toward a single signal model. The MVA response is shown in Figure~\ref{fig:bdt}, with the signal peaked toward 1 and the background peaked toward $-1$. 

\begin{figure}[tbh]
\centering
\includegraphics[width=5in]{figures/f_Dm4lmet_1D_log.png}
\caption{MVA response trained on a weighted background sample and admixture of signal samples with input variables $m_{4l}$ and $\MET$. Simulated backgrounds are in pink and signals are in blue.}
\label{fig:bdt}
\end{figure}

\section{Background modeling}

\subsection{Irreducible backgrounds}
\label{sec:irrbkgd}

\subsubsection{$\qqZZ$ modeling}
\label{sec:redbkgd}

The $\qqZZ$ background is generated at NLO, while the fully differential cross section has been computed at 
NNLO~\cite{Grazzini2015407} but is not yet available in a partonic level event generator. Therefore, NNLO/NLO 
$k$-factors for the $\qqZZ$ background process are applied to the {\sc powheg} sample. The inclusive cross 
sections obtained using the same PDF and renormalization and factorization scales as the {\sc powheg} sample
at LO, NLO, and NNLO are shown in Table~\ref{tab:qqZZXS}. The NNLO/NLO $k$-factors are applied in the analysis
differentally as a function of $m(\cPZ\cPZ)$. 

Additional NLO electroweak corrections, which depend on the initial-state quark flavor and kinematics,
are also applied to the $\qqZZ$ background process in the region $m(\cPZ \cPZ)>2m(\cPZ)$. The differential QCD and electroweak $k$-factors can be seen in 
Figure~\ref{fig:qqZZKfactor}.

\begin{table}[h]
    \centering
    \begin{tabular}{l|c|c} 
\hline %----------------------------------------------------------------
QCD Order  & $\sigma_{2\ell2\ell^{\prime}} (\mathrm{fb})$  & $\sigma_{4\ell} (\mathrm{fb})$  \\
\hline %----------------------------------------------------------------
LO    & 218.5$^{+16\%}_{-15\%}$ & 98.4$^{+13\%}_{-13\%}$ \\
NLO   & 290.7$^{+5\%}_{-8\%}$   & 129.5$^{+4\%}_{-6\%}$ \\
NNLO  & 324.0$^{+2\%}_{-3\%}$   & 141.2$^{+2\%}_{-2\%}$ \\
\hline %----------------------------------------------------------------
    \end{tabular}
    \caption{Cross sections for $\qqZZ$ production at 13 \TeV}
    \label{tab:qqZZXS}
\end{table}

%=======
\begin{figure}[!htb]
\vspace*{0.3cm}
\begin{center}
\includegraphics[width=0.48\textwidth]{Figures/IrrBkg/Kfactor_qqZZ_mZZ.pdf}
\includegraphics[width=0.48\textwidth]{Figures/IrrBkg/K_ewk_qqZZ.pdf} 
\caption{Left: NNLO/NLO QCD $k$-factor for the \qqZZ~ background as a function of $m(ZZ)$ for the $4\ell$ and $2\ell2\ell^{\prime}$ final states. Right: NLO/NLO electroweak $k$-factor for the \qqZZ~ background as a function of $m(ZZ)$.
\label{fig:qqZZKfactor}}
\end{center}
\end{figure}


\subsubsection{$\ggZZ$ modeling}

Event simulation for the $\ggZZ$ background is done at LO with the generator MCFM 7.0 \cite{MCFM,Campbell:2011bn,Campbell:2013una}.
Although no exact calculation exists beyond LO for the $\ggZZ$ background, 
it has been recently shown 
that the soft collinear approximation is able to describe the background cross section and the 
interference term at NNLO \cite{Bonvini:1304.3053}. Further calculations also show that the $k$-factors are very similar at NLO for the signal 
and background~\cite{Melnikov:2015laa} and at NNLO for the signal and interference terms~\cite{Li:2015jva}. Therefore, the same $k$-factor 
is used for the signal and background~\cite{Passarino:1312.2397v1}. The NNLO $k$-factor for the signal is obtained as a function of $\mllll$ 
using the \textsc{hnnlo}~v2 MC program~\cite{Catani:2007vq,Grazzini:2008tf,Grazzini:2013mca} by calculating the NNLO and LO 
$\Pg\Pg\to\PH\to2\ell2\ell^\prime$ cross sections at the small $\PH$ decay width of $4.07$ $\MeV$ and taking their ratios. The NNLO amd NLO $k$-factors and the cross sections from which they are derived are illustrated in Figure~\ref{fig:ggHZZXsecKfactor}, 
along with the NNLO, NLO and LO cross sections at the SM $\PH$ boson decay width~\cite{Heinemeyer:2013tqa}.
 
\begin{figure}[!htb]
\centering
\includegraphics[width=0.48\linewidth]{Figures/IrrBkg/cCompare_hnnlo_ggHZZ2l2l_xsec.pdf}
\includegraphics[width=0.48\linewidth]{Figures/IrrBkg/cCompare_hnnlo_ggHZZ2l2l_narrowwidth_xsec.pdf}\\
\includegraphics[width=0.48\linewidth]{Figures/IrrBkg/cCompare_hnnlo_ggHZZ2l2l_narrowwidth_kfactor.pdf}
\caption{$\Pg\Pg\to\PH\to2\ell2\ell^\prime$ cross sections at NNLO, NLO and LO at each $\PH$ pole mass using the SM $\PH$ decay width (top left) or at the fixed and small decay width of $4.07$ $\MeV$ (top right). The cross sections using the fixed value are used to obtain the $k$-factor for both the signal and the continuum background contributions as a function of $\mllll$ (bottom).
}
\label{fig:ggHZZXsecKfactor}
\end{figure}

%\begin{figure}[!htb]
%\centering
%\includegraphics[width=0.48\linewidth]{Figures/IrrBkg/cCompare_hnnlo_ggHZZ2l2l_xsec.pdf}
%\includegraphics[width=0.48\linewidth]{Figures/IrrBkg/cCompare_hnnlo_ggHZZ2l2l_narrowwidth_xsec.pdf}\\
%\includegraphics[width=0.48\linewidth]{Figures/IrrBkg/cCompare_hnnlo_ggHZZ2l2l_narrowwidth_kfactor.pdf}
%\caption{$\Pg\Pg\to\PH\to2\ell2\ell^\prime$ cross sections at NNLO, NLO and LO at each $\PH$ boson pole mass using the SM $\PH$ boson decay width  (top \cmsLeft) or at the fixed and small decay width of $4.07$~MeV (top \cmsRight). The cross sections using the fixed value are used to obtain the K factor for both the signal and the continuum background contributions as a function of $\mllll$ (bottom).
%}
%\label{fig:ggHZZXsecKfactor}
%\end{figure}

\subsection{Reducible background estimation}
\label{sec:zxIntr}
%\input{Background/zxIntr.tex}

The reducible background for the $H\to ZZ\to 4\ell $ analysis, hereafter called $Z+X$, originates from processes which contain one or more non-prompt lepton in the four-lepton final state. 
The main sources of non-prompt leptons are non-isolated electrons and muons coming from decays of heavy-flavor mesons, mis-reconstructed jets (usually originating from light-flavor quarks), and electrons from $\gamma$-conversions. 
A fake lepton is defined as any jet mis-reconstructed as a lepton and any lepton originating from a heavy meson decay.
Similarly, any electron originating from a photon conversion will be considered a fake electron.

The lepton fake rates, $f_{e}$ and $f_{\mu}$, are defined as the ratio of the number of electrons/muons passing the tight selection criteria to the number passing the loose criteria. This measures the probability of a lepton passing the loose criteria to also pass the tight criteria. 
The fake rates are applied in dedicated control samples in order to extract the expected background yield in the SR. 

\subsubsection{Fake rate determination}
In order to measure the lepton fake rates $f_{e}$ and $f_{\mu}$, samples of $Z(\ell\ell)+e$ and $Z(\ell\ell)+\mu$ events are selected that are expected to be completely dominated by final states that include a $Z$ boson and a fake lepton. 
These events are required to have two same-flavor, oppositely-charged leptons with $p_{T} > 20/10$ $\GeV$ passing the tight selection criteria, thus forming the $Z$ candidate. In addition, the event must have exactly one lepton passing the loose selection criteria. 
This lepton is used as the probe lepton for the fake rate measurement. The invariant mass of the probe lepton and the opposite sign lepton from the reconstructed $Z$ candidate must satisfy $m_{2l} > 4$ $\GeV$. 

% Figure \ref{fig:Z1L_LooseTightVsMZ} shows distributions of the invariant mass of two leptons selected as the ones originating from the Z decay: in case of all loose leptons (top) and in case of loose leptons that pass the tight selection criteria (middle). The distributions for tight leptons show the presence of (asymmetric) conversion of photons that end up with one electron being reconstructed. Figure \ref{fig:Z1L_LooseTightVsMZ} also shows dependance of the electron and muon fake ratios on $M_{inv}(\ell_{1},\ell_{2})$ (bottom). From that it can be concluded that we would benefit by using a tight mass cut on $|M_{inv}(\ell_{1},\ell_{2}) - M_{Z}| < 7$~GeV. By observing the Figure \ref{fig:Z1L_LooseTightVsMET} it can also be concluded that the contamination from $WZ$ and $t \bar{t}$ processes can be greatly suppress by using a $E_{\mathrm{T}}^\text{miss}  < $ 25~GeV.

% %%%%%%%%%%%%%%%%%%%%%%%%
% \begin{figure}[!htb]
% \begin{center}
%    \subfigure [] {\resizebox{7.75cm}{!}{\includegraphics{Figures/RedBkg/mZ1el_EB_d_process_auto.pdf}}} 
%    \subfigure [] {\resizebox{7.75cm}{!}{\includegraphics{Figures/RedBkg/mZ1mu_EB_d_process_auto.pdf}}} \\
%    \subfigure [] {\resizebox{7.75cm}{!}{\includegraphics{Figures/RedBkg/mZ1el_EB_n_process_auto.pdf}}} 
%    \subfigure [] {\resizebox{7.75cm}{!}{\includegraphics{Figures/RedBkg/mZ1mu_EB_n_process_auto.pdf}}} \\
%   \caption{
% Distribution of the invariant mass of two leptons selected as the ones originating form the Z decay. Distributions are shown for the $Z(\ell\ell)+e$ (left) and $Z(\ell\ell)+\mu$ (right) samples, as defined in the text. The top row corresponds to all the events where we have the additional lepton passing the loose criteria, while middle row shows distributions for events when the loose lepton passes the tight selection criteria. }
% \label{fig:Z1L_LooseTightVsMZ}
% \end{center}
% \end{figure}
% %%%%%%%%%%%%%%%%%%%%%%%
% \begin{figure}[!htb]
% \begin{center}
%  \subfigure [] {\resizebox{7.75cm}{!}{\includegraphics{Figures/RedBkg/FR_electrons_mZ1_Data.pdf}}} 
%  \subfigure [] {\resizebox{7.75cm}{!}{\includegraphics{Figures/RedBkg/FR_muons_mZ1_Data.pdf}}} 
% \caption{
% Fake Rate at which electrons (muons), that pass the loose criteria, also pass the tight criteria. Distributions of $|M_{inv}(\ell_{1},\ell_{2})|$ show dependence of the fake ratios in the region below 85~GeV.
% }
% \label{fig:Z1L_LooseTightVsMZFR}
% \end{center}
% \end{figure}
% %%%%%%%%%%%%%%%%%%%%%%%
% \begin{figure}[!htb]
% \begin{center}
%    \subfigure [] {\resizebox{7.75cm}{!}{\includegraphics{Figures/RedBkg/mEtel_EB_d_process_auto.pdf}}} 
%    \subfigure [] {\resizebox{7.75cm}{!}{\includegraphics{Figures/RedBkg/mEtmu_EB_d_process_auto.pdf}}} \\
%    \subfigure [] {\resizebox{7.75cm}{!}{\includegraphics{Figures/RedBkg/mEtel_EB_n_process_auto.pdf}}} 
%    \subfigure [] {\resizebox{7.75cm}{!}{\includegraphics{Figures/RedBkg/mEtmu_EB_n_process_auto.pdf}}}
% \caption{
% Distribution of the of $E_{\mathrm{T}}^\text{miss}  $. Distributions are shown for the $Z(\ell\ell)+e$ (left) and $Z(\ell\ell)+\mu$ (right) samples, as defined in the text. The top row corresponds to all the events where we have the additional lepton passing the loose criteria, while bottom row shows distributions for events when the loose lepton passes the tight selection criteria.
% }
% \label{fig:Z1L_LooseTightVsMET}
% \end{center}
% \end{figure}
% %%%%%%%%%%%%%%%%%%%%%%%

The fake rates are evaluated using the tight requirement 
$|M_{inv}(\ell_{1},\ell_{2}) - M_{Z}| < 7 $ $\GeV$ to reduce the asymmetric
contribution from photon conversions populating low masses and using $\MET<25\ \GeV$ to reduce contamination from QCD events.
The muon fake rates measured in bins of the transverse momentum of the loose lepton in the barrel and end cap regions are shown in Figure~\ref{fig:os_fakerates}. 

\begin{figure}[tbh]
\centering
%\begin{subfigure}{0.95\textwidth}
%\centering
%\includegraphics[width=4.25in]{Figures/RedBkg/FR_electrons_ptl3_DataallTR.pdf}
%\caption{}
%\end{subfigure}
%\begin{subfigure}{0.95\textwidth}
%\centering
\includegraphics[width=4.25in]{Figures/RedBkg/FR_muons_ptl3_DataallTR.pdf}
%\caption{}
%\end{subfigure}
  \caption{
Fake rates as a function of the probe $p_T$ for electrons which satisfy the loose selection criteria. The fake rates are measured in
%Fake rates as a function of the probe $p_T$ for electrons (a) and muons (b) which satisfy the loose selection criteria. The rake rates are measured in
a $Z(\ell\ell)+\ell$ sample in the $13$ $\TeV$ data.
The barrel selection includes muons up to $|\eta|$ = 1.2.
%The barrel selection includes electrons (muons) up to $|\eta|$ = 1.479 (1.2).
}
\label{fig:os_fakerates}
\end{figure}

%\begin{figure}[!htb]
%\begin{center}
%    \subfigure [] {\includegraphics [width=0.45\textwidth]{Figures/RedBkg/FR_electrons_ptl3_DataallTR.pdf}}
%    \subfigure [] {\includegraphics [width=0.45\textwidth]{Figures/RedBkg/FR_muons_ptl3_DataallTR.pdf}}
%  \caption{
%Fake rates as a function of the probe $p_T$ for  electrons (a) and muons (b) which satisfy the loose selection criteria, measured in
%a $Z(\ell\ell)+\ell$ sample in the $13$~TeV data.
%The barrel selection includes electrons (muons) up to $|\eta|$ = 1.479 (1.2).
%}
%\label{fig:os_fakerates}
%\end{center}
%\end{figure}



\subsubsection{Fake rate application}
\label{sec:zxA}

Two control regions (CRs) are defined as subsets of four-lepton events
which pass the first step of the selection (see
Section~\ref{sec:zzcandsel}), requiring an additional pair of 
same-flavor, oppositely-charged loose leptons, that pass the ${\rm SIP_{3D}}$ cut. 
The events must satisfy all kinematic cuts applied for the Higgs phase space selection
(see Section~\ref{sec:zzcandsel}).

The first CR is obtained by
requiring that the two loose leptons that do not make the $Z_1$ candidate 
do not pass the final identification and
isolation criteria.
The other two leptons pass
the final selection criteria by definition of the $Z_1$. 
This sample is denoted as the ``2
Fail'' (FF) sample, referring to the two leptons that do not form $Z_1$. The FF CR is expected to be populated with
events that intrinsically have only two prompt leptons, mostly $DY$, with a small fraction of $t \bar{t}$ and $Z \gamma$ events.

The second CR is obtained by requiring one of
the four leptons to fail the final identification and isolation
criteria.
The other three
leptons should pass the final selection criteria. This control sample
is denoted as ``1 Fail + 1 Prompt'' (FP), referring to the two leptons that do not form the $Z_1$. The FP CR is
expected to be populated with the type of events that populate the
FF CR, 
albeit with different relative proportions,
as well as with $WZ$ events that intrinsically have three
prompt leptons. 


%\begin{figure}[!htb]
%\begin{center}
%    {\includegraphics [width=0.45\textwidth] {Figures/RedBkg/HZZ_3P1Fuw_ZZMass_4m.pdf}}
%    {\includegraphics [width=0.45\textwidth] {Figures/RedBkg/HZZ_3P1Fuw_ZZMass_4e.pdf}} \\
%    {\includegraphics [width=0.45\textwidth] {Figures/RedBkg/HZZ_3P1Fuw_ZZMass_2m2e.pdf}}
%    {\includegraphics [width=0.45\textwidth] {Figures/RedBkg/HZZ_3P1Fuw_ZZMass_2e2m.pdf}} \\
%\caption{
%Invariant mass distribution of the events selected in the 3P+1F control sample in the
%$13$ $\TeV$ dataset, (top left)  $4\mu$ , (top right) $4e$ , (bottom left)  $2\mu2e$ and (bottom right)  $2e2\mu$ channels.
%}
%\label{fig:CR_3P1F}
%\end{center}
%\end{figure}

%\begin{figure}[!htb]
%\begin{center}
%    {\includegraphics [width=0.85\textwidth] {Figures/RedBkg/HZZ_2P2Fuw_ZZMass_4m.pdf}}
%    {\includegraphics [width=0.45\textwidth] {Figures/RedBkg/HZZ_2P2Fuw_ZZMass_4e.pdf}} \\
%    {\includegraphics [width=0.45\textwidth] {Figures/RedBkg/HZZ_2P2Fuw_ZZMass_2m2e.pdf}}
%    {\includegraphics [width=0.45\textwidth] {Figures/RedBkg/HZZ_2P2Fuw_ZZMass_2e2m.pdf}} \\
%\caption{
%Four-muon invariant mass distribution of the events selected in the 2P+2F control sample in the
%$13$ $\TeV$ dataset.
%$13$ $\TeV$ dataset, (top left)  $4\mu$ , (top right) $4e$ , (bottom left)  $2\mu2e$ and (bottom right)  $2e2\mu$ channels.
%}
%\label{fig:2P2F_dataMC}
%\end{center}
%\end{figure}

%\begin{figure}[!htb]
%\begin{center}
%    {\includegraphics [width=0.45\textwidth] {Figures/RedBkg/HZZ_3P1Fuw_ZZMass_4m.pdf}}
%    {\includegraphics [width=0.45\textwidth] {Figures/RedBkg/HZZ_3P1Fuw_ZZMass_4e.pdf}} \\
%    {\includegraphics [width=0.45\textwidth] {Figures/RedBkg/HZZ_3P1Fuw_ZZMass_2m2e.pdf}}
%    {\includegraphics [width=0.45\textwidth] {Figures/RedBkg/HZZ_3P1Fuw_ZZMass_2e2m.pdf}} \\
%\caption{
%Invariant mass distribution of the events selected in the 3P+1F control sample in the
%$13$ $\TeV$ dataset, (top left)  $4\mu$ , (top right) $4e$ , (bottom left)  $2\mu2e$ and (bottom right)  $2e2\mu$ channels.
%}
%\label{fig:CR_3P1F}
%\end{center}
%\end{figure}

%\begin{figure}[!htb]
%\begin{center}
%    {\includegraphics [width=0.45\textwidth] {Figures/RedBkg/HZZ_3P1Fuw_ZZMass_4m.pdf}}
%    {\includegraphics [width=0.45\textwidth] {Figures/RedBkg/HZZ_3P1Fuw_ZZMass_4e.pdf}} \\
%    {\includegraphics [width=0.45\textwidth] {Figures/RedBkg/HZZ_3P1Fuw_ZZMass_2m2e.pdf}}
%    {\includegraphics [width=0.45\textwidth] {Figures/RedBkg/HZZ_3P1Fuw_ZZMass_2e2m.pdf}} \\
%\caption{
%Invariant mass distribution of the events selected in the 3P+1F control sample in the
%$13$ $\TeV$ dataset, (top left)  $4\mu$ , (top right) $4e$ , (bottom left)  $2\mu2e$ and (bottom right)  $2e2\mu$ channels.
%}
%\label{fig:CR_3P1F}
%\end{center}
%\end{figure}

The CRs obtained in this way, orthogonal by construction
to the SR, are enriched with fake leptons and are used to
estimate the reducible background in the SR. 
The invariant mass distribution of events selected in the FF control sample
is shown in Figure~\ref{fig:FF}(a). 
%The fake rate scale factors derived from the
%FF distribution are applied in the PF region, shown in Figure~\ref{fig:FF}(b).
The expected number of reducible background events in the FP region,
$N^{\rm bkg}_{\rm FP}$, can be computed from the number of events
observed in the FF control region, $N_{\rm FF}$, by weighting each
event with the factor $(\frac{f_{i}}{1-f_{i}}
+ \frac{f_{j}}{1-f_{j}})$, where $f_{i}$ and $f_{j}$ correspond to the
fake rates of the two loose leptons:

\begin{equation} 
\label{eq:Prediction3P1F}
N^{\rm bkg}_{\rm FP} = \sum (\frac{f_{i}}{1-f_{i}}
+ \frac{f_{j}}{1-f_{j}}) N_{\rm FF}
\end{equation} 
Figure~\ref{fig:FF}(b) shows the invariant mass distributions of the
events selected in the FP CR, together with the expected
reducible background estimated from Equation~\ref{eq:Prediction3P1F} in red,
stacked on the distribution
of $WZ$ and of irreducible background ($ZZ, Z\gamma* \to 4\ell$) taken from simulation.

\begin{figure}[!htb]
\begin{center}
\begin{subfigure}{0.49\textwidth}
\centering
    {\includegraphics [width=0.95\textwidth] {Figures/RedBkg/HZZ_2P2Fuw_ZZMass_4m.pdf}}
\caption{}
\end{subfigure}
%    {\includegraphics [width=0.45\textwidth] {Figures/RedBkg/HZZ_2P2Fuw_ZZMass_4e.pdf}} \\
%    {\includegraphics [width=0.45\textwidth] {Figures/RedBkg/HZZ_2P2Fuw_ZZMass_2m2e.pdf}}
%    {\includegraphics [width=0.45\textwidth] {Figures/RedBkg/HZZ_2P2Fuw_ZZMass_2e2m.pdf}} \\
\begin{subfigure}{0.49\textwidth}
\centering
    {\includegraphics [width=0.95\textwidth] {Figures/RedBkg/HZZ_3P1Fuw_ZZMass_4m.pdf}}
\caption{}
\end{subfigure}
\caption{
(a) Four-muon invariant mass distribution of events selected in the ``2 Fake'' control sample. (b) Four-muon invariant mass distribution in the ``1 Prompt + 1 Fake'' control sample, with the extrapolated estimate shown in red.
%$13$ $\TeV$ dataset, (top left)  $4\mu$ , (top right) $4e$ , (bottom left)  $2\mu2e$ and (bottom right)  $2e2\mu$ channels.
}
\label{fig:FF}
\end{center}
\end{figure}

If the fake rates were measured in a sample that had exactly the same
background composition as the FF sample, the difference between the
observed number of events in the FP sample and the expected background
predicted from the FF sample would solely amount to the $WZ$ and $Z\gamma_{conv}$
contribution, which is small. Large differences arise because the fake rates used in
Equation~\ref{eq:Prediction3P1F} do not properly account for the background
composition of the FF control sample.

%The difference seen in Figure~\ref{fig:CR_3P1F} between the observed
%3P+1F distribution and the expectation from 2P+2F in the
%channels with loose electrons ($4e$ and $2\mu 2e$), concentrated at low
%masses, is due to photon conversions. This is confirmed explicitly by simulation.
%% : Fig.~\ref{fig:CR_3P1F}c shows how events with a real photon populate
%% this low mass region. Indeed, as the fake rates of method A are measured
%% in a sample that is largely devoid of photon conversions, Eq.~\ref{eq:Prediction3P1F}
%% considerably underestimates their contribution to the 3P+1F sample. 
%%
%The difference between the 3P+1F observation and the prediction
%from 2P+2F is corrected to recover the missing contribution from conversions.
%More precisely, the expected reducible background in the SR is given
%by the sum of two terms: a 2P2F component and a 3P1F component.

The FF component is obtained from the number of
  events observed in the FF control region, $N_{\rm FF}$, by
  weighting each event in that region with the factor
  $\frac{f_{i}}{1-f_{i}} \frac{f_{j}}{1-f_{j}}$, where $f_{i}$ and
  $f_{j}$ correspond to the fake rates of the two loose leptons.
The FP component is obtained from the
   difference between the number of observed events in the FP control
   region, $N_{\rm FP}$, and the expected contribution from the FF
   region and $ZZ$ processes in the signal region, $N^{\rm ZZ}_{\rm FP} +
   N^{\rm bkg}_{\rm FP}$. The $N^{\rm bkg}_{\rm FP}$ is given by 
   Equation~\ref{eq:Prediction3P1F} and $N^{\rm ZZ}_{\rm 3P1F}$ is the
   contribution from $ZZ$ which is taken from simulation. 
   The difference $N_{\rm FP} -  N^{\rm bkg}_{\rm FP} - N^{\rm ZZ}_{\rm FP}$,
   which may be negative,
   is obtained for each $(p_T, \eta)$ bin of the probe lepton and is weighted 
   by $\frac{f_i} {1 - f_i}$, where $f_i$ denotes the fake rate of
   this lepton.
   %This FP component accounts for the contribution of reducible background
   %processes with only one fake lepton (like $WZ$ events) and for the contribution
   %of other processes (e.g. photon conversions) that are not properly estimated
   %by the 2P2F component because of the fake rates used.

The full expression for the prediction can be symbolically written as:
%
\begin{equation} 
\label{eq:PredictionSR}
N^{bkg}_{\rm SR} = \sum \frac{f_{i}}{(1-f_{i})} (N_{\rm FP} - N^{\rm
bkg}_{\rm FP} - N^{\rm ZZ}_{\rm FP})
+ \sum \frac{f_{i}}{(1-f_{i})} \frac{f_{j}}{(1-f_{j})}N_{\rm FF} \end{equation}
%a
or equivalently:
\begin{equation}
\label{eq:PredictionSR2}
N^{bkg}_{\rm SR}= (1-\frac{N_{FP}^{ZZ}}{N_{FP}})\sum_j^{N_{FP}}\frac{f_a^j}{1-f_a^j} - \sum_i^{N_{FF}}\frac{f_3^i}{1-f_3^i}\frac{f_4^i}{1-f_4^i}
\end{equation}
%For channels where the $Z_2$ candidate is made from two electrons, 
%the contribution of the FP component is 
%positive and amounts to typically $30 \%$ of the total predicted background.
%For channels with loose muons (e.g. $4\mu$ and $2e2\mu$), the FP sample is rather well described by
%the prediction from FF, and the
%FP component is mainly driven by statistical fluctuations in the 3P+1F sample,
%which are larger than the expectation from $WZ$ production.
%Table~\ref{tab:reducibleMethodA} shows the expected number of
%events in the four-lepton SRs from the reducible background processes. 
%% The first error is the statistical uncertainty, which is dominated by the
%% large statistical uncertainty of the ``3P1F" component. The second error
%% is a systematic uncertainty due to the statistical uncertainty of the
%% fake rates.
%\begin{table}[h]
%\begin{center}
%     \begin{tabular}{ l | c | c | c | c } \hline
% baseline	& $4e$ 	 & 4$\mu$ & $2e2\mu$  & 2$\mu2e$   \\ \hline
% 13 $\TeV$		& $22.19$ & $32.81$ & $22.48$    & $41.72$  \\  \hline
% 	\end{tabular}
%\end{center}
%    \caption{ The contribution of reducible background
%    processes in the four-lepton signal region predicted from measurements in data
%    using the Opposite-Sign Leptons method. The predictions correspond to \usedLumi of data at 13 $\TeV$.}
%     \label{tab:reducibleMethodA}
%\end{table}
The expected number of events in the four-muon SR from the reducible background processes is $32.81$. The systematic uncertainty associated with this measurement is discussed in Section~\ref{sec:zxUncert}.


\section{Event yields}

Table~\ref{tab:smyields} shows the event yields for the primary backgrounds, a benchmark signal, and the observed events at the major event selection stages. This is called the analysis cutflow. The selection of tight leptons and the formation of the first $Z$ candidate are sufficient to remove most of the background events. After the second $Z$ is formed, the remaining selection steps do not reduce backgrounds significantly but assist in the selection of events with clean $ZZ^*$ candidates. 

\begin{table*}[htbH]
\begin{center}
\resizebox{\textwidth}{!}{%
\begin{tabular}{ l | c | c | c | c | c | c | c | c }
\hline
Sample & $q\bar{q} \rightarrow ZZ$ & $gg \rightarrow ZZ$ & $Z+X$ & $ZH$ & Other $H$ & Total & Signal & Observed \\
\hline
Initial & 1.89e+04 & 866 & 5.88e+08 & 17.4 & 377 & 5.88e+08 & 3.57 & 8.21e+07 \\
%\hline
HLT & 1.89e+04 & 866 & 5.88e+08 & 17.4 & 377 & 5.88e+08 & 3.57 & 8.21e+07 \\
%\hline
$Z_1$ lepton cuts & 1.94e+03 & 144 & 5.9e+04 & 2.35 & 57 & 6.12e+04 & 0.516 & 7.76e+04 \\
%\hline
$m_{Z_1}$ & 1.58e+03 & 128 & 4.39e+04 & 2.13 & 53.8 & 4.57e+04 & 0.484 & 3.52e+04 \\
%\hline
At least one $Z_2$ & 379 & 76.4 & 30.3 & 0.987 & 24.3 & 511 & 0.228 & 367 \\
%\hline
$m_{Z_2}$ & 379 & 76.4 & 56.7 & 0.987 & 24.3 & 537 & 0.228 & 367 \\
%\hline
$m_{ll}>4$ for OS-SF & 312 & 33.7 & 27.7 & 0.92 & 11.3 & 386 & 0.213 & 367 \\
%\hline
$m_{llll} > 70$ $\GeV$ & 312 & 33.7 & 27.7 & 0.92 & 11.3 & 385 & 0.213 & 365 \\
\hline
\end{tabular}}
\caption{Cutflow table for the $4\mu$ channel. The benchmark signal sample shown is Zp2HDM with $m_{Z'}=600$ $\GeV$.}\label{tab:smyields}
\end{center}
\end{table*}

The final yields in the cut-and-count based SR are shown in Table~\ref{tab:cacyields}. 
Since the MVA SR definition does not apply additional cuts in the event selection, the yields in this SR are the same as in the last step of Table~\ref{tab:smyields}. These yields are shown with uncertainties in Table~\ref{tab:mvayields}.
%The final yields in the MVA-based SR, requiring the MVA response to be greater than 0.9, are shown in Table~\ref{tab:mvayields}. 


\begin{table*}[htbH]
\begin{center}
\resizebox{\textwidth}{!}{%
\begin{tabular}{ l | c | c | c | c | c | c | c }
\hline
Sample & $q\bar{q} \rightarrow ZZ$ & $gg \rightarrow ZZ$ & $Z+X$ & $H$ & Total & Signal & Observed \\
\hline
Yield & $0.35 \pm 0.00 \pm_{\ 0.00}^{\ 0.00}$ & $0.034 \pm 0.00 \pm_{\ 0.00}^{\ 0.00}$ & $0.023 \pm 0.00 \pm_{\ 0.00}^{\ 0.00}$ & $0.33 \pm 0.00 \pm_{\ 0.00}^{\ 0.00}$ & $0.74 \pm 0.00 \pm_{\ 0.00}^{\ 0.00}$ & $0.20 \pm 0.00 \pm_{\ 0.00}^{\ 0.00}$ & $1.0 \pm 0.00 \pm_{\ 0.00}^{\ 0.00}$ \\
\hline
\end{tabular}}
\caption{Final signal region yields using the cut-and-count selection strategy for the $4\mu$ channel with statistical and systematic uncertainties. The benchmark signal sample shown is Zp2HDM with $m_{Z'}=600$ $\GeV$.}\label{tab:cacyields}
\end{center}
\end{table*}

\begin{table*}[htbH]
\begin{center}
\resizebox{\textwidth}{!}{%
\begin{tabular}{ l | c | c | c | c | c | c | c }
\hline
Sample & $q\bar{q} \rightarrow ZZ$ & $gg \rightarrow ZZ$ & $Z+X$ & $H$ & Total & Signal & Observed \\
\hline
Yield & $117.5 \pm 0.00 \pm_{\ 0.00}^{\ 0.00}$ & $4.54 \pm 0.00 \pm_{\ 0.00}^{\ 0.00}$ & $17.6 \pm 0.00 \pm_{\ 0.00}^{\ 0.00}$ & $11.7 \pm 0.00 \pm_{\ 0.00}^{\ 0.00}$ & $151.3 \pm 0.00 \pm_{\ 0.00}^{\ 0.00}$ & $0.21 \pm 0.00 \pm_{\ 0.00}^{\ 0.00}$ & $151.0 \pm 0.00 \pm_{\ 0.00}^{\ 0.00}$ \\
\hline
\end{tabular}}
\caption{Final signal region yields using the MVA selection strategy for the $4\mu$ channel with statistical and systematic uncertainties. The benchmark signal sample shown is Zp2HDM with $m_{Z'}=600$ $\GeV$.}\label{tab:mvayields}
\end{center}
\end{table*}


%\include{ucdavisthesis_Chap7}
%\include{ucdavisthesis_Chap8}
\chapter{Statistical analysis results}

\section{Systematic uncertainties}

This section covers the systematic, including statistical, errors associated with the analyis. The general strategy is to duplicate the systematics applied in the SM $H\rightarrow ZZ$ search, including errors on the background estimation from data, while adding the errors associated with MET modeling, guided by the mono-$H \rightarrow \gamma\gamma$ strategy \cite{CMS-AN-15-203}. 

% Uncertainties
\subsection{Uncertainties on the reducible background estimation} %using Opposite-Sign leptons}
\label{sec:zxUncert}


The main source of systematic uncertainty on the background estimation method is due to the different compositions of the reducible background processes ($DY$, $t \bar{t}$, $WZ$, $Z\gamma^{(*)}$) in the regions where the fake rates are measured and where they are applied.
The OS method corrects for the resulting bias with the 3P1F component of its prediction.
The closure tests presented here are used to assess a possible residual bias in the OS method.

The systematic uncertainty due to different compositions of events can be estimated by measuring the fake rates for individual background processes in the $Z+1L$ region in simulation. The weighted average of these individual fake rates is used for the overall fake rate. The exact composition of the background processes in the 2P+2F region where the fake rates are applied can be determined from simulation. The individual fake rates can then be reweighted according to the 2P+2F composition. The difference between the reweighed fake rate and the average one can be used as a measure of the uncertainty on the measurement of the fake rates. The effect of this systematic uncertainty is propagated to the final estimates, and it amounts to approximately 32\% for $4e$, 33\% for $2e2\mu$, and 35\% for the $4\mu$ final state. 

% \begin{table}[h]
% \scriptsize
%     \centering
%     \caption{
%     The fake ratios for individual background processes, the average fake ratio and the fake ratio reweighed according to the composition of backgrounds in 2P+2F region.}  
    
%     \begin{tabular}[!htb]{| l | c | c | c | c | c | c |} \hline
% Sample				& Light Jets		  & HF Jets& From $\gamma$ conv	    & average (in $Z+1\ell$) & reweighed (2P2F)	& uncertainty \\ \hline \hline
% electron fake ratio	& $0.012^{+0.001}_{-0.001}$ 	& $0.115^{+0.005}_{-0.006}$  & $0.293^{+0.043}_{-0.043}$  &$0.021^{+0.001}_{-0.001}$& $0.024^{+0.004}_{-0.004}$& 15\%    \\ 
% muon fake ratio 	& $0.057^{+0.003}_{-0.003}$ 	& $0.225^{+0.008}_{-0.008}$ & $0.003^{+0.455}_{-0.003}$ &$0.120^{+0.004}_{-0.004}$& $0.105^{+0.020}_{-0.017}$ & 13\% \\  \hline
% 	\end{tabular}
%     \label{tab:uncertFR}
% \end{table}

Additional uncertainties arise from the limited size of the samples in the four-lepton control regions, where the fake rates are measured and applied contributes a statistical uncertainty. The dominate statistical uncertainty is driven by the number of events in the control region and is typically in the range of 1--10\%.

In order to estimate the uncertainty on the $m_{4l}$ shape, the differences among the shapes of predicted background distributions for all three channels are studied. The envelope of differences among these distribution shapes is used as an estimate of the shape uncertainty. The uncertainty is estimated to be roughly in the range of 5--15\%. Since the difference of the shapes slowly varies with $m_{4l}$, it is taken as a constant versus $m_{4l}$ and is absorbed in the much larger uncertainty on the predicted yield of backgrounds. 
%The shapes of predicted background $m_{4l}$ distributions for the three channels are shown in Figure~\ref{fig:SR_CombinedShapes} (left).	


\subsection{MET systematics}\label{sec:metsyst}

There are two types of systematic uncertainties related to the modeling of MET: (1) those from the measurement of real MET, as from the signal samples or backgrounds with neutrinos, and (2) those from fake MET, due to the mismeasurement of jets and other objects. The fake MET systematics apply to the $H$ signals with no associated $W$ production and to the non-resonant backgrounds. 

The uncertainties from the mismodeling of real MET are measuremed by varying several corrections used to calculate MET, then propogating these variations to the efficiency of MC samples to pass the MET selection \cite{mettwiki}. The corrections used in this calculation are:
jet energy,
jet resolution,
muon energy,
electron energy,
tau energy,
photon energy, and
unclustered jet energy. Each correction is varied up and down by one standard deviation of the input distribution, with the systematic uncertainty taken as the maximim difference in efficiencies accross all correction variations. The efficiencies for the V+H samples to pass the MET selection vary by a few percent and the variation in signal sample efficiencies is less than one percent.
%The efficiencies for the V+H and a benchmark signal sample to pass the MET selection of $E_{\rm{T}}^{\rm{MISS}}>60$ GeV after each correcion is varied up and down are given in Table~\ref{tab:} and Table~\ref{tab:}, respectively. 

%\begin{table*}[htbH]
%\begin{center}
%\topcaption{Efficiencies for V+H sample to pass MET selection after varying corrections up and down.}\label{tab:yields}
%\begin{tabular}{ l | c | c }
%\hline
%\hline
%Correction & Efficiency Up & Efficiency Down \\
%\hline
%Original PFMET & & \\
%\hline
%Jet energy & & \\
%\hline
%Jet resolution & & \\
%\hline
%Muon energy & & \\
%\hline
%Electron energy & & \\
%\hline
%Tau energy & & \\
%\hline
%Photon energy & & \\
%\hline
%Unclustered jet energy & & \\
%\hline
%\hline
%\end{tabular}
%\end{center}
%\end{table*}


%\begin{table*}[htbH]
%\begin{center}
%\topcaption{Efficiencies for a benchmark signal sample, Zp2HDM(600), to pass MET selection after varying corrections up and down.}\label{tab:yields}
%\begin{tabular}{ l | c | c }
%\hline
%\hline
%Correction & Efficiency Up & Efficiency Down \\
%\hline
%Original PFMET & & \\
%\hline
%Jet energy & & \\
%\hline
%Jet resolution & & \\
%\hline
%Muon energy & & \\
%\hline
%Electron energy & & \\
%\hline
%Tau energy & & \\
%\hline
%Photon energy & & \\
%\hline
%Unclustered jet energy & & \\
%\hline
%\hline
%\end{tabular}
%\end{center}
%\end{table*}

The second systematic uncertainty results from the mismodeling of fake MET, primarily due to the mismeasurement of jets (see Section~\ref{sec:zxIntr}). This uncertainty is measured in the sideband CR as the percent difference between the efficiency for the data and total background sample to pass the MET selection. These efficiencies differ by 42\%, which is taken as the systematic on backgrounds without real MET.

\subsection{Additional systematics}

Both signal and background samples are affected by several additional systematic uncertainties, including the uncertainty on the integrated luminosity
(2.6\%) and the uncertainty on the lepton identification and reconstruction efficiency (ranging from 2.5--9\%). Experimental uncertainties for the reducible background estimation, 
described in Section~\ref{sec:redbkgd},
vary between 36\% ($4\mu$)  and 43\% ($4e$).  The uncertainty on the lepton energy scale is determined by considering the 
$Z\rightarrow\ell\ell$ mass distributions in data and simulation. Events are separated into categories based on the 
$\pt$ and $\eta$ of one of the two leptons, determined randomly, and integrating over the other. The dilepton mass 
distributions are then fit to a Breit-Wigner 
parameterization convolved with a double-sided Crystal-Ball function. The offset in the measured peak position with 
respect to the nominal $\cPZ$ boson 
mass in data and simulation are extracted. The results are shown in Figure~\ref{fig:lepScale}. The relative difference 
between data and simulation is propagated to the reconstructed four-lepton mass 
from simulated $H$ events. The results of the propagation can be seen in Figure~\ref{fig:lepScaleM4l}.
In the case of electrons, since the same dataset is used to derive and validate the 
momentum scale corrections, the size 
of the corrections are taken into account for the final value of the uncertainty.
The uncertainty is determined to be 0.04\% (0.3\%) for the  $4\mu$ ($4\Pe$) channel. The uncertainty 
on the $4\ell$ mass resolution coming from the uncertainty on the per-lepton energy resolution is 20\%. The experimental systematic uncertainties are summarized in Table~\ref{tab:SystOverview}.


\begin{figure}[!htb]
\begin{center}
\includegraphics[width=0.48\linewidth]{Figures/Results/mass/lepScale_vs_pt_eta_mu.pdf}
\includegraphics[width=0.48\linewidth]{Figures/Results/mass/lepScale_vs_pt_eta_e.pdf}
\caption{ Difference between the ${\rm Z}\rightarrow\ell\ell$ mass peak positions in data and simulation normalized by the 
nominal $\cPZ$ boson mass obtained as a function of the $\pt$ and $|\eta|$ of one of the leptons, regardless of the second,
for muons (left) and electrons (right).
\label{fig:lepScale}}
\end{center}
\end{figure}

\begin{figure}[!htb]
\begin{center}
\includegraphics[width=0.32\linewidth]{Figures/Results/mass/m4lreco_4mu_dn.pdf}
\includegraphics[width=0.32\linewidth]{Figures/Results/mass/m4lreco_4mu.pdf}
\includegraphics[width=0.32\linewidth]{Figures/Results/mass/m4lreco_4mu_up.pdf} \\
\includegraphics[width=0.32\linewidth]{Figures/Results/mass/m4lreco_4e_dn.pdf}
\includegraphics[width=0.32\linewidth]{Figures/Results/mass/m4lreco_4e.pdf}
\includegraphics[width=0.32\linewidth]{Figures/Results/mass/m4lreco_4e_up.pdf} 
\caption{ Different $\mllll$ distributions after propagating the biases in Fig.~\ref{fig:lepScale} to Higgs boson events. The change in the mean of the 
double Crystal-Ball is used to determine the systematic uncertainty due to the lepton momentum scale. The middle plot shows the nominal distribution, while
the left (right) plots show the down (up) systematic variations. The $4\mu$ channel is shown in the top row and the $4e$ channel is shown in the bottom row.
\label{fig:lepScaleM4l}}
\end{center}
\end{figure}

Theoretical uncertainties which affect both the background signal and background estimation 
include uncertainties from the renormalization and factorization scale and choice of PDF set. 
The uncertainty from the renormalization and factorization scale is determined by varying these scales between 
0.5 and 2 times their nominal value while keeping their ratio between 0.5 and 2. 
The uncertainty from the PDF set is determined 
by taking the root mean square of the variation when using different replicas of the default NNPDF set. An additional
uncertainty of 10\% on the $k$-factor used for the $\ggZZ$ prediction is applied as described in Section~\ref{sec:irrbkgd}.
A systematic uncertainty of 2\% on the branching ratio of $\HZZfl$ only affects the $H$ signal yields. 
In the case of event categorization, experimental and theoretical uncertainties that account for
possible migration of signal and background events between categories are included. The main sources 
of uncertainty on the event categorization include the QCD scale, PDF set, and the modeling of hadronization and the underlying 
event. These uncertainties amount to between 4--20\% for the signal and 3--20\% for the background depending on the category.
The lower range corresponds to the VBF and VH processes and the upper range corresponds to the $\ggH$ process yield in the VBF-2jet-tagged category. 
Additional uncertainties come from the imprecise knowledge of the jet energy scale (from 2\% for the $\ggH$ yield in the untagged category to 15\% for  $\ggH$ yield in the VBF-2jet-tagged category) and b-tagging efficiency and mistag 
rate (up to 6\% in the tagged category). The theoretical systematic uncertainties are summarized in Table~\ref{tab:SystOverviewTheo}.
%rate (up to 6\% in the $\ttH$-tagged category). 


\begin{table}[!htb]
\begin{center}
\small
\begin{tabular}{l|c} 
%\hline %---------------------------------------------------------
%\hline %---------------------------------------------------------
%\multicolumn{2}{|c|}{\textbf{Summary of relative systematic uncertainties}} \\
%\hline %---------------------------------------------------------
\hline %---------------------------------------------------------
%\multicolumn{2}{|c|}{Common experimental uncertainties} \\
%\hline %---------------------------------------------------------
%\vspace{-0.4cm} & \\
Source of uncertainty & Value \\
\hline %---------------------------------------------------------
Luminosity & 2.6 \%  \\ 
%\vspace{-0.4cm} & \\
%\hline %---------------------------------------------------------
Lepton identification/reconstruction efficiencies & 2.5 -- 9 \% \\ 
%\vspace{-0.4cm} & \\
%\hline %---------------------------------------------------------
%\hline %---------------------------------------------------------
%\multicolumn{2}{|c|}{Background related uncertainties} \\
%\hline %--------------------------------------------------------
%\vspace{-0.4cm} & \\
Reducible background (Z+X) & 36 -- 43 \% \\ 
% \vspace{-0.4cm} & \\
% Event categorization (experimental) & 2 -- 15 \% \\ 
% Event categorization (theoretical) & 3 -- 20 \% \\ 
%\vspace{-0.4cm} & \\
%\hline %---------------------------------------------------------
%\hline %---------------------------------------------------------
%\multicolumn{2}{|c|}{Signal related uncertainties} \\
%\hline %---------------------------------------------------------
%\vspace{-0.4cm} & \\
Lepton energy scale & 0.04 -- 0.3 \% \\ 
%\hline %---------------------------------------------------------
%\vspace{-0.4cm} & \\
Lepton energy resolution & 20 \% \\ 
% \vspace{-0.4cm} & \\
% Event categorization (experimental) & 2 -- 15 \% \\ 
% Event categorization (theoretical) & 4 -- 20 \% \\ 
%\vspace{-0.4cm} & \\
%\hline %---------------------------------------------------------
\hline %---------------------------------------------------------
\end{tabular}
\caption{
Summary of the experimental systematic uncertainties in the $\Hllll$ measurements. %Details about the derivation of each uncertainty can be found in the text.
\label{tab:SystOverview}
}
\normalsize
\end{center}
\end{table}



\begin{table}[!htb]
\begin{center}
\small
\begin{tabular}{l|c}
\hline %---------------------------------------------------------
%\hline %---------------------------------------------------------
%\multicolumn{2}{|c|}{\textbf{Summary of inclusive theory uncertainties}} \\
%\hline %---------------------------------------------------------
%\hline %---------------------------------------------------------
%\vspace{-0.4cm} & \\
Source of uncertainty & Value \\
\hline
QCD scale (${\rm gg}$) & $\pm$ 3.9 \% \\
%\vspace{-0.4cm} & \\
PDF set (${\rm gg}$) & $\pm$ 3.2 \% \\
%\vspace{-0.4cm} & \\
Bkg K factor (${\rm gg}$) & $\pm$ 10 \% \\
\hline
%\vspace{-0.4cm} & \\
QCD scale (${\rm VBF}$) & +0.4/-0.3 \% \\
%\vspace{-0.4cm} & \\
PDF set (${\rm VBF}$) & $\pm$ 2.1 \% \\
\hline
%\vspace{-0.4cm} & \\
QCD scale (${\rm WH}$) & +0.5/-0.7 \% \\
%\vspace{-0.4cm} & \\
PDF set (${\rm WH}$) & $\pm$ 1.9 \% \\
\hline
%\vspace{-0.4cm} & \\
QCD scale (${\rm ZH}$) & +3.8/-3.1 \% \\
%\vspace{-0.4cm} & \\
PDF set (${\rm ZH}$) & $\pm$ 1.6 \% \\
\hline
%\vspace{-0.4cm} & \\
QCD scale (${\rm \ttH}$) & +5.8/-9.2 \% \\
%\vspace{-0.4cm} & \\
PDF set (${\rm \ttH}$) & $\pm$ 3.6 \% \\
\hline
%\vspace{-0.4cm} & \\
BR($\HZZfl$) & 2 \% \\
%\hline %--------------------------------------------------------
%\vspace{-0.4cm} & \\
\hline
QCD scale ($\qqZZ$) & +3.2/-4.2 \% \% \\
%\vspace{-0.4cm} & \\
PDF set ($\qqZZ$) & +3.1/-3.4 \% \\
%\vspace{-0.4cm} & \\
Electroweak corrections ($\qqZZ$) & $\pm$ 0.1 \% \\
\hline %---------------------------------------------------------
%\hline %---------------------------------------------------------
\end{tabular}
\caption{Summary of the theory systematic uncertainties in the $\Hllll$ measurements for the inclusive analysis.
\label{tab:SystOverviewTheo}
}
\normalsize
\end{center}
\end{table}
 
\section{Limit setting}

The primary tool used to interpret the analysis described above in the context of the signal models is the HiggsAnalysis-CombinedLimit package \cite{combinetwiki}, a collection of RooStats-based software \cite{roostatstwiki} used within the Higgs physics analysis group (PAG) \cite{higgspagtwiki}, hereafter called the combine tool, or simply "combine". Specifically, one-sided Bayesian credible interval limits are set on the expected and observed signal production cross section times branching ratio (BR) of $\HZZfl$ ($\sigma\times \rm{BR}$). Limits are set on the various signal benchmarks using the asymptotic $\rm{CL}_S$ method \cite{Cowan:2010js}, an approach to calculating a profile likelihood ratio using an approximation of the LHC test-statistic distributions. The upper limit on the cross section gives the maximum number of events that can be attributed to the signal process, consistent with the data that is observed. 

Combine finds the upper limit as the numerical solution to Equation~\ref{eq:pCL}, which sets the integral of the posterior probability $p(\sigma|D)d\sigma$ equal to the desired confidence level for the measurement, typically 95\%. The posterior probability gives the degree of belief that $\sigma$ lies in the interval $[\sigma,\ \sigma+d\sigma]$ and is formed by inverting a multi-Poisson model, $p(D|\sigma,\ \bm{\theta})$ using Bayes' theorem (Equation~\ref{eq:pbayes}), after numerically marginalizing priors describing the uncertainties $\pi(\bm{\theta})$ (Equation~\ref{eq:pmarg}).

\begin{equation}\label{eq:pCL}
\int_0^{\sigma_{\rm{Up}}} p(\sigma|D)d\sigma = 1 - \alpha
\end{equation}
where,
\begin{equation}\label{eq:pbayes}
p(\sigma|D) = p(D|\sigma)\pi(\sigma) / p(D)
\end{equation}
and,
\begin{equation}\label{eq:pmarg}
p(D|\sigma) = \int p(D|\sigma,\ \bm{\theta}) \pi(\bm{\theta}) d\bm{\theta}.
\end{equation}

Combine takes specially formatted text files called data cards as input. The data cards contain the signal, background, and data yields, along with associated systematic uncertainties, and the location of the file containing the shape distributions for each sample. When the shape distributions are included, combine computes the limits accross all bins, then combines the results. A non-shape-based limit is equivalent to a shape-based limit with one bin. This can be used as a crosscheck that the shape-based limit is behaving correctly and to analyze the improvement of using the shape-based approach over non-shape-based.

Combine returns the 95\% confidence level (CL) expected and observed limits on the signal strength parameter $\mu$, as well as one and two standard deviations of the expected limit, where $\mu$ is the signal scale factor in the $\rm{i}^{\rm{th}}$ bin in the mean count, $n_i = \mu * s_i + b_i$, where $s_i$ and $b_i$ are the signal and background yields, respectively.
%Limits are set for the three decay channels, 4e, 4$\mu$, and 2e2$\mu$ individually, then combined for the 4l limit. 
The signal strength parameter gives the ratio of the 95\% CL expected signal yield to the theoretical yield, or more generally:

\begin{equation}\label{eq:mu}
\mu[\sigma_{\rm{norm}}*\rm{BR}] = \frac{\sigma_{95\% \rm{CL}}*\rm{BR}}{\sigma_{\rm{norm}}*\rm{BR}},
\end{equation}
where $\sigma_{\rm{norm}}$ is the cross section used to normalize the signal yields given as input to combine, typically the theoretical production cross section, or set to 1 pb for comparison to other $H$ decay channels. This formula is used to calculate $\sigma_{95\% \rm{CL}}\times \rm{BR}$, which is the desired output variable. This limit can then be compared to the theoretical $\sigma\times \rm{BR}$. The points where the upper limit is lower than the theory value are interpreted as exclusions.


\subsection{Cross section limits}

After the event selection is optimized and all of the uncertainties are accounted for, limits are set on the signal strength parameter $\mu$, which is scaled according to Equation~\ref{eq:mu} to obtain the cross section limit. For each model, these results are found for both the cut-and-count and MVA-based strategies. For Zp2HDM, the one-dimensional slice of mass points fixing $m_{A^0} = 300$ $\GeV$ is selected for the limit plots since this region has the highest cross section and the largest branching fraction to DM production. For ZpBaryonic, the one-dimensional slice of mass points fixing $m_{\chi} = 1$ $\GeV$ is selected for the limit plots since this region has the highest cross section.

\subsubsection{Cut-and-count based limits}

The one-dimensional limits, obtained using the optimized selection given in Section~\ref{cutandcountopt}, are presented for $\sigma_{95\% \rm{CL}}\times \rm{BR}$ in Figure~\ref{fig:limzp2hdm} for Zp2HDM and Figure~\ref{fig:limzpbaryonic} for ZpBaryonic. In both models, the observed limits do not deviate more than one standard deviation from the expected limits, indicating the absence of a large excess of events in the SR above the SM prediction. Since the limit curves lie above the theoretical cross section curves, no benchmark points can be excluded using this analysis alone. However, when the Zp2HDM results are combined with those from the other $H$ decay channels, there is sufficient sensitivity to exclude a large portion of the parameter space, up to mediator masses in the $\TeV$ range. These are the first results obtained for ZpBaryonic from any of the mono-$H$ analyses. Since the ZpBaryonic theoretical cross section curve is flat up to about 1 $\TeV$, any small increase in sensitivity by combining with other analyses will allow for the exclusion of these benchmark models. 

The limiting factor in improving the sensitivity of the limits obtained with this strategy is the signal efficiency. The cross section limit is directly proportional to the signal selection efficiency, and each additional cut reduces the signal efficiency by a few percent or more. From Table~\ref{tab:smyields}, the largest fraction of signal events is lost during the tight lepton and $Z$ selection steps, even before the mono-$H$ selection is applied. One method of improving the sensitivity of this analysis is to loosen the tight lepton selection criteria, but allowing more signal to pass a looser cut would also let more background pass. However, since MET is such a powerful discriminating variable, the additional backgrounds could potentially be reduced by a harder MET cut. Defining a new working point for the selection of tight leptons would require remeasuring lepton systematics and background estimates from data and the validation of modeling in additional CRs. This analysis is left to a future study.

\begin{figure}[tbh]
\centering
\includegraphics[width=6in]{figures/sigma_limits_4mu_Zp2HDM.png}
\caption{One-dimensional cross section times branching fraction limits for the Zp2HDM simplified model using the cut-and-count based event selection strategy.}
\label{fig:limzp2hdm}
\end{figure}

\begin{figure}[tbh]
\centering
\includegraphics[width=6in]{figures/sigma_limits_4mu_ZpBaryonic.png}
\caption{One-dimensional cross section times branching fraction limits for the ZpBaryonic simplified model using the cut-and-count based event selection strategy.}
\label{fig:limzpbaryonic}
\end{figure}

\subsubsection{MVA-based limits}

The one-dimensional limits, obtained using the optimized selection given in Section~\ref{mvaopt}, are presented for $\sigma_{95\% \rm{CL}}\times \rm{BR}$ in Figure~\ref{fig:limzp2hdmmva} for Zp2HDM and Figure~\ref{fig:limzpbaryonicmva} for ZpBaryonic. The sensitivity obtained using the MVA approach is not as good as for the cut-and-count approach. This reiterates the strong discriminating power of MET used in the cut-and-count strategy, rather than indicating a poor performance of the MVA strategy. Again, the signal efficiency is the limiting factor in improving the sensitivity. Since there are no additional cuts beyond the SM selection in this selection strategy, the SM selection itself would need to be modified, requiring additional study. 

\begin{figure}[tbh]
\centering
\includegraphics[width=6in]{figures/sigma_limits_4mu_Zp2HDM_MVA.png}
\caption{One-dimensional cross section times branching fraction limits for the Zp2HDM simplified model using the MVA-based event selection strategy.}
\label{fig:limzp2hdmmva}
\end{figure}

\begin{figure}[tbh]
\centering
\includegraphics[width=6in]{figures/sigma_limits_4mu_ZpBaryonic_MVA.png}
\caption{One-dimensional cross section times branching fraction limits for the ZpBaryonic simplified model using the MVA-based event selection strategy.}
\label{fig:limzpbaryonicmva}
\end{figure}


\chapter{Conclusions}

The discovery of the Higgs boson was the highlight of the physics results of Run 1 at the LHC. In addition to measuring the properties and couplings of the Higgs, efforts were launched in Run 2 to use the Higgs as a probe for new physics, including to search for dark matter. In parallel with these studies, progress was made by complementary, non-collider searches. Although no direct detection has been confirmed using any approach, large areas of parameter space have been excluded. Collider searches, including the search for the mono-Higgs signature, have much better sensitivity to low-mass dark matter than direct searches, as well as the ability to study higher order couplings that are not accessible to direct searches. The main results of this dissertation are (1) the rediscovery of the Higgs boson with Run 2 data and (2) the cross section limits on two dark matter models, contributing to the world-leading low-mass limits and mediator mass exclusions for these models.

The impact of this study is very high, being among the first-ever searches for the mono-Higgs signature. Several key contributions were made to the standard model Higgs search in the four-lepton final state, including (1) a cross-check exercise to validate the event selection with other groups and (2) the definition of a new event category for Higgs candidate events with large missing transverse momentum. The key ingredient that this study added to the Standard Model Higgs search and collaboration documentation is the study of missing energy, in particular, whether the background and statistical modeling techniques remain valid in the high missing energy regime. 

Numerous extensions of this analysis would have a large impact on the field. The selection used to identify events with a Higgs boson can be reoptimized for mono-Higgs signals rather than for the Standard Model Higgs signals. This could significantly increase the limit-setting sensitivity. The limits found with the four-lepton final state can be combined with those from the other Higgs decay channel analyses to obtain even stronger results. This work is underway. There are additional models that predict a mono-Higgs signature that can be studied with the datasets currently available. The most commonly studied models are those that predict very hard missing energy spectra, which gives an advantage to other Higgs decay channels. However, the four-lepton channel would have better sensitivity for the unstudied models with softer spectra. Finally, a more detailed study can be done to understand the relationship of the limits found here with those set by other search strategies in order to guide the design of future analyses and experiments. 

There is strong motivation for the existence of a coupling between the dark sector and ordinary matter. Studies such as this one shed light on the nature of this interaction. Technology and analysis techniques are constantly advancing. If such an interaction exists, it is only a matter of time before dark matter is observed, expanding our understanding of the mass-energy content of the universe beyond the mere 5\% that we currently know.


\begin{appendices}
\chapter{Production cross sections for benchmark signal models}

\begin{sidewaystable}[htbH]
\begin{adjustbox}{width=\textwidth,totalheight=\textheight,keepaspectratio}
\label{tab:eftxsecs}
\begin{tabular}{l | c | c | c | c | c | c | c | c | c | c}
\hline 
$m_\chi$ [GeV] & 1 & 10 & 50 & 65 & 100 & 200 & 400 & 800 & 1000 & 1300 \\
\hline
EFT$\_$HHxx$\_$scalar & & & & & & & & & & \\
 & 0.10071E+01 & 0.99793E+00 & 0.60671E+00 & 0.48291E-04 & 0.22725E-05 & 0.11059E-06 & 0.36569E-08 & 0.40762E-10 & 0.64956E-11 & 0.51740E-12 \\
EFT$\_$HHxx$\_$combined & & & & & & & & & & \\
 & 0.15731E+01 & 0.15194E+01 & 0.34134E+00 & 0.41039E-04 & 0.10581E-04 & 0.16553E-05 & 0.14628E-06 & 0.40608E-08 & 0.85950E-09 & 0.96480E-10 \\
EFT$\_$HHxg5x & & & & & & & & & & \\
 & 0.15735E+03 & 0.15594E+03 & 0.94804E+02 & 0.12990E-01 & 0.23075E-02 & 0.41820E-03 & 0.45743E-04 & 0.16734E-05 & 0.39327E-06 & 0.49769E-07 \\
EFT$\_$xdxHDHc & & & & & & & & & & \\
$\quad \Lambda = 100$ GeV & 0.29530E+00 & 0.29067E+00 & 0.10540E+00 & 0.89849E-01 & 0.64959E-01 & 0.30639E-01 & 0.88644E-02 & 0.97986E-03 & 0.33847E-03 & 0.68674E-04 \\
$\quad \Lambda = 1000$ GeV & 0.16306E-04 & 0.15508E-04 & 0.12088E-05 & 0.88288E-06 & 0.53312E-06 & 0.18046E-06 & 0.34918E-07 & 0.27514E-08 & 0.90662E-09 & 0.19313E-09 \\
EFT$\_$xgxFHDH & & & & & & & & & & \\
 & 0.57027E+00 & 0.57001E+00 & 0.56025E+00 & 0.55337E+00 & 0.53270E+00 & 0.45792E+00 & 0.29777E+00 & 0.10288E+00 & 0.57444E-01 & 0.23260E-01 \\
\hline
\end{tabular}
\end{adjustbox}
\caption{Effective field theory model production cross sections [pb]}
\end{sidewaystable}

\begin{sidewaystable*}[htbH]
\begin{adjustbox}{width=\textwidth,totalheight=\textheight,keepaspectratio}
\label{tab:scalarxsecs}
\begin{tabular}{l | c | c | c | c | c | c | c | c | c}
\hline 
$m_\chi$ [GeV] & \multicolumn{9}{c}{$m_{S}$ [GeV]} \\ 
\hline
1 & 0.21915E+01 & 0.20798E+01 & 0.19192E+01 & 0.18118E+01 & 0.16735E+01 & 0.52244E+01 & 0.41877E+01 & 0.28732E+01 & 0.18028E+01\\
10 & 0.17416E+01 & 0.17420E+01 & 0.18581E+01 & 0.17510E+01 & & & & & 0.17398E+01\\
50 & 0.39053E+00 & & 0.38877E+00 & 0.38409E+00 & 0.37097E+00 & 0.12861E+01 & & & 0.39096E+00\\
150 & 0.24136E-05 & & & & 0.38372E-05 & 0.21922E-04 & 0.42337E-03 & 0.57124E-04 & 0.11105E-04\\
500 & 0.34099E-08 & & & & & & 0.49399E-08 & 0.25206E-06 & 0.36823E-06\\
1000 & 0.17012E-10 & & & & & & & 0.55260E-10 & 0.11067E-07\\
\hline
\end{tabular}
\end{adjustbox}
\caption{Scalar simplified model production cross sections [pb] corresponding to mass points in Table~\ref{tab:MMScalar}}
\end{sidewaystable*}

\begin{sidewaystable*}[htbH]
\begin{adjustbox}{width=\textwidth,totalheight=\textheight,keepaspectratio}
\label{tab:zpbaryonicxsecs}
\begin{tabular}{l | c | c | c | c | c | c | c | c | c | c}
\hline 
$m_\chi$ [GeV] & \multicolumn{10}{c}{$m_{Z'}$ [GeV]} \\ 
\hline
1 & 0.26615E+01 & 0.27802E+01 & 0.33248E+01 & 0.32341E+01 & 0.26566E+01 & 0.23191E+01 & 0.10842E+01 & 0.18700E+00 & 0.11728E-01 & 0.17399E-07\\
10 & 0.21182E-01 & 0.74027E-01 & 0.32732E+01 & 0.32250E+01 & & & & & & 0.17380E-07\\
50 & 0.36342E-03 & & 0.12726E-01 & 0.31337E+00  & 0.21226E+01 & 0.20120E+01 & & & & 0.17340E-07\\
150 & 0.55972E-05 & & & & 0.56526E-02 & 0.18000E+00 & 0.67266E+00 & 0.18111E+00 & & 0.16918E-07\\
500 & 0.80295E-08 & & & & & & 0.36591E-04 & 0.10368E-01 & 0.10375E-01 & 0.13179E-07\\
1000 & 0.49387E-10 & & & & & & & 0.98079E-06 & 0.57596E-03 & 0.80146E-08\\
\hline
\end{tabular}
\end{adjustbox}
\caption{ZpBaryonic simplified model production cross sections [pb] corresponding to mass points in Table~\ref{tab:MMVector}}
\end{sidewaystable*}

\begin{sidewaystable*}[htbH]
\begin{adjustbox}{width=\textwidth,totalheight=\textheight,keepaspectratio}
\label{tab:zphsxsecs}
\begin{tabular}{l | c | c | c | c | c | c | c | c | c | c}
\hline 
$m_\chi$ [GeV] & \multicolumn{10}{c}{$m_{Z'}$ [GeV]} \\ 
\hline
1 & 0.61935E-02 & 0.63192E-02 & 0.82991E-02 & 0.11942E-01 & 0.19171E-01 & 0.21560E-01 & 0.16010E-01 & 0.64416E-02 & 0.56526E-02 & 0.58902E-02 \\
10 & 0.58781E-02 & 0.58938E-02 & 0.82944E-02 & 0.11937E-01 & & & & & & 0.58805E-02\\
50 & 0.10294E-03 & & 0.77820E-04 & 0.15258E-04  & 0.12066E-01 & 0.12105E-01 & & & & 0.10387E-03 \\
150 & 0.28382E-06 & & & & 0.83917E-05 & 0.72889E-03 & 0.65401E-02 & 0.68337E-03 & & 0.29033E-06 \\
500 & 0.34689E-09 & & & & & & 0.43355E-06 & 0.87799E-04 & 0.28292E-05 & 0.36327E-09 \\
1000 & 0.20703E-11 & & & & & & & 0.44782E-07 & 0.19974E-06 & 0.27001E-11 \\
\hline
\end{tabular}
\end{adjustbox}
\caption{ZpHS simplifed model production cross sections [pb] corresponding to mass points in Table~\ref{tab:MMVector}}
\end{sidewaystable*}

\begin{sidewaystable*}[htbH]
%\begin{table*}[htbH]
\begin{center}
\begin{tabular}{l | c | c | c | c | c | c | c | c }
\hline 
$m_{A^0}$ [GeV] & \multicolumn{8}{c}{$m_{Z'}$ [GeV]} \\ 
\hline
300 & 42.386 & 45.097 & 35.444 & 26.07 & 18.942 & 11.778 & 7.4456 & 3.6446 \\  
400 & 5.8513 & 14.847 & 14.534 & 11.792 & 9.029 & 5.851 & 3.7819 & 1.8758 \\  
500 & & 5.9605 & 8.4961 & 7.9575 & 6.5515 & 4.5063 & 3.0028 & 1.5235 \\  
600 & & 1.5853 & 4.6972 & 5.4808 & 4.9946 & 3.7044 & 2.5694 & 1.3447 \\
700 & & & 2.1092 & 3.4848 & 3.6766 & 3.0253 & 2.2023 & 1.1984 \\  
800 & & & 0.65378 & 1.9638 & 2.5511 & 2.4077 & 1.8689 & 1.0692 \\  
\hline
\end{tabular}
\caption{Zp2HDM simplified model production cross sections [pb] corresponding to mass points in Table~\ref{tab:MM2HDM} \label{tab:zp2hdmxsecs}}
\end{center}
%\end{table*}
\end{sidewaystable*}

\end{appendices}

%\makeintropages %Processes/produces the preliminary pages

\bibliography{ucdavisthesis}

% The UMI abstract uses square brackets!
%\UMIabstract[
%The study presented in this dissertation is a search for dark matter produced in 13 TeV proton-proton collisions at the Large Hadron Collider (LHC) using $\usedLumi$ of data collected in 2016 with the Compact Muon Solenoid (CMS) detector. Dark matter escapes the detector without interacting, resulting in a large imbalance of transverse momentum, which can be observed when a Higgs boson is tagged in the opposite direction. A variety of models which motivate a dark matter and Higgs interaction are discussed. The experimental signature of these models is called mono-Higgs. 
%
%In this search, the Higgs is produced primarily from gluon fusion and decays to four leptons via two $Z$ bosons ($\HZZfl$). In addition to observing the Higgs in the four-lepton final state, an extensive study of missing transverse energy (MET) is required to search for the mono-Higgs signature. 
%A background model is developed for the Standard Model processes that result in the same final state as the signal, then a counting experiment is performed in an optimized signal region. There is no evidence for an excess of events in the signal region above the backgrounds, so cross section limits are set for two kinematically distinct signal models.
%The study presented in this dissertation is a search for dark matter produced in 13 TeV proton-proton collisions at the Large Hadron Collider (LHC) using $\usedLumi$ of data collected in 2016 with the Compact Muon Solenoid (CMS) detector. Dark matter escapes the detector without interacting, resulting in a large imbalance of transverse momentum, which can be observed when a Higgs boson is tagged in the opposite direction. A variety of models which motivate a dark matter and Higgs interaction are discussed. The experimental signature of these models is called mono-Higgs. 
%
%In this search, the Higgs is produced primarily from gluon fusion and decays to four leptons via two $Z$ bosons ($\HZZfl$). In addition to rediscovering the Higgs in the four-lepton final state, cross-validating the analysis strategy with other teams, an extensive study of missing transverse energy (MET) is required to search for the mono-Higgs signature. 
%
%Once a background model is developed for the Standard Model processes that can mimic the signal, a counting experiment is performed in an optimized signal region. There is no evidence for an excess of events in the signal region above the backgrounds, so cross section limits are set for two kinematically distinct signal models.
]
%\UMIabstract[The abstract that is submitted to UMI must be formatted as shown in the example here. The body of the abstract cannot exceed 350 words. It should be in typewritten form, double-spaced, and on bond paper. It is important to write an abstract that gives a clear description of the content and major divisions of the dissertation, since UMI will publish the abstract exactly as submitted. Students completing their requirements under Plan A should provide extra copies of the typed summary for use by the dissertation committee during the examination.
%
%The abstract that is submitted to UMI must be formatted as shown in the example here. The body of the abstract cannot exceed 350 words. It should be in typewritten form, double-spaced, and on bond paper. It is important to write an abstract that gives a clear description of the content and major divisions of the dissertation, since UMI will publish the abstract exactly as submitted. Students completing their requirements under Plan A should provide extra copies of the typed summary for use by the dissertation committee during the examination.]

\end{document} 
