\chapter{Statistical analysis results}

\section{Systematic uncertainties}

This section covers the systematic, including statistical, errors associated with the analyis. The general strategy is to duplicate the systematics applied in the SM $H\rightarrow ZZ$ search, including errors on the background estimation from data, while adding the errors associated with MET modeling, guided by the mono-$H \rightarrow \gamma\gamma$ strategy \cite{CMS-AN-15-203}. 

% Uncertainties
\subsection{Uncertainties on the reducible background estimation} %using Opposite-Sign leptons}
\label{sec:zxUncert}


The main source of systematic uncertainty on the background estimation method is due to the different compositions of the reducible background processes ($DY$, $t \bar{t}$, $WZ$, $Z\gamma^{(*)}$) in the regions where the fake rates are measured and where they are applied.
The OS method corrects for the resulting bias with the 3P1F component of its prediction.
The closure tests presented here are used to assess a possible residual bias in the OS method.

The systematic uncertainty due to different compositions of events can be estimated by measuring the fake rates for individual background processes in the $Z+1L$ region in simulation. The weighted average of these individual fake rates is used for the overall fake rate. The exact composition of the background processes in the 2P+2F region where the fake rates are applied can be determined from simulation. The individual fake rates can then be reweighted according to the 2P+2F composition. The difference between the reweighed fake rate and the average one can be used as a measure of the uncertainty on the measurement of the fake rates. The effect of this systematic uncertainty is propagated to the final estimates, and it amounts to approximately 32\% for $4e$, 33\% for $2e2\mu$, and 35\% for the $4\mu$ final state. 

% \begin{table}[h]
% \scriptsize
%     \centering
%     \caption{
%     The fake ratios for individual background processes, the average fake ratio and the fake ratio reweighed according to the composition of backgrounds in 2P+2F region.}  
    
%     \begin{tabular}[!htb]{| l | c | c | c | c | c | c |} \hline
% Sample				& Light Jets		  & HF Jets& From $\gamma$ conv	    & average (in $Z+1\ell$) & reweighed (2P2F)	& uncertainty \\ \hline \hline
% electron fake ratio	& $0.012^{+0.001}_{-0.001}$ 	& $0.115^{+0.005}_{-0.006}$  & $0.293^{+0.043}_{-0.043}$  &$0.021^{+0.001}_{-0.001}$& $0.024^{+0.004}_{-0.004}$& 15\%    \\ 
% muon fake ratio 	& $0.057^{+0.003}_{-0.003}$ 	& $0.225^{+0.008}_{-0.008}$ & $0.003^{+0.455}_{-0.003}$ &$0.120^{+0.004}_{-0.004}$& $0.105^{+0.020}_{-0.017}$ & 13\% \\  \hline
% 	\end{tabular}
%     \label{tab:uncertFR}
% \end{table}

Additional uncertainties arise from the limited size of the samples in the four-lepton control regions, where the fake rates are measured and applied contributes a statistical uncertainty. The dominate statistical uncertainty is driven by the number of events in the control region and is typically in the range of 1--10\%.

In order to estimate the uncertainty on the $m_{4l}$ shape, the differences among the shapes of predicted background distributions for all three channels are studied. The envelope of differences among these distribution shapes is used as an estimate of the shape uncertainty. The uncertainty is estimated to be roughly in the range of 5--15\%. Since the difference of the shapes slowly varies with $m_{4l}$, it is taken as a constant versus $m_{4l}$ and is absorbed in the much larger uncertainty on the predicted yield of backgrounds. 
%The shapes of predicted background $m_{4l}$ distributions for the three channels are shown in Figure~\ref{fig:SR_CombinedShapes} (left).	


\subsection{MET systematics}\label{sec:metsyst}

There are two types of systematic uncertainties related to the modeling of MET: (1) those from the measurement of real MET, as from the signal samples or backgrounds with neutrinos, and (2) those from fake MET, due to the mismeasurement of jets and other objects. The fake MET systematics apply to the $H$ signals with no associated $W$ production and to the non-resonant backgrounds. 

The uncertainties from the mismodeling of real MET are measuremed by varying several corrections used to calculate MET, then propogating these variations to the efficiency of MC samples to pass the MET selection \cite{mettwiki}. The corrections used in this calculation are:
jet energy,
jet resolution,
muon energy,
electron energy,
tau energy,
photon energy, and
unclustered jet energy. Each correction is varied up and down by one standard deviation of the input distribution, with the systematic uncertainty taken as the maximim difference in efficiencies accross all correction variations. The efficiencies for the V+H samples to pass the MET selection vary by a few percent and the variation in signal sample efficiencies is less than one percent.
%The efficiencies for the V+H and a benchmark signal sample to pass the MET selection of $E_{\rm{T}}^{\rm{MISS}}>60$ GeV after each correcion is varied up and down are given in Table~\ref{tab:} and Table~\ref{tab:}, respectively. 

%\begin{table*}[htbH]
%\begin{center}
%\topcaption{Efficiencies for V+H sample to pass MET selection after varying corrections up and down.}\label{tab:yields}
%\begin{tabular}{ l | c | c }
%\hline
%\hline
%Correction & Efficiency Up & Efficiency Down \\
%\hline
%Original PFMET & & \\
%\hline
%Jet energy & & \\
%\hline
%Jet resolution & & \\
%\hline
%Muon energy & & \\
%\hline
%Electron energy & & \\
%\hline
%Tau energy & & \\
%\hline
%Photon energy & & \\
%\hline
%Unclustered jet energy & & \\
%\hline
%\hline
%\end{tabular}
%\end{center}
%\end{table*}


%\begin{table*}[htbH]
%\begin{center}
%\topcaption{Efficiencies for a benchmark signal sample, Zp2HDM(600), to pass MET selection after varying corrections up and down.}\label{tab:yields}
%\begin{tabular}{ l | c | c }
%\hline
%\hline
%Correction & Efficiency Up & Efficiency Down \\
%\hline
%Original PFMET & & \\
%\hline
%Jet energy & & \\
%\hline
%Jet resolution & & \\
%\hline
%Muon energy & & \\
%\hline
%Electron energy & & \\
%\hline
%Tau energy & & \\
%\hline
%Photon energy & & \\
%\hline
%Unclustered jet energy & & \\
%\hline
%\hline
%\end{tabular}
%\end{center}
%\end{table*}

The second systematic uncertainty results from the mismodeling of fake MET, primarily due to the mismeasurement of jets (see Section~\ref{sec:zxIntr}). This uncertainty is measured in the sideband CR as the percent difference between the efficiency for the data and total background sample to pass the MET selection. These efficiencies differ by 42\%, which is taken as the systematic on backgrounds without real MET.

\subsection{Additional systematics}

Both signal and background samples are affected by several additional systematic uncertainties, including the uncertainty on the integrated luminosity
(2.6\%) and the uncertainty on the lepton identification and reconstruction efficiency (ranging from 2.5--9\%). Experimental uncertainties for the reducible background estimation, 
described in Section~\ref{sec:redbkgd},
vary between 36\% ($4\mu$)  and 43\% ($4e$).  The uncertainty on the lepton energy scale is determined by considering the 
$Z\rightarrow\ell\ell$ mass distributions in data and simulation. Events are separated into categories based on the 
$\pt$ and $\eta$ of one of the two leptons, determined randomly, and integrating over the other. The dilepton mass 
distributions are then fit to a Breit-Wigner 
parameterization convolved with a double-sided Crystal-Ball function. The offset in the measured peak position with 
respect to the nominal $\cPZ$ boson 
mass in data and simulation are extracted. The results are shown in Figure~\ref{fig:lepScale}. The relative difference 
between data and simulation is propagated to the reconstructed four-lepton mass 
from simulated $H$ events. The results of the propagation can be seen in Figure~\ref{fig:lepScaleM4l}.
In the case of electrons, since the same dataset is used to derive and validate the 
momentum scale corrections, the size 
of the corrections are taken into account for the final value of the uncertainty.
The uncertainty is determined to be 0.04\% (0.3\%) for the  $4\mu$ ($4\Pe$) channel. The uncertainty 
on the $4\ell$ mass resolution coming from the uncertainty on the per-lepton energy resolution is 20\%. The experimental systematic uncertainties are summarized in Table~\ref{tab:SystOverview}.


\begin{figure}[!htb]
\begin{center}
\includegraphics[width=0.48\linewidth]{Figures/Results/mass/lepScale_vs_pt_eta_mu.pdf}
\includegraphics[width=0.48\linewidth]{Figures/Results/mass/lepScale_vs_pt_eta_e.pdf}
\caption{ Difference between the ${\rm Z}\rightarrow\ell\ell$ mass peak positions in data and simulation normalized by the 
nominal $\cPZ$ boson mass obtained as a function of the $\pt$ and $|\eta|$ of one of the leptons, regardless of the second,
for muons (left) and electrons (right).
\label{fig:lepScale}}
\end{center}
\end{figure}

\begin{figure}[!htb]
\begin{center}
\includegraphics[width=0.32\linewidth]{Figures/Results/mass/m4lreco_4mu_dn.pdf}
\includegraphics[width=0.32\linewidth]{Figures/Results/mass/m4lreco_4mu.pdf}
\includegraphics[width=0.32\linewidth]{Figures/Results/mass/m4lreco_4mu_up.pdf} \\
\includegraphics[width=0.32\linewidth]{Figures/Results/mass/m4lreco_4e_dn.pdf}
\includegraphics[width=0.32\linewidth]{Figures/Results/mass/m4lreco_4e.pdf}
\includegraphics[width=0.32\linewidth]{Figures/Results/mass/m4lreco_4e_up.pdf} 
\caption{ Different $\mllll$ distributions after propagating the biases in Fig.~\ref{fig:lepScale} to Higgs boson events. The change in the mean of the 
double Crystal-Ball is used to determine the systematic uncertainty due to the lepton momentum scale. The middle plot shows the nominal distribution, while
the left (right) plots show the down (up) systematic variations. The $4\mu$ channel is shown in the top row and the $4e$ channel is shown in the bottom row.
\label{fig:lepScaleM4l}}
\end{center}
\end{figure}

Theoretical uncertainties which affect both the background signal and background estimation 
include uncertainties from the renormalization and factorization scale and choice of PDF set. 
The uncertainty from the renormalization and factorization scale is determined by varying these scales between 
0.5 and 2 times their nominal value while keeping their ratio between 0.5 and 2. 
The uncertainty from the PDF set is determined 
by taking the root mean square of the variation when using different replicas of the default NNPDF set. An additional
uncertainty of 10\% on the $k$-factor used for the $\ggZZ$ prediction is applied as described in Section~\ref{sec:irrbkgd}.
A systematic uncertainty of 2\% on the branching ratio of $\HZZfl$ only affects the $H$ signal yields. 
In the case of event categorization, experimental and theoretical uncertainties that account for
possible migration of signal and background events between categories are included. The main sources 
of uncertainty on the event categorization include the QCD scale, PDF set, and the modeling of hadronization and the underlying 
event. These uncertainties amount to between 4--20\% for the signal and 3--20\% for the background depending on the category.
The lower range corresponds to the VBF and VH processes and the upper range corresponds to the $\ggH$ process yield in the VBF-2jet-tagged category. 
Additional uncertainties come from the imprecise knowledge of the jet energy scale (from 2\% for the $\ggH$ yield in the untagged category to 15\% for  $\ggH$ yield in the VBF-2jet-tagged category) and b-tagging efficiency and mistag 
rate (up to 6\% in the tagged category). The theoretical systematic uncertainties are summarized in Table~\ref{tab:SystOverviewTheo}.
%rate (up to 6\% in the $\ttH$-tagged category). 


\begin{table}[!htb]
\begin{center}
\small
\begin{tabular}{l|c} 
%\hline %---------------------------------------------------------
%\hline %---------------------------------------------------------
%\multicolumn{2}{|c|}{\textbf{Summary of relative systematic uncertainties}} \\
%\hline %---------------------------------------------------------
\hline %---------------------------------------------------------
%\multicolumn{2}{|c|}{Common experimental uncertainties} \\
%\hline %---------------------------------------------------------
%\vspace{-0.4cm} & \\
Source of uncertainty & Value \\
\hline %---------------------------------------------------------
Luminosity & 2.6 \%  \\ 
%\vspace{-0.4cm} & \\
%\hline %---------------------------------------------------------
Lepton identification/reconstruction efficiencies & 2.5 -- 9 \% \\ 
%\vspace{-0.4cm} & \\
%\hline %---------------------------------------------------------
%\hline %---------------------------------------------------------
%\multicolumn{2}{|c|}{Background related uncertainties} \\
%\hline %--------------------------------------------------------
%\vspace{-0.4cm} & \\
Reducible background (Z+X) & 36 -- 43 \% \\ 
% \vspace{-0.4cm} & \\
% Event categorization (experimental) & 2 -- 15 \% \\ 
% Event categorization (theoretical) & 3 -- 20 \% \\ 
%\vspace{-0.4cm} & \\
%\hline %---------------------------------------------------------
%\hline %---------------------------------------------------------
%\multicolumn{2}{|c|}{Signal related uncertainties} \\
%\hline %---------------------------------------------------------
%\vspace{-0.4cm} & \\
Lepton energy scale & 0.04 -- 0.3 \% \\ 
%\hline %---------------------------------------------------------
%\vspace{-0.4cm} & \\
Lepton energy resolution & 20 \% \\ 
% \vspace{-0.4cm} & \\
% Event categorization (experimental) & 2 -- 15 \% \\ 
% Event categorization (theoretical) & 4 -- 20 \% \\ 
%\vspace{-0.4cm} & \\
%\hline %---------------------------------------------------------
\hline %---------------------------------------------------------
\end{tabular}
\caption{
Summary of the experimental systematic uncertainties in the $\Hllll$ measurements. %Details about the derivation of each uncertainty can be found in the text.
\label{tab:SystOverview}
}
\normalsize
\end{center}
\end{table}



\begin{table}[!htb]
\begin{center}
\small
\begin{tabular}{l|c}
\hline %---------------------------------------------------------
%\hline %---------------------------------------------------------
%\multicolumn{2}{|c|}{\textbf{Summary of inclusive theory uncertainties}} \\
%\hline %---------------------------------------------------------
%\hline %---------------------------------------------------------
%\vspace{-0.4cm} & \\
Source of uncertainty & Value \\
\hline
QCD scale (${\rm gg}$) & $\pm$ 3.9 \% \\
%\vspace{-0.4cm} & \\
PDF set (${\rm gg}$) & $\pm$ 3.2 \% \\
%\vspace{-0.4cm} & \\
Bkg K factor (${\rm gg}$) & $\pm$ 10 \% \\
\hline
%\vspace{-0.4cm} & \\
QCD scale (${\rm VBF}$) & +0.4/-0.3 \% \\
%\vspace{-0.4cm} & \\
PDF set (${\rm VBF}$) & $\pm$ 2.1 \% \\
\hline
%\vspace{-0.4cm} & \\
QCD scale (${\rm WH}$) & +0.5/-0.7 \% \\
%\vspace{-0.4cm} & \\
PDF set (${\rm WH}$) & $\pm$ 1.9 \% \\
\hline
%\vspace{-0.4cm} & \\
QCD scale (${\rm ZH}$) & +3.8/-3.1 \% \\
%\vspace{-0.4cm} & \\
PDF set (${\rm ZH}$) & $\pm$ 1.6 \% \\
\hline
%\vspace{-0.4cm} & \\
QCD scale (${\rm \ttH}$) & +5.8/-9.2 \% \\
%\vspace{-0.4cm} & \\
PDF set (${\rm \ttH}$) & $\pm$ 3.6 \% \\
\hline
%\vspace{-0.4cm} & \\
BR($\HZZfl$) & 2 \% \\
%\hline %--------------------------------------------------------
%\vspace{-0.4cm} & \\
\hline
QCD scale ($\qqZZ$) & +3.2/-4.2 \% \% \\
%\vspace{-0.4cm} & \\
PDF set ($\qqZZ$) & +3.1/-3.4 \% \\
%\vspace{-0.4cm} & \\
Electroweak corrections ($\qqZZ$) & $\pm$ 0.1 \% \\
\hline %---------------------------------------------------------
%\hline %---------------------------------------------------------
\end{tabular}
\caption{Summary of the theory systematic uncertainties in the $\Hllll$ measurements for the inclusive analysis.
\label{tab:SystOverviewTheo}
}
\normalsize
\end{center}
\end{table}
 
\section{Limit setting}

The primary tool used to interpret the analysis described above in the context of the signal models is the HiggsAnalysis-CombinedLimit package \cite{combinetwiki}, a collection of RooStats-based software \cite{roostatstwiki} used within the Higgs physics analysis group (PAG) \cite{higgspagtwiki}, hereafter called the combine tool, or simply "combine". Specifically, one-sided Bayesian credible interval limits are set on the expected and observed signal production cross section times branching ratio (BR) of $\HZZfl$ ($\sigma\times \rm{BR}$). Limits are set on the various signal benchmarks using the asymptotic $\rm{CL}_S$ method \cite{Cowan:2010js}, an approach to calculating a profile likelihood ratio using an approximation of the LHC test-statistic distributions. The upper limit on the cross section gives the maximum number of events that can be attributed to the signal process, consistent with the data that is observed. 

Combine finds the upper limit as the numerical solution to Equation~\ref{eq:pCL}, which sets the integral of the posterior probability $p(\sigma|D)d\sigma$ equal to the desired confidence level for the measurement, typically 95\%. The posterior probability gives the degree of belief that $\sigma$ lies in the interval $[\sigma,\ \sigma+d\sigma]$ and is formed by inverting a multi-Poisson model, $p(D|\sigma,\ \bm{\theta})$ using Bayes' theorem (Equation~\ref{eq:pbayes}), after numerically marginalizing priors describing the uncertainties $\pi(\bm{\theta})$ (Equation~\ref{eq:pmarg}).

\begin{equation}\label{eq:pCL}
\int_0^{\sigma_{\rm{Up}}} p(\sigma|D)d\sigma = 1 - \alpha
\end{equation}
where,
\begin{equation}\label{eq:pbayes}
p(\sigma|D) = p(D|\sigma)\pi(\sigma) / p(D)
\end{equation}
and,
\begin{equation}\label{eq:pmarg}
p(D|\sigma) = \int p(D|\sigma,\ \bm{\theta}) \pi(\bm{\theta}) d\bm{\theta}.
\end{equation}

Combine takes specially formatted text files called data cards as input. The data cards contain the signal, background, and data yields, along with associated systematic uncertainties, and the location of the file containing the shape distributions for each sample. When the shape distributions are included, combine computes the limits accross all bins, then combines the results. A non-shape-based limit is equivalent to a shape-based limit with one bin. This can be used as a crosscheck that the shape-based limit is behaving correctly and to analyze the improvement of using the shape-based approach over non-shape-based.

Combine returns the 95\% confidence level (CL) expected and observed limits on the signal strength parameter $\mu$, as well as one and two standard deviations of the expected limit, where $\mu$ is the signal scale factor in the $\rm{i}^{\rm{th}}$ bin in the mean count, $n_i = \mu * s_i + b_i$, where $s_i$ and $b_i$ are the signal and background yields, respectively.
%Limits are set for the three decay channels, 4e, 4$\mu$, and 2e2$\mu$ individually, then combined for the 4l limit. 
The signal strength parameter gives the ratio of the 95\% CL expected signal yield to the theoretical yield, or more generally:

\begin{equation}\label{eq:mu}
\mu[\sigma_{\rm{norm}}*\rm{BR}] = \frac{\sigma_{95\% \rm{CL}}*\rm{BR}}{\sigma_{\rm{norm}}*\rm{BR}},
\end{equation}
where $\sigma_{\rm{norm}}$ is the cross section used to normalize the signal yields given as input to combine, typically the theoretical production cross section, or set to 1 pb for comparison to other $H$ decay channels. This formula is used to calculate $\sigma_{95\% \rm{CL}}\times \rm{BR}$, which is the desired output variable. This limit can then be compared to the theoretical $\sigma\times \rm{BR}$. The points where the upper limit is lower than the theory value are interpreted as exclusions.


\subsection{Cross section limits}

After the event selection is optimized and all of the uncertainties are accounted for, limits are set on the signal strength parameter $\mu$, which is scaled according to Equation~\ref{eq:mu} to obtain the cross section limit. For each model, these results are found for both the cut-and-count and MVA-based strategies. For Zp2HDM, the one-dimensional slice of mass points fixing $m_{A^0} = 300$ $\GeV$ is selected for the limit plots since this region has the highest cross section and the largest branching fraction to DM production. For ZpBaryonic, the one-dimensional slice of mass points fixing $m_{\chi} = 1$ $\GeV$ is selected for the limit plots since this region has the highest cross section.

\subsubsection{Cut-and-count based limits}

The one-dimensional limits, obtained using the optimized selection given in Section~\ref{cutandcountopt}, are presented for $\sigma_{95\% \rm{CL}}\times \rm{BR}$ in Figure~\ref{fig:limzp2hdm} for Zp2HDM and Figure~\ref{fig:limzpbaryonic} for ZpBaryonic. In both models, the observed limits do not deviate more than one standard deviation from the expected limits, indicating the absence of a large excess of events in the SR above the SM prediction. Since the limit curves lie above the theoretical cross section curves, no benchmark points can be excluded using this analysis alone. However, when the Zp2HDM results are combined with those from the other $H$ decay channels, there is sufficient sensitivity to exclude a large portion of the parameter space, up to mediator masses in the $\TeV$ range. These are the first results obtained for ZpBaryonic from any of the mono-$H$ analyses. Since the ZpBaryonic theoretical cross section curve is flat up to about 1 $\TeV$, any small increase in sensitivity by combining with other analyses will allow for the exclusion of these benchmark models. 

The limiting factor in improving the sensitivity of the limits obtained with this strategy is the signal efficiency. The cross section limit is directly proportional to the signal selection efficiency, and each additional cut reduces the signal efficiency by a few percent or more. From Table~\ref{tab:smyields}, the largest fraction of signal events is lost during the tight lepton and $Z$ selection steps, even before the mono-$H$ selection is applied. One method of improving the sensitivity of this analysis is to loosen the tight lepton selection criteria, but allowing more signal to pass a looser cut would also let more background pass. However, since MET is such a powerful discriminating variable, the additional backgrounds could potentially be reduced by a harder MET cut. Defining a new working point for the selection of tight leptons would require remeasuring lepton systematics and background estimates from data and the validation of modeling in additional CRs. This analysis is left to a future study.

\begin{figure}[tbh]
\centering
\includegraphics[width=6in]{figures/sigma_limits_4mu_Zp2HDM.png}
\caption{One-dimensional cross section times branching fraction limits for the Zp2HDM simplified model using the cut-and-count based event selection strategy.}
\label{fig:limzp2hdm}
\end{figure}

\begin{figure}[tbh]
\centering
\includegraphics[width=6in]{figures/sigma_limits_4mu_ZpBaryonic.png}
\caption{One-dimensional cross section times branching fraction limits for the ZpBaryonic simplified model using the cut-and-count based event selection strategy.}
\label{fig:limzpbaryonic}
\end{figure}

\subsubsection{MVA-based limits}

The one-dimensional limits, obtained using the optimized selection given in Section~\ref{mvaopt}, are presented for $\sigma_{95\% \rm{CL}}\times \rm{BR}$ in Figure~\ref{fig:limzp2hdmmva} for Zp2HDM and Figure~\ref{fig:limzpbaryonicmva} for ZpBaryonic. The sensitivity obtained using the MVA approach is not as good as for the cut-and-count approach. This reiterates the strong discriminating power of MET used in the cut-and-count strategy, rather than indicating a poor performance of the MVA strategy. Again, the signal efficiency is the limiting factor in improving the sensitivity. Since there are no additional cuts beyond the SM selection in this selection strategy, the SM selection itself would need to be modified, requiring additional study. 

\begin{figure}[tbh]
\centering
\includegraphics[width=6in]{figures/sigma_limits_4mu_Zp2HDM_MVA.png}
\caption{One-dimensional cross section times branching fraction limits for the Zp2HDM simplified model using the MVA-based event selection strategy.}
\label{fig:limzp2hdmmva}
\end{figure}

\begin{figure}[tbh]
\centering
\includegraphics[width=6in]{figures/sigma_limits_4mu_ZpBaryonic_MVA.png}
\caption{One-dimensional cross section times branching fraction limits for the ZpBaryonic simplified model using the MVA-based event selection strategy.}
\label{fig:limzpbaryonicmva}
\end{figure}

