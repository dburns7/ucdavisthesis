\chapter{Motivation and theoretical background}

The first section of this chapter lays the theoretical framework of the Standard Model (SM) of particle physics, including the historical discovery timeline of symmetries of nature, each corresponding to a leap forward in our physical understanding, and the model's particle content and their properties. The second section extends the symmetry group of the SM to include candidates for the observed cosmic dark matter (DM) and the particles that could mediate the DM-SM interactions.

\section{The Standard Model}

\subsection{Symmetries of nature}

Each major advance in the history of physics has corresponded to the discovery and mathematical implementation of a new global space-time, global discrete, or local gauge symmetry. In this section, I will review these discoveries, informally developing the mathematical framework needed to understand the symmetry groups of the SM. Along the way, two important subplots will play out: the development of our understanding of physics at smaller distance scales and higher energies, and the unification of previously separate physical sectors.

\indent Our ancient ancestors were aware that certain geometrical shapes, e.g. the Platonic solids, possessed the quality of symmetry, and were driven to understand the composition of physical substances by breaking them down into fundamental, indivisible units. These units, known as atoms by the ancient Greeks, interact and rearrange themselves according to physical laws to account for the variety of substances and physical phenomena we observe. The attraction to applying symmetry to nature is evidenced by the centuries-long belief that the Earth lie at the center of the universe, with the celestial bodies orbiting in perfect, divine circles.

\indent The end of the scientifically repressive middle ages brought along an improvement in astronomical observations and the growth of the pseudo-scientific field of alchemy, which attempted to reduce, understand, and manipulate the fundamental elements of physical substances. When Kepler discovered three laws of planetary motion, he unified the description of the motion of the planets. For the first time, the conservation of a physical quantity, what we now know as angular momentum, was associated with a general set of physical objects. At the burgeoning of the scientific revolution, Newton championed the idea that the same physical laws can be applied to all physical events, and that properties of these laws can be abstracted to apply to nature at a fundamental level.

\indent Newton defined an inertial reference frame implicitly as one where his first law held, that is, that an object remains at constant motion unless acted on by an outside force. Since his laws were the same in all inertial reference frames, a new symmetry of nature was discovered, now called symmetry under Euclidean transformations. Since Euclidean transformations form a mathematical group, it is said that classical mechanics is invariant under the Euclidean group. The invariance under the Euclidean transformations can be used to derive conservation laws: Newton's laws don't depend on spatial translations or rotations, implying the conservation of linear and angular momentum, respectively. These relationships foreshadow Noether's abstraction of the connection between symmetries and conserved quantities. She proved that there is a conserved quantity, or current, associated with every symmetry of a physical system. This famous theorem facilitates the derivation of conserved quantities and will be used extensively in the theories that follow. The extension of the Euclidean transformations to include time translations and motion at constant velocity (boosts) forms the Galilean group.

\indent The next symmetry of nature to be discovered came when Lorentz found that Maxwell's equations, which unified the classical eletricity and magnetism sectors, were invariant under a set of transformations that generalized the classical Galiliean translations, now called Lorentz transformations, which form the Lorentz group. Einstein

\indent Maxwell's equations unify EM, invariant under Lorentz group, conservation of charge/currents/etc? 

\indent Einstein developes special relativity, deriving Lorentz group as fundamental symmetry of nature

\indent Euclidean group plus Lorentz group is Poincare group, global space-time symmetry of SM. Conservation of ? 

\indent Quantum theory developed to describe both bosons and fermions, with respective statistics and permutational invariance

\indent Weyl applies gauge invariance in quantum theory, electric charge conserved

\indent Dirac incorporates relativity in quantum theory of spin 1/2 particles. Invariant under Poincare group, predicts antiparticles discovered in CR by Carl Anderson

\indent T, C, P, PCT, P breaking observed by Wu, Yang, Lee, CP breaking obseved by Cronin, Fitch

\indent Spontaneous symmetry breaking, Higgs mechanism

\indent Electroweak unification

\indent Gell-mann Neeman SU(3) symmetry from strong force
 
\indent Standard model

\subsection{Particle content}



\section{Dark matter}

\subsection{Observational evidence}

\subsection{Beyond the Standard Model}

\cite{Carpenter:2013xra}
\cite{Gibaldi:80}
