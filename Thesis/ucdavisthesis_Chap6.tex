\chapter{Event Selection}

\section{Trigger Selection}
\label{sec:HLTsel}

The events are required to have fired the High-Level Trigger paths described in section~\ref{sec:trigpaths}. Unlike in the Run I analysis, the trigger requirement does not depend on the selected final state: it is always the OR of all 10 HLT paths. The reason is in Run II we will be targeting associated production modes that can come with additional leptons, thus improving trigger efficiency further.


\section{Vertex Selection}
\label{sec:vertexsel}

The events are required to have at least one good primary
vertex (PV) fullfilling the following criteria: high number of degree
of freedom ($N_{PV}>4$), collisions restricted along the $z-$axis
($z_{PV}<24$ cm) and small radius of the PV ($r_{PV}<2$ cm).


\section{ZZ Candidate Selection}
\label{sec:zzcandsel}

The four-lepton candidates are built from what we call {\bf selected leptons}, which  
are the tight leptons (defined in sections~\ref{sec:eleID} and~\ref{sec:muonReco}) that pass the ${\rm SIP_{3D}} < 4$ vertex constraint
and the isolation cuts (defined in sections~\ref{sec:eleiso} and~\ref{sec:muoniso}), 
where FSR photons are subtracted as described in Section~\ref{sec:FSRphotons}.
A lepton cross cleaning is applied 
by discarding electrons which are within $\Delta R < 0.05$ of selected muons. 


The construction and selection of four-lepton candidates proceeds 
according to the following sequence:
\begin{enumerate}
\item {\bf Z candidates} are defined as pairs of selected leptons
 of opposite charge and matching flavour ($e^+ e^-$, \, $\mu^+\mu^-$)
 that satisfy $12 < m_{\ell\ell(\gamma)} < 120~\GeV$, where the Z candidate mass
 includes the selected FSR photons if any.
\item {\bf ZZ candidates} are defined as pairs of non-overlapping Z candidates.
 The Z candidate with reconstructed mass $m_{\ell\ell}$ closest to the nominal Z boson
 mass is denoted as ${\rm Z_1}$, and the second one is denoted as ${\rm Z_2}$.
 ZZ candidates are required to satisfy the following list of requirements:
  \begin{itemize} 
  \item {\bf Ghost removal }: $\Delta R(\eta,\phi) > 0.02$ between each of the four leptons.
  \item {\bf lepton $p_T$}: Two of the four selected leptons should pass 
     $p_{T,i} > 20~\GeV$ and $p_{T,j} > 10~\GeV$.
  \item {\bf QCD suppression}: all four opposite-sign pairs that can
     be built with the four leptons (regardless of lepton flavor)
     must satisfy $m_{\ell\ell} > 4~\GeV$.
     Here, selected FSR photons are not used in computing $m_{\ell\ell}$, 
     since a QCD-induced low mass dilepton (eg. $J/\Psi$) 
     may have photons nearby (e.g. from $\pi_0$). 
  \item {\bf ${\rm Z_1}$ mass}: $m_{\rm Z1} > 40~\GeV$
  \item {\bf 'smart cut'}: defining ${\rm Z_a}$ and ${\rm Z_b}$ as 
     the mass-sorted alternative pairing Z candidates 
     (${\rm Z_a}$ being the one closest to the nominal Z boson mass),
     require NOT($|m_{\rm Za}-m_{\rm Z}| < |m_{\rm Z1}-m_{\rm Z}|$ AND $m_{\rm Zb}<12$).
     Selected FSR photons are included in $m_{\rm Z}$'s computations.
     This cut discards $4\mu$ and $4e$ candidates where the alternative pairing
     looks like an on-shell Z + low-mass $\ell^+ \ell^-$. 
     (NB. In Run I, such a situation was avoided by choosing the best ZZ candidate
     before applying kinematic cuts to it, most precisely before the $m_{\rm Z2} > 12~\GeV$ cut.
     The present smart cut allows to choose the best ZZ candidate after all kinematic cuts.)   
  \item {\bf four-lepton invariant mass}: $\mllll > 70~\GeV$
  \end{itemize}	
\item Events containing at least one selected ZZ candidate form the {\bf SM signal region}.
\end{enumerate}	


\section{Choice of the best ZZ Candidate}
\label{sec:zzbestcand}

Unlike in the Run I analysis, the best ZZ candidate is now chosen after all kinematic cuts, a change that allows to test other selection strategies for this candidate choice. 
This is especially relevant for events with more than four selected leptons, having in mind the search for the Higgs boson production modes where associated particles can decay to leptons, such as VH and ttH.

For the current analysis, we adopt a different approach compared to Run 1: if more than one ZZ candidate survives the above selection,
we choose the one with the highest value of $\KD$ (defined in Section~\ref{sec:observables}, except if
two candidates are composed of them four leptons in which case the candidate with ${\rm Z_1}$ closest in mass to nominal 
Z boson mass is chosen.

